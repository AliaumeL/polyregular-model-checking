%! TEX program = xelatex
% WARNING: this is a generated file.
%
% Please do not edit this file directly. 
% - If you want to update the medatata of the paper (title, authors, abstract), please
%   edit the `paper-meta.yaml` file in the root of the repository.
% - If you want to update the content of the paper, please edit the latex files
%   in the `src` directory.
% - If you want to update the template itself (e.g., change the layout), please
%   edit the `templates/plain-article.tex` file instead.
\documentclass[11pt,a4paper,twosided]{article}

% we setup a custom geometry because the default one is too narrow
\usepackage{geometry}
\geometry{margin=3.5cm}

% we include whatever the user wants to include in the header

% we include libraries (tex files) usually written in the `lib` directory
% utf-8 for old systems
\usepackage[utf8]{inputenc}
\usepackage[T1]{fontenc}

% babel for language settings
\usepackage[english]{babel}

% microtype for better typography
\usepackage{microtype}


% math packages
\usepackage{amsmath,amsthm,amssymb,stmaryrd,thmtools,upgreek}
% configure some theorems
\newtheorem{theorem}{Theorem}
\newtheorem{lemma}[theorem]{Lemma}
\newtheorem{corollary}[theorem]{Corollary}
\newtheorem{proposition}[theorem]{Proposition}
\newtheorem{conjecture}[theorem]{Conjecture}
\theoremstyle{definition}
\newtheorem{definition}[theorem]{Definition}
\newtheorem{remark}[theorem]{Remark}
\newtheorem{example}[theorem]{Example}


% graphics packages
\usepackage{graphicx}
\usepackage[obeyclassoptions,mode=tex]{standalone}
\usepackage{tikz}
\usetikzlibrary{backgrounds}
\usetikzlibrary{shapes.geometric}
\usetikzlibrary{positioning}
\usetikzlibrary{automata}
\usetikzlibrary{tikzmark}
\usetikzlibrary{patterns}
\usetikzlibrary{arrows}
\tikzset{every state/.style={minimum size=1pt}}
\usepackage{tikz-cd}


% links inside the document
\usepackage{hyperref}
\usepackage[capitalise,noabbrev,nameinlink]{cleveref}
\usepackage[composition,hyperref,xcolor,cleveref]{knowledge}
\knowledgeconfigure{notion}

% Tables 
\usepackage{booktabs}
\usepackage{varwidth}

% Packages for macro definitions
\usepackage{xparse}
\usepackage{xpatch}
\usepackage{tokcycle}
\usepackage{ifthen}

% Colors 
\usepackage{ensps-colorscheme}

% Knowledge logo
\newcommand{\klogo}{%
\begin{tikzpicture}[scale=0.2,line/.style={draw, line width=0.2pt, line cap=round, line join=round}]
\coordinate (A00) at (0,0);
\coordinate (A01) at (0,1);
\coordinate (A10) at (1,0);
\coordinate (B10) at (1,0.2);
\coordinate (B01) at (0.2,1);

\coordinate (C01) at (0.4,0.7);
\coordinate (C10) at (0.7,0.4);
\coordinate (C12) at (0.4,1.2);
\coordinate (C21) at (1.2, 0.4);
\coordinate (C22) at (1.2, 1.2);

\coordinate (D00) at (C10);
\coordinate (D01) at (0.8,0.5);
\coordinate (D10) at (0.8,0.3);

\coordinate (E01) at (0.3,0.7);
\coordinate (E10) at (0.5,0.7);

\draw[line] (B01) -- (A01) -- (A00) -- (A10) -- (B10);
\draw[line] (C01) -- (C12) -- (C22) -- (C21) -- (C10);

\draw[line] (D01) -- (D00) -- (D10);
\draw[line] (E01) -- (E10);

\end{tikzpicture}%
}




% Upgreek letters
\makeatletter
\newcommand\mathgr[1]{\tokcycle
  {\addcytoks{##1}}
  {\processtoks{##1}}
  {\ifcsname up\expandafter\@gobble\string##1\endcsname
   \addcytoks[1]{\csname up\expandafter\@gobble\string##1\endcsname}%
    \else\addcytoks{##1}\fi}
  {\addcytoks{##1}}{#1}%
  \expandafter\mathrm\expandafter{\the\cytoks}%
}
\makeatother



% Create a new macro proofof
% taking as input a label of a theorem
% and creating a proof with a reference to that
% label
\NewDocumentEnvironment{proofof}{ m O{appendix} }{
    % if the command \#1 exists, then 
    % call \#1* to restate the theorem
    \ifcsname #1\endcsname
        \def\isInsideRestatedTheorem{1}
        \csname #1\endcsname*
    \fi
    \begin{proof}[Proof of {\cref{#1}} as stated on page {\pageref{#1}}]
        \phantomsection
        \label{#1:proof}
}{
        % if the optional argument is "appendix" 
        % then printout a "backlink"
        % and otherwise do nothing
        \ifthenelse{\equal{#2}{appendix}}{
        % Some link to go back to the theorem
        \marginpar{\vspace{-2em}\texttt{\small{\hyperref[#1]{$\triangleright$ Back to p.\pageref{#1}}}}}
        }{}
    \end{proof}
}

% Create a new macro proofref
% that takes as input a label of a theorem
% and creates a reference to its proof
\NewDocumentCommand{\proofref}{ m }{
    % checks if the label #1:proof exists, if yes
    % it creates a link to it, otherwise it writes nothing
    \IfRefUndefinedExpandable{#1:proof}{}{
        % Checks if we are inside a restated theorem
        % if yes, we do not print anything
        \ifdefined\isInsideRestatedTheorem
        \else
            \marginpar{\vspace{0.6em}\texttt{\small{\hyperref[#1:proof]{$\triangleright$ Proven p.\pageref{#1:proof}}}}}
        \fi
    }
}



% Automate the creation of new orderings
% based on a given symbol.
% For instance,
% \NewDocumentOrdering{\pref}{\preceq}{\prec}
% will create the following commands:
% \prefleq and \preflt
% that will respectively expand to
% \mathrel{\kl[\pref]{\preceq}} and \mathrel{\kl[\pref]{\prec}}
\NewDocumentCommand{\NewDocumentOrdering}{ m m m }{
    \expandafter\newcommand\csname #1leq\endcsname{
        \mathrel{\kl[#1]{#2}}
    }
    \expandafter\newcommand\csname #1lt\endcsname{
        \mathrel{\kl[#1]{#3}}
    }
    \knowledge{#1}{notion}
}

% Little math macros
\NewDocumentCommand{\set}{ m }{\{ #1 \}}
\NewDocumentCommand{\setof}{ m m }{\{ #1 \mid #2 \}}
\NewDocumentCommand{\card}{ m }{\left| #1 \right|}
\NewDocumentCommand{\Nat}{ }{\mathbb{N}}
\NewDocumentCommand{\Rel}{ }{\mathbb{Z}}
\NewDocumentCommand{\seqof}{ m O{n \in \Nat} }{\left( #1 \right)_{#2}}
\NewDocumentCommand{\defined}{ }{\triangleq}

\NewDocumentCommand{\range}{ O{1} m }{[#1, #2]}

% Order macros
\NewDocumentCommand{\upset}{ O{} m }{{\uparrow_{#1} #2}}
\NewDocumentCommand{\dwset}{ O{} m }{{\downarrow_{#1} #2}}


% Number theory
\NewDocumentCommand{\factorial}{ O{} m }{
    \if\relax\detokenize{#1}\relax
        #2!
    \else
        (#2)!
    \fi
}


\newcommand{\circled}[2]{%
\hypertarget{#2}{}%
\tikz[baseline=(char.base),color=A2,thick]{%
\node[shape=circle,draw,inner sep=1pt,font=\tiny] (char) {#1};%
}}
\newcommand{\circleref}[2]{%
\hyperlink{#2}{%
\tikz[baseline=(char.base),color=A2,thick]{%
\node[shape=circle,draw,inner sep=1pt,font=\tiny] (char) {#1};}%
}}

% Classical sets



% Variables
\NewDocumentCommand{\PVars}{}{\kl[\PVars]{\mathbb{V}_{\text{pos}}}}
\knowledge{\PVars}{notion}

\NewDocumentCommand{\FunVars}{}{\kl[\FunVars]{\mathbb{V}_{\text{fun}}}}
\knowledge{\FunVars}{notion}

\NewDocumentCommand{\BVars}{}{\kl[\BVars]{\mathbb{V}_{\text{bool}}}}
\knowledge{\BVars}{notion}

\NewDocumentCommand{\OVars}{}{\kl[\OVars]{\mathbb{V}_{\text{list}}}}
\knowledge{\OVars}{notion}

% Basic types
\NewDocumentCommand{\Bools}{}{\kl[\Bools]{\mathbb{B}}}
\knowledge{\Bools}{notion}

\NewDocumentCommand{\OutputType}{O{\Sigma}}{\kl[\OutputType]{\mathcal{C}_{#1}}}
\knowledge{\OutputType}{notion}

% Language constructs
\NewDocumentCommand{\hlprogram}{}{\kl[\hlprogram]{\mathsf{Program}}}
\knowledge{\hlprogram}{notion}

\NewDocumentCommand{\hlfun}{}{\kl[\hlfun]{\mathsf{Fun}}}
\knowledge{\hlfun}{notion}

\NewDocumentCommand{\hlstmt}{}{\kl[\hlstmt]{\mathsf{Stmt}}}
\knowledge{\hlstmt}{notion}

\NewDocumentCommand{\bexpr}{}{\kl[\bexpr]{\mathsf{BExpr}}}
\knowledge{\bexpr}{notion}

\NewDocumentCommand{\oexpr}{}{\kl[\oexpr]{\mathsf{OExpr}}}
\knowledge{\oexpr}{notion}

\NewDocumentCommand{\cexpr}{}{\kl[\cexpr]{\mathsf{CExpr}}}
\knowledge{\cexpr}{notion}

\NewDocumentCommand{\aexpr}{}{\kl[\aexpr]{\mathsf{AExpr}}}
\knowledge{\aexpr}{notion}

% Operators
\NewDocumentCommand{\bBinOp}{}{\kl[\bBinOp]{\mathsf{BBin}}}
\knowledge{\bBinOp}{notion}

\NewDocumentCommand{\pCmpOp}{}{\kl[\pCmpOp]{\mathsf{PComp}}}
\knowledge{\pCmpOp}{notion}

% Control flow constructs
\NewDocumentCommand{\hlif}{m m m}{\withkl{\kl[\hlif]}{\cmdkl{\mathsf{if}} \; #1 \; \cmdkl{\mathsf{then}} \; #2 \; \cmdkl{\mathsf{else}} \; #3}}
\knowledge{\hlif}{notion}

\NewDocumentCommand{\hlyield}{m}{\withkl{\kl[\hlyield]}{\cmdkl{\mathsf{yield}} \; #1}}
\knowledge{\hlyield}{notion}

\NewDocumentCommand{\hlreturn}{m}{\withkl{\kl[\hlreturn]}{\cmdkl{\mathsf{return}} \; #1}}
\knowledge{\hlreturn}{notion}

\NewDocumentCommand{\hlletoutput}{m m m}{\withkl{\kl[\hlletoutput]}{\mathsf{let} \; #1 = #2 \; \mathsf{in} \; #3}}
\knowledge{\hlletoutput}{notion}

\NewDocumentCommand{\hlletboolean}{m m}{\withkl{\kl[\hlletboolean]}{\cmdkl{\mathsf{let~mut}} \; #1\cmdkl{~=~\bfalse \; \mathsf{in}} \; #2}}
\knowledge{\hlletboolean}{notion}

\NewDocumentCommand{\hlsettrue}{m}{\withkl{\kl[\hlsettrue]}{\; #1 \; \cmdkl{\leftarrow} \; \cmdkl{\mathsf{true}}}}
\knowledge{\hlsettrue}{notion}

\NewDocumentCommand{\hlfor}{m m m}{\withkl{\kl[\hlfor]}{\cmdkl{\mathsf{for}^{\rightarrow}} \; #1 \; \cmdkl{\mathsf{in}} \; #2 \; \cmdkl{\mathsf{do}} \; #3}}
\knowledge{\hlfor}{notion}

\NewDocumentCommand{\hlforRev}{m m m}{\withkl{\kl[\hlforRev]}{\cmdkl{\mathsf{for}^{\leftarrow}} \; #1 \; \cmdkl{\mathsf{in}} \; #2 \; \cmdkl{\mathsf{do}} \; #3}}
\knowledge{\hlforRev}{notion}

\NewDocumentCommand{\hlseq}{m m}{\withkl{\kl[\hlseq]}{#1 \; \cmdkl{;} \; #2}}
\knowledge{\hlseq}{notion}

% Boolean operations
\NewDocumentCommand{\btrue}{}{\kl[\btrue]{\mathsf{true}}}
\knowledge{\btrue}{notion}

\NewDocumentCommand{\bfalse}{}{\kl[\bfalse]{\mathsf{false}}}
\knowledge{\bfalse}{notion}

\NewDocumentCommand{\bnot}{m}{\withkl{\kl[\bnot]}{\cmdkl{\neg} #1}}
\knowledge{\bnot}{notion}

\NewDocumentCommand{\bbin}{ m m m }{\withkl{\kl[\bbin]}{#1 \; #2 \; #3}}
\knowledge{\bbin}{notion}

\NewDocumentCommand{\bcomp}{m m m}{\withkl{\kl[\bcomp]}{#1 \; #2 \; #3}}
\knowledge{\bcomp}{notion}

\NewDocumentCommand{\bapp}{m m}{\withkl{\kl[\bapp]}{\mathop{#1}(#2)}}
\knowledge{\bapp}{notion}

\NewDocumentCommand{\bliteq}{m m}{\withkl{\kl[\bliteq]}{#1 \; \cmdkl{=_{\text{lit}}} \; #2}}
\knowledge{\bliteq}{notion}

\NewDocumentCommand{\bgen}{m}{\withkl{\kl[\bgen]}{\mathopen{\cmdkl{\langle}} #1 \mathclose{\cmdkl{\rangle}}}}
\knowledge{\bgen}{notion}

% Output operations
\NewDocumentCommand{\ogen}{m}{\withkl{\kl[\ogen]}{\langle #1 \rangle}}
\knowledge{\ogen}{notion}

\NewDocumentCommand{\cchar}{m}{\withkl{\kl[\cchar]}{\mathsf{char}(#1)}}
\knowledge{\cchar}{notion}

\NewDocumentCommand{\clist}{m}{\withkl{\kl[\clist]}{\mathsf{list}(#1)}}
\knowledge{\clist}{notion}

\NewDocumentCommand{\olist}{m}{\withkl{\kl[\olist]}{\mathsf{list}(#1)}}
\knowledge{\olist}{notion}

\NewDocumentCommand{\hlfundef}{m m m}{\withkl{\kl[\hlfundef]}{\mathsf{def} \; #1(#2) \; \{ #3 \}}}
\knowledge{\hlfundef}{notion}

% Types
\NewDocumentCommand{\TPos}{O{} }{\kl[\TPos]{\mathsf{Pos}_{#1}}}
\knowledge{\TPos}{notion}

\NewDocumentCommand{\TBool}{}{\kl[\TBool]{\mathsf{Bool}}}
\knowledge{\TBool}{notion}

\NewDocumentCommand{\TUnit}{}{\kl[\TUnit]{\mathsf{Any}}}
\knowledge{\TUnit}{notion}

\NewDocumentCommand{\TOut}{O{}}{\kl[\TOut]{\mathsf{List}_{#1}}}
\knowledge{\TOut}{notion}


% Semantics

\NewDocumentCommand{\semB}{ m m }{ \llbracket #1 \rrbracket_{#2} }
\NewDocumentCommand{\semS}{ m m }{ \llbracket #1 \rrbracket_{#2} }
\NewDocumentCommand{\semO}{ m m }{ \llbracket #1 \rrbracket_{#2} }
\NewDocumentCommand{\semC}{ m }{ \llbracket #1 \rrbracket }
\NewDocumentCommand{\semF}{ m m }{ \llbracket #1 \rrbracket_{#2} }
\NewDocumentCommand{\semT}{ m }{ \llbracket #1 \rrbracket }


% We include the title and author information based on the 
% `paper-meta.yaml` file.
 
\title{Polyregular Model Checking}

\author{
Aliaume Lopez\thanks{University of Warsaw, Poland}
 \and
Rafał Stefański\thanks{University of Warsaw, Poland}
}

% For the date, we first check if the user has provided a date,
% and otherwise use the git meta inforamtion (if available).
\date{2024-09-03 12:59:43
+0200 -- 0c52ab1e2303f5d2959eba5a98264a411874675d\footnote{branch main at git@github.com:AliaumeL/polyregular-model-checking.git}}


% Now, we create the document itself.
\begin{document}
% Generate the title page
\maketitle
% Print the abstract
\begin{abstract}
    In this paper, we demonstrate how the theory of polyregular
    functions can be leveraged to verify simple string-to-string
    for-loop programs defined in a subset of Python in an efficient way.
\end{abstract}

% Include the content of the paper
% LTeX: language=EN
\section{Introduction}
\label{sec:intro}

% In this paper, we introduce a programming language that is expressive
% enough to capture a wide range of \emph{string-to-string} functions, and yet
% simple enough to be amenable to formal verification. Specifically, we will 
% show that one can automatically verify Hoare triples of the form
% $\hoaretriple{P}{\texttt{code}}{Q}$, where $P$ and $Q$ are pre- and post-conditions 
% written in the first order logic on words, meaning that whenever the input satisfies property
% $P$, the output of the program satisfies property $Q$.



String manipulating programs of low complexity are ubiquitous in modern
software. They are often used to transform data and do not perform complex
computations themselves. In this paper, we are interested in verifying
\kl{Hoare triples} for such string manipulating programs, i.e.
specifications of the form $\hoaretriple{P}{\texttt{code}}{Q}$, where $P$ and $Q$ are pre- and
post-conditions, meaning that whenever the input satisfies property $P$, the
output of the program satisfies property $Q$.

\paragraph{Regularity preserving programs.} \AP One particularly interesting
class of specifications in the case of string-to-string functions are
\emph{regular languages}, which can be efficiently verified using
automata-based techniques. We say that a function $f$ is \intro{regularity
preserving} if it preserves regular languages under pre-image, i.e. if
$f^{-1}(L)$ is regular for all regular languages $L$. For \kl{regularity
preserving functions}, the verification of a Hoare triple
$\hoaretriple{L_P}{f}{L_Q}$ can be reduced to the nonemptiness problem of the
language $L_P \cap f^{-1}(L_Q)$, where $L_P$ and $L_Q$ are regular languages.
This is a well-studied problem in the literature, and is at the core of several
more involved techniques \cite{ALCE11,CHLRW19,JLMR23}. The key challenge of
this approach is that there exist uncomputable \kl{regularity preserving
functions}, so such approaches will only work on classes of functions for which
pre-images of regular languages are (relatively) efficiently computable.
Usually, these classes come from generalisation of automata models to
functions, also known as \emph{string-to-string transducers}.

\paragraph{String-to-string transducer models.} There is a wide variety of
models for string-to-string transducers \cite{MUSC19}, and one of the most
prominent ones is called \intro{linear regular functions}, that are
equivalently defined using two-way finite transducers (2DFTs)
\cite{RASCO59}, streaming-string-transducers (SSTs) \cite{ALUR11}, or linear
regular list functions \cite{BDK18}. Notably, Alur and Černý have proven
that SSTs have a \PSPACE-complete model checking problem when the functions are
given as SSTs, and the specifications are given as automata
\cite[Theorem 13]{ALCE11}. This was used for instance by Chen, Taolue, Hague,
Lin, Rümer, and Wu to study \emph{path feasibility} in string-manipulating
programs \cite{CHLRW19}.

A similar approach was used by Jeż, Lin, Markgraf, and Rümmer,
who leveraged the \emph{rational functions} (a strict subclass of \kl{linear regular functions}) 
to study programs manipulating strings with infinite alphabets (see Kaminski&Francez1994). 
Remark that in the setting of ininifte alphabets the landscape of automata and transdcuers
is much more complex: In partcicular, the class of languages recognised by two-way automata is 
stronger that the class of languages recognised by one-way automata, and has undecidable emptiness
\cite[Figure 1.1]{BOJA19}.

\AP One limitation of the \kl{linear regular functions} is that they only 
allow for linear growth of the output, excluding many useful string manipulating 
programs. The class of \reintro{polyregular functions} is an interesting 
generalization of \kl{linear regular functions} that allows for polynomial behaviour, 
and is much closer to real life string manipulating programs. The model 
is relatively old, first introduced in \cite{ENMA02},
and has recently gained a lot of traction now that
several other characterizations have been obtained
\cite{bojanczyk2018polyregular,bojanczyk2019string}.\footnote{Note that for this extended model, being
  \kl{regularity preserving} is tightly connected to being closed under
function composition \cite[Proposition III.3]{FIRELH25}, and this closure under
composition was one of the surprising conclusions of
\cite{bojanczyk2018polyregular}}. However, the proof of the \kl{regularity
preserving} property for polyregular functions is of theoretical nature (no
implementation or complexity bounds are given), and writing programs using any
of the existing equivalent definitions of \kl{polyregular functions} is
cumbersome and error-prone. Because \kl{polyregular functions} can succinctly
encode formulas in first-order logic on words, and
since the satisfiability problem for such formulas is known to be
\TOWER-complete \cite[Theorem 13.5]{REINH02}, one can expect that verifying
\kl{polyregular functions} to be quite complex.

\paragraph{MSO vs FO.} \AP Instead of using the full power of regular languages
(defined equivalently using finite automata, monadic second order logic ($\mathsf{MSO}$), finite monoid
recognition, or regular expressions \cite{buchi1960weak,KLEE56,TRAK66,SCHU61}),
we will use specifications written in \kl{first-order logic} ($\FO$)
on finite words. A cornerstone result of the theory is establishing the
equivalence between languages described in this logic, \emph{star-free
languages}, and \emph{counter free automata} \cite{PEPI86,SCHU65,MNPA71}. The
advantage of using this weaker specification model is twofold: first, it allows
us to focus on a simpler class of \kl{star-free polyregular
functions}\footnote{ The notion of being \emph{star-free} has been extended to
various classes of transducers, see
\cite{CADA15,BDK18,MUSC19,bojanczyk2018polyregular}}, which are easier to work
with in practice. Second, it allows us to reduce the satisfiability of a Hoare
triple to the satisfiability of a \emph{first-order} formula on finite words,
for which one can use general purpose SMT solvers, in addition to automata 
based tools (\kl{MONA}) which also work for the $\MSO$ logic on words.
Even though the SMT solver are \emph{incomplete}, 
they can, in some cases, lead to faster decision procedures. Indeed, the
satisfiability problem for \kl{first-order logic on finite words}, while
decidable, is \TOWER-complete \cite[Theorem 13.5]{REINH02}. 

\paragraph{Contributions.} \AP In this paper, we introduce a high-level
programming language for implementing \kl{star-free polyregular functions}
in a Python-like syntax, including features such as boolean variables, index variables,
immutable list variables, function calls, and nested for-loops.
The language was carefully designed not become too expressive -- 
this is ensured by a number of syntactic restrictions and a novel type system
for index variables. We show that this language can be
compiled into one of the equivalent definitions of \kl{polyregular functions}
(namely, \kl{simple for-programs}), which does not allow for function calls nor
list variables. We also provide an implementation of the previously known
abstract result stating that \kl{polyregular functions} are \kl{regularity
preserving} (in the case of star-free functions and languages), being careful
about the complexity of the transformations. Finally, we reduce the
verification of Hoare triples to the satisfiability of first-order formulas on
words. Since we are using \kl{first-order logic} as a target language, we are
not restricted to using automata based tools like \intro{MONA} \cite{MONA01},
but can also employ general purpose SMT solvers like \intro{Z3} \cite{z3} and
\intro{CVC5} \cite{cvc5}, generating proof obligations in the \texttt{SMT-LIB}
format \cite{BARRETT17}.

All the steps described above have been implemented in a \texttt{Haskell}
program, and tested on a number of examples with encouraging
results.\footnote{An anonymized version of our code is available at
\repositoryUrl.} While this is not a tool paper, we believe that the
proof-of-concept implementation is a good starting point to demonstrate the viability of
our approach, and we believe that there is a potential for further
investigations in this direction.


\paragraph{Outline.} The structure of the paper is as follows. We introduce our
\kl{high-level language} in \cref{sec:high-level}. In \cref{sec:polyregular},
we recall the theory of \kl{polyregular functions} by introducing them in terms
of \kl{simple for-programs} and \kl{$\FO$-interpretations}. We will also
provide an efficient reduction of the verification of Hoare triples to the
satisfiability of a \kl{first-order formula on words} in \cref{sec:pullback}.
In order to verify \kl{for-programs}, we compile them into \kl{simple
for-programs} in \cref{sec:htl}, and then compile \kl{simple for-programs} into
\kl{$\FO$-interpretations} in \cref{sec:low-level}. 
%These steps 
%are summarized in the following diagram:
%\begin{center}
%    \begin{tikzpicture}[
%        syntaxNode/.style={
%                    rectangle, draw, 
%                    text width=5em, 
%                    text centered, 
%                    rounded corners, 
%                    minimum height=4em}
%    ]
%        %
%        % Write a tikz picture with nodes explaining the different 
%        % steps of the rewriting system 
%        % 
%        % (a) high-level language
%        % (b) simple for-programs
%        % (c) first-order string-to-string interpretations
%        % (c1) precondition
%        % (c2) postcondition
%        % (d) first-order formula
%        \node[syntaxNode] (a) {\kl{For-program}};
%        \node[syntaxNode, right=of a] (b)  {\kl{Simple for-program}};
%        \node[syntaxNode, right=of b] (c)  {\kl{$\FO$ interpretation}};
%        \node[syntaxNode, right=of c] (d) {\kl{First-order formula}};
%
%        \draw[->] (a) -- node[above,rotate=90,xshift=2.3em,yshift=-0.3em] {\cref{sec:htl}} (b);
%        \draw[->] (b) -- node[above,rotate=90,xshift=2.3em,yshift=-0.3em] {\cref{sec:low-level}} (c);
%        \draw[->] (c) -- node[above,rotate=90,xshift=2.7em,yshift=-0.3em] {\cref{sec:pullback}} (d);
%    \end{tikzpicture}
%\end{center}
Then, in \cref{sec:benchmarks}, we present
tests of our implementation on various examples, discussing
the complexity of the transformations and the main bottlenecks of our approach.
Finally, we conclude in \cref{sec:conclusion} by discussing potential
optimizations and future work.



% Include the bibliography
\bibliographystyle{plainurl}
\bibliography{papers.bib}

% If there are any appendices, we include them here.
\appendix
\section{Subwords Containing the Substring $ab$}

\begin{figure}
    \centering
\begin{Shaded}
\begin{Highlighting}[numbers=left]
\KeywordTok{def}\NormalTok{ getBetween( l : }\DataTypeTok{[Char]} \KeywordTok{with}\NormalTok{ (i,j) ) : }\DataTypeTok{[Char]}\NormalTok{ := }
\KeywordTok{    for}\NormalTok{ (k,c) }\KeywordTok{in} \KeywordTok{enumerate}\NormalTok{(l) }\KeywordTok{do}
        \KeywordTok{if}\NormalTok{ i \textless{}= k }\KeywordTok{and}\NormalTok{ k \textless{}= j }\KeywordTok{then}
            \KeywordTok{yield}\NormalTok{ c}
        \KeywordTok{endif}
    \KeywordTok{done}

\KeywordTok{def}\NormalTok{ containsAB(w : }\DataTypeTok{[Char]}\NormalTok{) : }\DataTypeTok{Bool}\NormalTok{ := }
    \KeywordTok{let} \KeywordTok{mut}\NormalTok{ seen\_a := }\KeywordTok{False} \KeywordTok{in} 
\KeywordTok{    for}\NormalTok{ (i, c) }\KeywordTok{in} \KeywordTok{enumerate}\NormalTok{(w) }\KeywordTok{do}
    \KeywordTok{if}\NormalTok{ c === }\StringTok{\textquotesingle{}a\textquotesingle{} }\KeywordTok{then}
\NormalTok{            seen\_a := }\KeywordTok{True}
    \KeywordTok{else} \KeywordTok{if}\NormalTok{ c === }\StringTok{\textquotesingle{}b\textquotesingle{} }\KeywordTok{and}\NormalTok{ seen\_a }\KeywordTok{then}
            \KeywordTok{return} \KeywordTok{True}
        \KeywordTok{endif} \KeywordTok{endif}
    \KeywordTok{done}
    \KeywordTok{return} \KeywordTok{False}

\KeywordTok{def}\NormalTok{ subwordsWithAB(w : }\DataTypeTok{[Char]}\NormalTok{) : }\DataTypeTok{[[Char]]}\NormalTok{ := }
\KeywordTok{    for}\NormalTok{ (i,c) }\KeywordTok{in} \KeywordTok{enumerate}\NormalTok{(w) }\KeywordTok{do}
\KeywordTok{        for}\NormalTok{ (j,d) }\KeywordTok{in} \KeywordTok{reversed}\NormalTok{(}\KeywordTok{enumerate}\NormalTok{(w)) }\KeywordTok{do}
            \KeywordTok{let}\NormalTok{ s := getBetween(w }\KeywordTok{with}\NormalTok{ (i,j)) }\KeywordTok{in}
            \KeywordTok{if}\NormalTok{ containsAB(s) }\KeywordTok{then}
                \KeywordTok{yield}\NormalTok{ s}
            \KeywordTok{endif}
        \KeywordTok{done}
    \KeywordTok{done}

\KeywordTok{def}\NormalTok{ main (w : }\DataTypeTok{[Char]}\NormalTok{) : }\DataTypeTok{[Char]}\NormalTok{ := }
    \KeywordTok{let}\NormalTok{ subwrds := subwordsWithAB(w) }\KeywordTok{in}
\KeywordTok{    for}\NormalTok{ (j,s) }\KeywordTok{in} \KeywordTok{enumerate}\NormalTok{(subwrds) }\KeywordTok{do}
\KeywordTok{        for}\NormalTok{ (i,c) }\KeywordTok{in} \KeywordTok{enumerate}\NormalTok{(s) }\KeywordTok{do}
            \KeywordTok{yield}\NormalTok{ c}
        \KeywordTok{done}
        \KeywordTok{yield}\StringTok{ \textquotesingle{}\#\textquotesingle{}}
    \KeywordTok{done}
\end{Highlighting}
\end{Shaded}
\caption{The \kl{high-level program} computing all subwords of a word containing the substring $ab$,
corresponding to the Python code in \cref{fig:python-example-nested}.}
\label{fig:high-level-example-nested}
\end{figure}

\begin{figure}
    \centering
\begin{Shaded}
\begin{Highlighting}[numbers=left]
\KeywordTok{for}\NormalTok{ i }\KeywordTok{in} \KeywordTok{input} \KeywordTok{do}
    \KeywordTok{for}\NormalTok{ j }\KeywordTok{in} \KeywordTok{reversed}\NormalTok{(}\KeywordTok{input}\NormalTok{) }\KeywordTok{do}
        \KeywordTok{let}\NormalTok{ b2, b3, b4 := }\KeywordTok{false} \KeywordTok{in}
        \KeywordTok{for}\NormalTok{ k }\KeywordTok{in} \KeywordTok{input} \KeywordTok{do}
            \KeywordTok{if}\NormalTok{ (i <= k) and (k <= j) }\KeywordTok{then}
                \KeywordTok{if}\NormalTok{ label(k) == \textquotesingle{}a\textquotesingle{} }\KeywordTok{then}
\NormalTok{                    b4 := }\KeywordTok{true}
                \KeywordTok{else}
                    \KeywordTok{if}\NormalTok{ (label(k) == \textquotesingle{}b\textquotesingle{}) and (b4) }\KeywordTok{then}
                        \KeywordTok{if}\NormalTok{ b3 }\KeywordTok{then}
                            \KeywordTok{skip}
                        \KeywordTok{else}
\NormalTok{                            b3 := }\KeywordTok{true}
\NormalTok{                            b2 := }\KeywordTok{true}
                        \KeywordTok{endif}
                    \KeywordTok{else}
                        \KeywordTok{skip}
                    \KeywordTok{endif}
                \KeywordTok{endif}
            \KeywordTok{else}
                \KeywordTok{skip}
            \KeywordTok{endif}
        \KeywordTok{done}
        \KeywordTok{if}\NormalTok{ b2 }\KeywordTok{then}
            \KeywordTok{for}\NormalTok{ l }\KeywordTok{in} \KeywordTok{input} \KeywordTok{do}
                \KeywordTok{if}\NormalTok{ (i <= l) and (l <= j) }\KeywordTok{then}
                    \KeywordTok{print}\NormalTok{ label(l)}
                \KeywordTok{else}
                    \KeywordTok{skip}
                \KeywordTok{endif}
            \KeywordTok{done}
            \KeywordTok{print}\NormalTok{ \textquotesingle{}\#\textquotesingle{}}
        \KeywordTok{else}
            \KeywordTok{skip}
        \KeywordTok{endif}
     \KeywordTok{done}
\KeywordTok{done}
\end{Highlighting}
\end{Shaded}
\caption{The \kl{simple for-program} computing all subwords of a word containing the substring $ab$,
corresponding to the Python code in \cref{fig:python-example-nested}, and obtained
by compiling \cref{fig:high-level-example-nested}.}
\label{fig:low-level-example-nested}
\end{figure}


\section{Syntax and Semantics of High Level Programs}

\subsection{Syntax} 

\begin{figure}[h]
    \centering
    \begin{align*}
        \intro*\bBinOp := &~ \land ~|~ \lor ~|~ \Rightarrow ~|~ \Leftrightarrow \\
        \intro*\pCmpOp := &~ = ~|~ \neq ~|~ < ~|~ \leq ~|~ > ~|~ \geq \\
        \intro*\bexpr :=&~ \intro*\btrue ~|~ \intro*\bfalse ~|~ \bnot{\bexpr} \\
               |&~ \bbin{\bexpr}{\bBinOp}{\bexpr}   \\
               |&~ \bcomp{i}{\pCmpOp}{j} & i,j \in \PVars \\
               |&~ \bapp{f}{\bexpr} & f \in \FunVars \\
               |&~ \bliteq{\cexpr}{\oexpr}
    \end{align*}
    \caption{Boolean expressions.}
    \label{fig:bool-expr}
\end{figure}


\begin{figure}[h]
    \centering
    \begin{align*}
        \intro*\cexpr :=&~ \mathsf{char} \; c & c \in \Sigma \\
               |&~ \mathsf{list}(\cexpr_1, \ldots, \cexpr_n)
    \end{align*}
    \caption{Constant expressions.}
    \label{fig:const-expr}
\end{figure}

\begin{figure}[h]
    \centering
    \begin{align*}
        \intro*\oexpr :=&~ x & x \in \OVars \\
               |&~ \cexpr \\
               |&~ \olist{\oexpr_1, \dots,  \oexpr_n}  \\
               |&~ f(\aexpr_1, \dots, \aexpr_n) & f \in \FunVars \\
    \end{align*}
    \caption{Output expressions.}
    \label{fig:out-expr}
\end{figure}

\begin{figure}[h]
    \centering
    \AP
    \begin{align*}
        \intro*\hlstmt :=&~ 
                   \intro*\hlif{\bexpr}{\hlstmt}{\hlstmt} \\
               |&~ \intro*\hlyield{\oexpr} \\
               |&~ \intro*\hlreturn{\oexpr} \\
               |&~ \intro*\hlletoutput{x}{\oexpr}{\hlstmt} & x \in \OVars \\
               |&~ \intro*\hlletboolean{x}{\hlstmt} & x \in \BVars \\
               |&~ \intro*\hlsettrue{x} & x \in \BVars \\
               |&~ \intro*\hlfor{(i,x)}{\oexpr}{\hlstmt} & (i,x) \in \PVars \times \OVars \\
               |&~ \intro*\hlforRev{(i,x)}{\oexpr}{\hlstmt} & (i,x) \in \PVars \times \OVars \\
               |&~ \intro*\hlseq{\hlstmt}{\hlstmt}
    \end{align*}
    \caption{Control statements.}
    \label{fig:high-level-stmt}
\end{figure}

\begin{figure}[h]
    \centering
    \begin{align*}
        \aexpr :=&~ (\oexpr, p_1, \dots, p_n) & \forall 1 \leq i \leq n, p_i \in \PVars \\
        \hlfun :=&~ \hlfundef{f}{\aexpr_1, \dots, \aexpr_n}{\hlstmt} & f \in \FunVars \\
        \hlprogram :=&~ ([\hlfun_1, \dots, \hlfun_n], f) & f \in \FunVars \\
    \end{align*}
    \caption{High Level For Programs.}
    \label{fig:high-level-program}
\end{figure}


\subsection{Semantics}

\begin{figure}[h]
    \centering
    \begin{align*}
        \semB{ \cdot }{\rho}               \colon&~ \bexpr \to \Bools \\
        \semB{\btrue}{\rho}               =&~ \top \\
        \semB{\bfalse}{\rho}              =&~ \bot \\
        \semB{b}{\rho}                    =&~ \rho(b) & b \in \BVars \\
        \semB{\bnot{b}}{\rho}             =&~ \neg \semB{b}{\rho} \\
        \semB{\bbin{b_1}{op}{b_2}}{\rho}  =&~ \semB{b_1}{\rho} \mathbin{op} \semB{b_2}{\rho} \\
        \semB{\bcomp{i}{op}{j}}{\rho}     =&~ \rho(i) \mathbin{op} \rho(j) \\
        \semB{\bapp{f}{b}}{\rho}          =&~ \semF{f}{\rho}(\semB{b}{\rho}) \\
        \semB{\bliteq{c}{o}}{\rho}        =&~ \semC{c}{\rho} = \semO{o}{\rho}
    \end{align*}
    \caption{Semantics of boolean expressions}.
\end{figure}

\begin{figure}
    \centering
    \begin{align*}
        \semC{ \cdot }                  \colon&~ \cexpr \to \OutputType \\
        \semC{\cchar{c}}                =&~ c \\
        \semC{\clist{c_1, \ldots, c_n}} =&~ [c_1, \ldots, c_n]
    \end{align*}
    \caption{Semantics of constant expressions.}
\end{figure}

\begin{figure}
    \centering
    \begin{align*}
        \semO{ \cdot }{\rho}       \colon&~ \oexpr \to \OutputType \\
        \semO{x}{\rho}                  =&~ \rho(x) & x \in \OVars \\
        \semO{\cexpr}{\rho}             =&~ \semC{\cexpr} \\
        \semO{\olist{o_1, \ldots, o_n}}{\rho} =&~ [\semO{o_1}{\rho}, \ldots, \semO{o_n}{\rho}] \\
        \semO{f(\aexpr_1, \dots, \aexpr_n)}{\rho} =&~ \semF{f}{\rho}(\semO{\aexpr_1}{\rho}, \dots, \semO{\aexpr_n}{\rho})
    \end{align*}
    \caption{Semantics of output expressions.}
\end{figure}

\begin{figure}
    \centering
    \begin{align*}
        \semC{\cdot}{\rho} \colon&~ \hlstmt \to \rho \to \rho \\
    \end{align*}
    \caption{Semantics of control statements.}
    \label{fig:semantics-control}
\end{figure}

\begin{figure}
    \centering
    \begin{align*}
        \semF{\cdot}{\rho} \colon&~ \hlfun \to \rho \to \rho \\
        \semF{\hlfundef{f}{\aexpr_1, \dots, \aexpr_n}{\hlstmt}}{\rho} =&~ \lambda x_1, \dots, x_n. \semC{\hlstmt}{\rho[x_1 \mapsto \semO{\aexpr_1}{\rho}, \dots, x_n \mapsto \semO{\aexpr_n}{\rho}]}
    \end{align*}
    \caption{Semantics of functions.}
    \label{fig:semantics-functions}
\end{figure}

\begin{figure}
\end{figure}


% Basic typing judgments
\begin{figure}[h]
    \begin{prooftree}
    \AxiomC{}
    \RightLabel{(T-True)}
    \UnaryInfC{$\Gamma \vdash \btrue : \TBool$}
    \end{prooftree}

    \begin{prooftree}
    \AxiomC{}
    \RightLabel{(T-False)}
    \UnaryInfC{$\Gamma \vdash \bfalse : \TBool$}
    \end{prooftree}

    % Boolean expressions
    \begin{prooftree}
    \AxiomC{$\Gamma \vdash e : \TBool$}
    \RightLabel{(T-Not)}
    \UnaryInfC{$\Gamma \vdash \bnot{e} : \TBool$}
    \end{prooftree}

    \begin{prooftree}
    \AxiomC{$\Gamma \vdash e_1 : \TBool$}
    \AxiomC{$\Gamma \vdash e_2 : \TBool$}
    \RightLabel{(T-BBin)}
    \BinaryInfC{$\Gamma \vdash \bbin{e_1}{\odot}{e_2} : \TBool$}
    \end{prooftree}

    % Position comparisons
    \begin{prooftree}
    \AxiomC{$\Gamma(i) = \TPos[o_i]$}
    \AxiomC{$\Gamma(j) = \TPos[o_j]$}
    \AxiomC{$o_i = o_j$}
    \RightLabel{(T-PComp)}
    \TrinaryInfC{$\Gamma \vdash \bcomp{i}{\odot}{j} : \TBool$}
    \end{prooftree}

    % Function application in boolean expressions
    \begin{prooftree}
    \AxiomC{$\Delta(f) = (\tau_1,\ldots,\tau_n) \to \TBool$}
    \AxiomC{$\Gamma \vdash e : \tau_1$}
    \RightLabel{(T-BApp)}
    \BinaryInfC{$\Gamma \vdash \bapp{f}{e} : \TBool$}
    \end{prooftree}
    \caption{Typing rules for boolean expressions.}
    \label{fig:typing-bool}
\end{figure}

\begin{figure}[h]
    % Output expressions
    \begin{prooftree}
    \AxiomC{$\Gamma(x) = \TOut[n]$}
    \RightLabel{(T-OVar)}
    \UnaryInfC{$\Gamma \vdash x : \TOut[n]$}
    \end{prooftree}

    \begin{prooftree}
        \AxiomC{$\Gamma \vdash e_i : \TOut[n]$ for all $i$}
    \RightLabel{(T-OList)}
    \UnaryInfC{$\Gamma \vdash \olist{e_1,\ldots,e_n} : \TOut[n+1]$}
    \end{prooftree}
    \caption{Typing rules for \kl{list expressions} and \kl{constant expressions}.}
    \label{fig:typing-output}
\end{figure}

\begin{figure}
% Statements
\begin{prooftree}
\AxiomC{$\Gamma \vdash e : \TBool$}
\AxiomC{$\Gamma \vdash s_1 : \tau$}
\AxiomC{$\Gamma \vdash s_2 : \tau$}
\RightLabel{(T-If)}
\TrinaryInfC{$\Gamma \vdash \hlif{e}{s_1}{s_2} : \tau$}
\end{prooftree}

\begin{prooftree}
\AxiomC{$\Gamma \vdash e : \TOut[n]$}
\RightLabel{(T-Yield)}
\UnaryInfC{$\Gamma \vdash \hlyield{e} : \TOut[n+1]$}
\end{prooftree}

\begin{prooftree}
\AxiomC{$\Gamma \vdash e : \TOut[n]$}
\RightLabel{(T-Return)}
\UnaryInfC{$\Gamma \vdash \hlreturn{e} : \TOut[n]$}
\end{prooftree}

\begin{prooftree}
\AxiomC{$\Gamma \vdash e : \TBool$}
\RightLabel{(T-Return)}
\UnaryInfC{$\Gamma \vdash \hlreturn{e} : \TBool$}
\end{prooftree}

% Let bindings
\begin{prooftree}
\AxiomC{$\Gamma \vdash e : \TOut[n]$}
\AxiomC{$\Gamma, x:\TOut[n] \vdash s : \tau$}
\RightLabel{(T-LetOut)}
\BinaryInfC{$\Gamma \vdash \hlletoutput{x}{e}{s} : \tau$}
\end{prooftree}

\begin{prooftree}
\AxiomC{$\Gamma, x:\TBool \vdash s : \tau$}
\RightLabel{(T-LetBool)}
\UnaryInfC{$\Gamma \vdash \hlletboolean{x}{s} : \tau$}
\end{prooftree}

% For loops
\begin{prooftree}
\AxiomC{$\Gamma \vdash o : \TOut[n]$}
\AxiomC{$\Gamma, i:\TPos[o], x:\TOut[n] \vdash s : \tau$}
\AxiomC{$n > 0$}
\RightLabel{(T-For)}
\TrinaryInfC{$\Gamma \vdash \hlfor{(i,x)}{o}{s} : \tau$}
\end{prooftree}

\begin{prooftree}
\AxiomC{$\Gamma \vdash o : \TOut[n]$}
\AxiomC{$\Gamma, i:\TPos[o], x:\TOut \vdash s : \tau$}
\AxiomC{$n > 0$}
\RightLabel{(T-ForRev)}
\TrinaryInfC{$\Gamma \vdash \hlforRev{(i,x)}{o}{s} : \tau$}
\end{prooftree}

\begin{prooftree}
\AxiomC{}
\RightLabel{(T-SetTrue)}
\UnaryInfC{$\Gamma \vdash \hlsettrue{x} : \tau$}
\end{prooftree}

% Sequence
\begin{prooftree}
\AxiomC{$\Gamma \vdash s_1 : \tau$}
\AxiomC{$\Gamma \vdash s_2 : \tau$}
\RightLabel{(T-Seq)}
\BinaryInfC{$\Gamma \vdash \hlseq{s_1}{s_2} : \tau$}
\end{prooftree}
\caption{Typing rules for control statements.}
\label{fig:typing-control}
\end{figure}

\begin{figure}
% Function definition
\begin{prooftree}
\AxiomC{$\Gamma \vdash \aexpr_i : \tau_i$ for all $i$}
\AxiomC{$\Gamma \vdash s : \tau$}
\RightLabel{(T-Fun)}
\BinaryInfC{$\Delta \vdash \hlfundef{f}{(\aexpr_1,\ldots,\aexpr_n)}{s} : (\tau_1,\ldots,\tau_n) \to \tau$}
\end{prooftree}

% Program
\begin{prooftree}
\AxiomC{$\Delta \vdash \hlfun_i : \tau_i$ for all $i$}
\AxiomC{$f \in \FunVars$}
\RightLabel{(T-Prog)}
\BinaryInfC{$\vdash ([\hlfun_1,\ldots,\hlfun_n], f) : \tau_f$}
\end{prooftree}
\caption{Typing rules for high-level programs.}
\label{fig:typing-high-level}
\end{figure}

\section{Program Counter Examples}

\begin{lemma}
    \label{lem:umc-equality-nested-words}
    Allowing unrestricted equality checks between two \kl{nested words}
    results in the undecidability of the \kl{model checking problem}.
\end{lemma}
\begin{proof}
        \begin{example}
            For every instance of the Post Correspondence Problem (PCP), we can 
            construct a function \texttt{f(x : list[list[Char]]) : Bool}
            in the \kl{high-level language} with unrestricted equality checks, such 
            that $f(x) = \btrue$ if and only if $x$ encodes a solution to the PCP instance.
            For example the PCP instance $\{ (ab, a), (b, aa), (ba, b) \}$ can be encoded 
            as the following function:
            \begin{verbatim}
                \input{programs/pcp_instance.py}
            \end{verbatim}
        \end{example}
\end{proof}


\section{High Level to Low Level Translation}


\begin{lemma}
    \label{lem:fungenexpr}
    For every \kl{high level function} $\hlfundef{f}{x_1, \dots, x_n}{s}$,
    for every arguments $a_1, \dots, a_n \in \aexpr$,
    for every \kl{evaluation environment} $\rho$,
    \begin{equation*}
        \semO{ f (a_1, \dots, a_n) }{\rho}
        = 
        \semO{ \ogen{ s[a_1/x_1, \dots, a_n/x_n] } }{\rho}
        \quad .
    \end{equation*}
\end{lemma}



\end{document}
