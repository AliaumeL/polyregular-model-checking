%! TEX program = xelatex
% WARNING: this is a generated file.
%
% Please do not edit this file directly. 
% - If you want to update the medatata of the paper (title, authors, abstract), please
%   edit the `paper-meta.yaml` file in the root of the repository.
% - If you want to update the content of the paper, please edit the latex files
%   in the `src` directory.
% - If you want to update the template itself (e.g., change the layout), please
%   edit the `templates/plain-article.tex` file instead.
\documentclass[11pt,a4paper,twosided]{article}

% we setup a custom geometry because the default one is too narrow
\usepackage{geometry}
\geometry{margin=3.5cm}

% we include whatever the user wants to include in the header

% we include libraries (tex files) usually written in the `lib` directory
\input{lib/packages}
\input{lib/aliaume}

% High Level For Programs Syntax and semantics
\NewDocumentCommand{\PVars}{}{\mathbb{V}_{\text{pos}}}
\NewDocumentCommand{\FunVars}{}{\mathbb{V}_{\text{fun}}}
\NewDocumentCommand{\BVars}{}{\mathbb{V}_{\text{bool}}}
\NewDocumentCommand{\OVars}{}{\mathbb{V}_{\text{out}}}

\NewDocumentCommand{\Bools}{}{\mathbb{B}}
\NewDocumentCommand{\OutputType}{O{\Sigma}}{\mathcal{C}_{#1}}

\NewDocumentCommand{\hlprogram}{}{\mathsf{Program}}
\NewDocumentCommand{\hlfun}{}{\mathsf{Fun}}
\NewDocumentCommand{\hlstmt}{}{\mathsf{Stmt}}
\NewDocumentCommand{\bexpr}{}{\mathsf{BExpr}}
\NewDocumentCommand{\oexpr}{}{\mathsf{OExpr}}
\NewDocumentCommand{\cexpr}{}{\mathsf{CExpr}}
\NewDocumentCommand{\aexpr}{}{\mathsf{AExpr}}

\NewDocumentCommand{\bBinOp}{}{\mathsf{BBin}}
\NewDocumentCommand{\pCmpOp}{}{\mathsf{PComp}}

\NewDocumentCommand{\hlif}{m m m}{\mathsf{if} \; #1 \; \mathsf{then} \; #2 \; \mathsf{else} \; #3}
\NewDocumentCommand{\hlyield}{m}{\mathsf{yield} \; #1}
\NewDocumentCommand{\hlreturn}{m}{\mathsf{return} \; #1}
\NewDocumentCommand{\hlletoutput}{m m m}{\mathsf{let} \; #1 = #2 \; \mathsf{in} \; #3}
\NewDocumentCommand{\hlletboolean}{m m}{\mathsf{let~mut} \; #1 = \bfalse \; \mathsf{in} \; #2}
\NewDocumentCommand{\hlsettrue}{m}{\; #1 \; \leftarrow \; \mathsf{true}}
\NewDocumentCommand{\hlfor}{m m m}{\mathsf{for}^{\rightarrow} \; #1 \; \mathsf{in} \; #2 \; \mathsf{do} \; #3}
\NewDocumentCommand{\hlforRev}{m m m}{\mathsf{for}^{\leftarrow} \; #1 \; \mathsf{in} \; #2 \; \mathsf{do} \; #3}
\NewDocumentCommand{\hlseq}{m m }{ #1 \; ; \; #2}


\NewDocumentCommand{\btrue}{}{\mathsf{true}}
\NewDocumentCommand{\bfalse}{}{\mathsf{false}}
\NewDocumentCommand{\bnot}{m}{\neg #1}
\NewDocumentCommand{\bbin}{ m m m }{#1 \; #2 \; #3}
\NewDocumentCommand{\bcomp}{m m m}{#1 \; #2 \; #3}
\NewDocumentCommand{\bapp}{m m}{\mathop{#1}(#2)}
\NewDocumentCommand{\bliteq}{m m}{#1 \; = \; #2}
\NewDocumentCommand{\bgen}{m}{\langle #1 \rangle}


\NewDocumentCommand{\ogen}{m}{\langle #1 \rangle}
\NewDocumentCommand{\cchar}{m}{\mathsf{char}(#1)}
\NewDocumentCommand{\clist}{m}{\mathsf{list}(#1)}
\NewDocumentCommand{\olist}{m}{\mathsf{list}(#1)}

\NewDocumentCommand{\hlfundef}{m m m}{\mathsf{def} \; #1(#2) \; \{ #3 \}}


% We include the title and author information based on the 
% `paper-meta.yaml` file.
 
\title{Polyregular Model Checking}

\author{
Aliaume Lopez\thanks{University of Warsaw, Poland}
 \and
Rafał Stefański\thanks{University of Warsaw, Poland}
}

% For the date, we first check if the user has provided a date,
% and otherwise use the git meta inforamtion (if available).
\date{2024-09-03 12:59:43
+0200 -- 0c52ab1e2303f5d2959eba5a98264a411874675d\footnote{branch main at git@github.com:AliaumeL/polyregular-model-checking.git}}


% Now, we create the document itself.
\begin{document}
% Generate the title page
\maketitle
% Print the abstract
\begin{abstract}
    In this paper, we demonstrate how the theory of polyregular
    functions can be leveraged to verify simple string-to-string
    for-loop programs defined in a subset of Python in an efficient way.
\end{abstract}

% Include the content of the paper
% LTeX: language=EN
\section{Introduction}
\label{sec:intro}

The goal of this paper is to define a programming language that is expressive
enough to capture a wide range of \emph{string-to-string} functions, and yet
simple enough to be amenable to formal verification. Specifically, we want to
be able to verify Hoare triples of the form
$\hoaretriple{P}{\texttt{code}}{Q}$, where $P$ and $Q$ are predicates and
\texttt{code} is a program, meaning that whenever the input satisfies property
$P$, the output of the program satisfies property $Q$.

\paragraph{Regularity preserving programs.}
\begin{itemize}
  \item we are interested in precondition and postcondition
    that are regular languages, which is a fairly general 
    framework.
  \item define regularity preserving.
  \item this is at the core of several decision procedures.
  \item one of them is path feasibility analysis, where 
    we are interested in compositions of regularity preserving functions.
  \item there are uncomputable functions that are
    regularity preserving, and one has to select a computation model.
\end{itemize}

\paragraph{String-to-string transducer models.}
\begin{itemize}
  \item While most automata models capture the same 
    class of languages (regular languages), there is zoo of models for string-to-string transducers
    in the litterature. 
  \item Notably, in POPL'11, X and Y have devised an effective pullback procedure
    for SSTs (linear regular functions, equivalent to 2DFTs), showing a PSPACE
    complete complexity.
  \item In POPL'19, they
    rely on the effective pullback procedure of
    linear regular functions. 

  \item In CAV'23, they use infinite input alphabets (also known as atoms/nominal sets/data words),
    and in this setting, the weaker class of rational functions is used 
    because full 2DFTs not regularity preserving in this nominal setting.

  \item Recently, the theory of \emph{polyregular functions} has gained a lot of traction.
    It goes beyond the previous models, by allowing polynomial growth of the output.
    One of the equivalent definitions of this model dates back to \cite{ENMA02}, and 
    several other characterizations have been proposed since then
    \cite{bojanczyk2018polyregular,bojanczyk2019string,bojanczyk2023growth}, 
    showing that it is a robust class of functions.
    It is known that they are regularity preserving [cite],

  \item However, this result is of theoretical nature
    (no implementation or complexity bounds are given), and writing programs using
    any of the existing equivalent definitions of this 
    model is cumbersome and error-prone. Furthermore, relying on monadic
    second order logic implies that one cannot use the vast majority of SMT
    solvers, which only handle first-order logic.

  \item Because polyregular functions are closed under compositions,
    one can reduce POPL'19 to a single model checking problem. Furthermore,
    polyregular fuctions contain linear regular functions. 

  \item Beware that \kl{simple for-programs} can encode any FO formula 
    with a linear blowup, hence their model checking is as hard as
    the emptiness problem for FO formulas, which is known to be
    TOWER-complete (stockmeyer).

  \item Let us also mention that the study of exponential growth functions is on the way.
\end{itemize}



\paragraph{MSO vs FO.} 
\begin{itemize}
  \item Usually, one considers monadic second order logic (MSO) to express properties of
    regular languages.
  \item In order to simplify the presentation, and allow for an encoding into
    SMTLib, we will use first-order logic (FO) instead.
  \item 
The correspondence between subsets of string to string
functions and first order logic dates back to the origins of automata theory
and the seminal results of \cite{PEPI86,SCHU65,MNPA71}, establishing the
equivalence between \emph{star-free languages}, \emph{first order definable
languages}, and \emph{counter free automata}.
  \item Extensions of this correspondence
to functions has been an active area of research \cite{CADA15,MUSC19}, which we
leverage in this work via the theory of \intro{polyregular functions}
\cite{ENMA02,bojanczyk2018polyregular,bojanczyk2019string,bojanczyk2023growth}.
\end{itemize}


\paragraph{Contributions.} 
\begin{itemize}
  \item In this paper our contributions are threefold.
  \item 
First, we introduce a programming language that corresponds to a rich subset of
\texttt{Python}, which we argue is expressive enough to be usable in practice.

\item Second, we demonstrate that this language can be compiled into a certain type
of \kl{polyregular functions}. 

\item Third, we prove that for these polyregular
functions, the verification of \kl{Hoare triples} (specified using
\kl{first-order logic} on words) effectively reduces to a satisfiability
problem of a first-order formula on finite words. 

\item While this last step was
known to be theoretically possible, an efficient and effective implementation
was lacking. 

\item Because we are using \kl{first-order logic} as a target language,
we are not restricted to using automata based tools like \intro{MONA}
\cite{MONA01}, but can also employ general purpose SMT solvers like \intro{Z3}
\cite{z3} and \intro{CVC5} \cite{cvc5}, generating proof
obligations in the \texttt{SMT-LIB} format \cite{BARRETT17}.

\item We implemented all of these conversions in a \texttt{Haskell} program, and
tested it on a number of examples with encouraging results.\footnote{An
anonymized version of our code is available at \repositoryUrl.}

\item That being said, we are not a tool paper, and the implementation should
  be seen as a proof of concept.

\item Our initial tests, while promising, are not fully-fledged benchmarks

\item more benchmarks are needed to assess the viability of our approach.

\item and we are aware that there is room for 
optimizations both in the implementation and the way encoding used
to discharge the proof obligations when using SMT solvers.

\end{itemize}

\hline

\paragraph{Outline.} The structure of the paper is as follows. We introduce our
\kl{high-level language} in \cref{sec:high-level}. In \cref{sec:polyregular},
we recall the theory of \kl{polyregular functions} by introducing them in terms
of \kl{simple for-programs} and \kl{$\FO$-interpretations}. We will also
provide an efficient reduction of the verification of Hoare triples to the
satisfiability of a \kl{first-order formula on words} in \cref{sec:pullback}.
In order to verify \kl{for-programs}, we compile them into \kl{simple
for-programs} in \cref{sec:htl}, and then compile \kl{simple for-programs} into
\kl{$\FO$-interpretations} in \cref{sec:low-level}. 
%These steps 
%are summarized in the following diagram:
%\begin{center}
%    \begin{tikzpicture}[
%        syntaxNode/.style={
%                    rectangle, draw, 
%                    text width=5em, 
%                    text centered, 
%                    rounded corners, 
%                    minimum height=4em}
%    ]
%        %
%        % Write a tikz picture with nodes explaining the different 
%        % steps of the rewriting system 
%        % 
%        % (a) high-level language
%        % (b) simple for-programs
%        % (c) first-order string-to-string interpretations
%        % (c1) precondition
%        % (c2) postcondition
%        % (d) first-order formula
%        \node[syntaxNode] (a) {\kl{For-program}};
%        \node[syntaxNode, right=of a] (b)  {\kl{Simple for-program}};
%        \node[syntaxNode, right=of b] (c)  {\kl{$\FO$ interpretation}};
%        \node[syntaxNode, right=of c] (d) {\kl{First-order formula}};
%
%        \draw[->] (a) -- node[above,rotate=90,xshift=2.3em,yshift=-0.3em] {\cref{sec:htl}} (b);
%        \draw[->] (b) -- node[above,rotate=90,xshift=2.3em,yshift=-0.3em] {\cref{sec:low-level}} (c);
%        \draw[->] (c) -- node[above,rotate=90,xshift=2.7em,yshift=-0.3em] {\cref{sec:pullback}} (d);
%    \end{tikzpicture}
%\end{center}
Then, in \cref{sec:benchmarks}, we present
benchmarks of our implementation on various examples, discussing
the complexity of the transformations and the main bottlenecks of our approach.
Finally, we conclude in \cref{sec:conclusion} by discussing potential
optimizations and future work.



% Include the bibliography
\bibliographystyle{plainurl}
\bibliography{papers.bib}

% If there are any appendices, we include them here.
\appendix
\section{Proofs}

\begin{lemma}
    \label{lem:umc-equality-nested-words}
    Allowing unrestricted equality checks between two \kl{nested words}
    results in the undecidability of the model checking problem.
\end{lemma}
\begin{proof}
    For every instance of the Post Correspondence Problem (PCP), we can 
    construct a function \texttt{f(x : list[list[Char]]) : Bool}
    in the \kl{high-level language} with unrestricted equality checks, such 
    that $f(x) = \btrue$ if and only if $x$ encodes a solution to the PCP instance.
    For example the PCP instance $\{ (ab, a), (b, aa), (ba, b) \}$ can be encoded 
    as the following function:
    \begin{verbatim}
        \input{programs/pcp_instance.py}
    \end{verbatim}
\end{proof}

\begin{lemma}
    \label{lem:fo-emptiness}
    \proofref{lem:fo-emptiness}
    The \kl{emptiness} problem for the \kl{first-order logic on words} is decidable for the infinite alphabet $\mathcal{D}$.
\end{lemma}
\begin{proof}
    Take a formula $\varphi$ and observe that is contains only a finite number of constants from $\mathcal{D}$ -- call this set $A$.
    It is not hard to see that the truth value of $\varphi$ is \kl{supported} by $A$: for every function $f : \mathcal{D} \to \mathcal{D}$
    that does not touch 
    elements of $A$, the truth value of $\varphi$ is the same for $w$ and $f^*(w)$. (Remember that 
    $f^*$ is the pointwise application of $f$). 
    Let $\mathtt{blank} \in \mathcal{D}$ be a letter that does not appear in $A$,
    and observe that the formula $\varphi$ is satisfied for some word in $\mathcal{D}^*$ if and only if it is satisfied by
    some word in $(A \cup \{\mathtt{blank}\})^*$. Indeed, if we take a function $g: \mathcal{D} \to \mathcal{D}$ that does not touch elements of $A$
    and maps all other letters to $\mathtt{blank}$, we can use it to map $\mathcal{D}^*$ to $(A \cup \{\mathtt{blank}\})^*$ in a way 
    that preserves the truth value of $\varphi$.
    This finishes the proof of the lemma, as we have reduced the general problem to a finite alphabet $A \cup \{\mathtt{blank}\}$.
\end{proof}

\clearpage
\section{Subwords Containing the Substring $ab$}

\begin{figure}
    \centering
\begin{Shaded}
\begin{Highlighting}[numbers=left]
\KeywordTok{def}\NormalTok{ getBetween( l : }\DataTypeTok{[Char]} \KeywordTok{with}\NormalTok{ (i,j) ) : }\DataTypeTok{[Char]}\NormalTok{ := }
\KeywordTok{    for}\NormalTok{ (k,c) }\KeywordTok{in} \KeywordTok{enumerate}\NormalTok{(l) }\KeywordTok{do}
        \KeywordTok{if}\NormalTok{ i \textless{}= k }\KeywordTok{and}\NormalTok{ k \textless{}= j }\KeywordTok{then}
            \KeywordTok{yield}\NormalTok{ c}
        \KeywordTok{endif}
    \KeywordTok{done}

\KeywordTok{def}\NormalTok{ containsAB(w : }\DataTypeTok{[Char]}\NormalTok{) : }\DataTypeTok{Bool}\NormalTok{ := }
    \KeywordTok{let} \KeywordTok{mut}\NormalTok{ seen\_a := }\KeywordTok{False} \KeywordTok{in} 
\KeywordTok{    for}\NormalTok{ (i, c) }\KeywordTok{in} \KeywordTok{enumerate}\NormalTok{(w) }\KeywordTok{do}
    \KeywordTok{if}\NormalTok{ c === }\StringTok{\textquotesingle{}a\textquotesingle{} }\KeywordTok{then}
\NormalTok{            seen\_a := }\KeywordTok{True}
    \KeywordTok{else} \KeywordTok{if}\NormalTok{ c === }\StringTok{\textquotesingle{}b\textquotesingle{} }\KeywordTok{and}\NormalTok{ seen\_a }\KeywordTok{then}
            \KeywordTok{return} \KeywordTok{True}
        \KeywordTok{endif} \KeywordTok{endif}
    \KeywordTok{done}
    \KeywordTok{return} \KeywordTok{False}

\KeywordTok{def}\NormalTok{ subwordsWithAB(w : }\DataTypeTok{[Char]}\NormalTok{) : }\DataTypeTok{[[Char]]}\NormalTok{ := }
\KeywordTok{    for}\NormalTok{ (i,c) }\KeywordTok{in} \KeywordTok{enumerate}\NormalTok{(w) }\KeywordTok{do}
\KeywordTok{        for}\NormalTok{ (j,d) }\KeywordTok{in} \KeywordTok{reversed}\NormalTok{(}\KeywordTok{enumerate}\NormalTok{(w)) }\KeywordTok{do}
            \KeywordTok{let}\NormalTok{ s := getBetween(w }\KeywordTok{with}\NormalTok{ (i,j)) }\KeywordTok{in}
            \KeywordTok{if}\NormalTok{ containsAB(s) }\KeywordTok{then}
                \KeywordTok{yield}\NormalTok{ s}
            \KeywordTok{endif}
        \KeywordTok{done}
    \KeywordTok{done}

\KeywordTok{def}\NormalTok{ main (w : }\DataTypeTok{[Char]}\NormalTok{) : }\DataTypeTok{[Char]}\NormalTok{ := }
    \KeywordTok{let}\NormalTok{ subwrds := subwordsWithAB(w) }\KeywordTok{in}
\KeywordTok{    for}\NormalTok{ (j,s) }\KeywordTok{in} \KeywordTok{enumerate}\NormalTok{(subwrds) }\KeywordTok{do}
\KeywordTok{        for}\NormalTok{ (i,c) }\KeywordTok{in} \KeywordTok{enumerate}\NormalTok{(s) }\KeywordTok{do}
            \KeywordTok{yield}\NormalTok{ c}
        \KeywordTok{done}
        \KeywordTok{yield}\StringTok{ \textquotesingle{}\#\textquotesingle{}}
    \KeywordTok{done}
\end{Highlighting}
\end{Shaded}
\caption{The \kl{for-program} computing all subwords of a word containing the substring $ab$,
corresponding to the Python code in \cref{fig:python-example-nested}.}
\label{fig:high-level-example-nested}
\end{figure}

\begin{figure}
    \centering
\begin{Shaded}
\begin{Highlighting}[numbers=left]
\KeywordTok{for}\NormalTok{ i }\KeywordTok{in} \KeywordTok{input} \KeywordTok{do}
    \KeywordTok{for}\NormalTok{ j }\KeywordTok{in} \KeywordTok{reversed}\NormalTok{(}\KeywordTok{input}\NormalTok{) }\KeywordTok{do}
        \KeywordTok{let}\NormalTok{ b2, b3, b4 := }\KeywordTok{false} \KeywordTok{in}
        \KeywordTok{for}\NormalTok{ k }\KeywordTok{in} \KeywordTok{input} \KeywordTok{do}
            \KeywordTok{if}\NormalTok{ (i <= k) and (k <= j) }\KeywordTok{then}
                \KeywordTok{if}\NormalTok{ label(k) == \textquotesingle{}a\textquotesingle{} }\KeywordTok{then}
\NormalTok{                    b4 := }\KeywordTok{true}
                \KeywordTok{else}
                    \KeywordTok{if}\NormalTok{ (label(k) == \textquotesingle{}b\textquotesingle{}) and (b4) }\KeywordTok{then}
                        \KeywordTok{if}\NormalTok{ b3 }\KeywordTok{then}
                            \KeywordTok{skip}
                        \KeywordTok{else}
\NormalTok{                            b3 := }\KeywordTok{true}
\NormalTok{                            b2 := }\KeywordTok{true}
                        \KeywordTok{endif}
                    \KeywordTok{else}
                        \KeywordTok{skip}
                    \KeywordTok{endif}
                \KeywordTok{endif}
            \KeywordTok{else}
                \KeywordTok{skip}
            \KeywordTok{endif}
        \KeywordTok{done}
        \KeywordTok{if}\NormalTok{ b2 }\KeywordTok{then}
            \KeywordTok{for}\NormalTok{ l }\KeywordTok{in} \KeywordTok{input} \KeywordTok{do}
                \KeywordTok{if}\NormalTok{ (i <= l) and (l <= j) }\KeywordTok{then}
                    \KeywordTok{print}\NormalTok{ label(l)}
                \KeywordTok{else}
                    \KeywordTok{skip}
                \KeywordTok{endif}
            \KeywordTok{done}
            \KeywordTok{print}\NormalTok{ \textquotesingle{}\#\textquotesingle{}}
        \KeywordTok{else}
            \KeywordTok{skip}
        \KeywordTok{endif}
     \KeywordTok{done}
\KeywordTok{done}
\end{Highlighting}
\end{Shaded}
\caption{The \kl{simple for-program} computing all subwords of a word containing the substring $ab$,
corresponding to the Python code in \cref{fig:python-example-nested}, and obtained
by compiling \cref{fig:high-level-example-nested}.}
\label{fig:low-level-example-nested}
\end{figure}

\clearpage
\section{Encoding Equality of Subwords}

\begin{figure}
    \centering
\begin{Shaded}
\begin{Highlighting}[]
\KeywordTok{def}\NormalTok{ eq(u, v):}
    \ControlFlowTok{for}\NormalTok{ (i, ui) }\KeywordTok{in} \BuiltInTok{enumerate}\NormalTok{(u):}
        \ControlFlowTok{for}\NormalTok{ (j, vj) }\KeywordTok{in} \BuiltInTok{enumerate}\NormalTok{(v):}
            \ControlFlowTok{if}\NormalTok{ i }\OperatorTok{==}\NormalTok{ j }\KeywordTok{and}\NormalTok{ ui }\OperatorTok{!=}\NormalTok{ vj:}
                \ControlFlowTok{return} \VariableTok{False}
    \ControlFlowTok{return} \VariableTok{True}
\end{Highlighting}
\end{Shaded}
\caption{Encoding the equality of two words $u$ and $v$ in Python,
using a comparison between indices of two different lists.}
\label{fig:eq-def-different-indices}
\end{figure}

\begin{figure}
    \centering
\begin{Shaded}
\begin{Highlighting}[]
\KeywordTok{def}\NormalTok{ switch(b, u, v):}
    \ControlFlowTok{if}\NormalTok{ b:}
        \ControlFlowTok{return}\NormalTok{ u}
    \ControlFlowTok{else}\NormalTok{:}
        \ControlFlowTok{return}\NormalTok{ v}

\KeywordTok{def}\NormalTok{ eq(u, v):}
\NormalTok{    b }\OperatorTok{=} \VariableTok{False}
    \ControlFlowTok{for}\NormalTok{ (i, ui) }\KeywordTok{in} \BuiltInTok{enumerate}\NormalTok{(switch(b, u, v)):}
\NormalTok{        b }\OperatorTok{=} \VariableTok{True}
        \ControlFlowTok{for}\NormalTok{ (j, vj) }\KeywordTok{in} \BuiltInTok{enumerate}\NormalTok{(switch(b, u, v)):}
            \ControlFlowTok{if}\NormalTok{ i }\OperatorTok{==}\NormalTok{ j }\KeywordTok{and}\NormalTok{ ui }\OperatorTok{!=}\NormalTok{ vj:}
                \ControlFlowTok{return} \VariableTok{False}
    \ControlFlowTok{return} \VariableTok{True}
\end{Highlighting}
\end{Shaded}
\caption{Encoding the equality of two words $u$ and $v$ in Python,
using a function taking a boolean as input.}
\label{fig:eq-def-boolean}
\end{figure}

\begin{figure}
    \centering
\begin{Shaded}
\begin{Highlighting}[]
\KeywordTok{def}\NormalTok{ eq(u, v):}
\NormalTok{    w }\OperatorTok{=}\NormalTok{ u}
    \ControlFlowTok{for}\NormalTok{ (i, ui) }\KeywordTok{in} \BuiltInTok{enumerate}\NormalTok{(w):}
\NormalTok{        w }\OperatorTok{=}\NormalTok{ v}
        \ControlFlowTok{for}\NormalTok{ (j, vj) }\KeywordTok{in} \BuiltInTok{enumerate}\NormalTok{(w):}
            \ControlFlowTok{if}\NormalTok{ i }\OperatorTok{==}\NormalTok{ j }\KeywordTok{and}\NormalTok{ ui }\OperatorTok{!=}\NormalTok{ vj:}
                \ControlFlowTok{return} \VariableTok{False}
    \ControlFlowTok{return} \VariableTok{True}
\end{Highlighting}
\end{Shaded}
\caption{Encoding the equality of two words $u$ and $v$ in Python,
using the shadowing of a variable to switch between two lists.}
\label{fig:eq-def-shadowing}
\end{figure}



\clearpage
\section{Syntax and Semantics of High Level Programs}

\begin{figure}[h]
    \centering
    \begin{align*}
        \intro*\bBinOp := &~ \land ~|~ \lor ~|~ \Rightarrow ~|~ \Leftrightarrow \\
        \intro*\pCmpOp := &~ = ~|~ \neq ~|~ < ~|~ \leq ~|~ > ~|~ \geq \\
        \intro*\bexpr :=&~ \intro*\btrue ~|~ \intro*\bfalse ~|~ \intro*\bnot{\bexpr} \\
               |&~ \bbin{\bexpr}{\bBinOp}{\bexpr}   \\
               |&~ \bcomp{i}{\pCmpOp}{j} & i,j \in \PVars \\
               |&~ f(\bexpr) & f \in \FunVars \\
               |&~ \intro*\bliteq{\cexpr}{\oexpr}
    \end{align*}
    \caption{The syntax of \kl{boolean expressions}.}
    \label{fig:bool-expr}
\end{figure}


\begin{figure}[h]
    \centering
    \begin{align*}
        \intro*\cexpr :=&~ \mathsf{char} \; c & c \in \Sigma \\
               |&~ \mathsf{list}(\cexpr_1, \ldots, \cexpr_n)
    \end{align*}
    \caption{The syntax of \kl{constant expressions}.}
    \label{fig:const-expr}
\end{figure}

\begin{figure}[h]
    \centering
    \begin{align*}
        \intro*\oexpr :=&~ x & x \in \OVars \\
               |&~ \cexpr \\
               |&~ \intro*\olist{\oexpr_1, \dots,  \oexpr_n}  \\
               |&~ f(\aexpr_1, \dots, \aexpr_n) & f \in \FunVars \\
    \end{align*}
    \caption{The syntax of \kl{list expressions}.}
    \label{fig:out-expr}
\end{figure}

\begin{figure}[h]
    \centering
    \AP
    \begin{align*}
        \intro*\hlstmt :=&~ 
                   \intro*\hlif{\bexpr}{\hlstmt}{\hlstmt} \\
               |&~ \intro*\hlyield{\oexpr} \\
               |&~ \intro*\hlreturn{\oexpr} \\
               |&~ \intro*\hlletoutput{x}{\oexpr}{\hlstmt} & x \in \OVars \\
               |&~ \intro*\hlletboolean{x}{\hlstmt} & x \in \BVars \\
               |&~ \intro*\hlsettrue{x} & x \in \BVars \\
               |&~ \intro*\hlfor{(i,x)}{\oexpr}{\hlstmt} & (i,x) \in \PVars \times \OVars \\
               |&~ \intro*\hlforRev{(i,x)}{\oexpr}{\hlstmt} & (i,x) \in \PVars \times \OVars \\
               |&~ \intro*\hlseq{\hlstmt}{\hlstmt}
    \end{align*}
    \caption{The syntax of \kl{high-level control statements}.}
    \label{fig:high-level-stmt}
\end{figure}

\begin{figure}[h]
    \centering
    \begin{align*}
        \intro*\aexpr :=&~ (\oexpr, p_1, \dots, p_n) & \forall 1 \leq i \leq n, p_i \in \PVars \\
        \intro*\hlfun :=&~ \hlfundef{f}{\aexpr_1, \dots, \aexpr_n}{\hlstmt} & f \in \FunVars \\
        \intro*\hlprogram :=&~ ([\hlfun_1, \dots, \hlfun_n], f) & f \in \FunVars \\
    \end{align*}
    \caption{The syntax of \kl{high-level for-programs}.}
    \label{fig:high-level-program}
\end{figure}

%
%\subsection{Semantics}
%
%\begin{figure}[h]
%    \centering
%    \begin{align*}
%        \semB{ \cdot }{\rho}               \colon&~ \bexpr \to \Bools \\
%        \semB{\btrue}{\rho}               =&~ \top \\
%        \semB{\bfalse}{\rho}              =&~ \bot \\
%        \semB{b}{\rho}                    =&~ \rho(b) & b \in \BVars \\
%        \semB{\bnot{b}}{\rho}             =&~ \neg \semB{b}{\rho} \\
%        \semB{\bbin{b_1}{op}{b_2}}{\rho}  =&~ \semB{b_1}{\rho} \mathbin{op} \semB{b_2}{\rho} \\
%        \semB{\bcomp{i}{op}{j}}{\rho}     =&~ \rho(i) \mathbin{op} \rho(j) \\
%        \semB{\bapp{f}{b}}{\rho}          =&~ \semF{f}{\rho}(\semB{b}{\rho}) \\
%        \semB{\bliteq{c}{o}}{\rho}        =&~ \semC{c}{\rho} = \semO{o}{\rho}
%    \end{align*}
%    \begin{align*}
%        \semC{ \cdot }                  \colon&~ \cexpr \to \NestedWords \\
%        \semC{\cchar{c}}                =&~ c \\
%        \semC{\clist{c_1, \ldots, c_n}} =&~ [c_1, \ldots, c_n]
%    \end{align*}
%    \begin{align*}
%        \semO{ \cdot }{\rho}       \colon&~ \oexpr \to \NestedWords \\
%        \semO{x}{\rho}                  =&~ \rho(x) & x \in \OVars \\
%        \semO{c}{\rho}             =&~ \semC{c} & c \in \cexpr \\
%        \semO{\olist{o_1, \ldots, o_n}}{\rho} =&~ [\semO{o_1}{\rho}, \ldots, \semO{o_n}{\rho}] \\
%        \semO{f(a_1, \dots, a_n)}{\rho} =&~ \semF{f}{\rho}(\semA{a_1}{\rho}, \dots, \semA{a_n}{\rho})
%    \end{align*}
%    \begin{align*}
%        \semA{ \cdot }{\rho}       \colon&~ \aexpr \to \cup_{n \in \Nat} \NestedWords \times \Nat^n \\
%        \semA{ (o, p_1, \dots, p_n) }{\rho} =&~ (\semO{o}{\rho}, (\rho(p_1), \dots, \rho(p_n)))
%    \end{align*}
%    \caption{Semantics of \kl{boolean expressions}, \kl{constant expressions}
%        and \kl{list expressions}, in a given an \kl{evaluation environment} $\rho$
%    mapping variables to $\Bools$, $\NestedWords$, $\Nat$, or a function, depending 
%    on their type.}
%    \label{fig:semantics-expr}
%\end{figure}
%
%\begin{figure}
%    \centering
%    \begin{align*}
%        \semS{\cdot}{\cdot} \colon&~ \hlstmt \to \Env \to \Env \times (\Bools \uplus \NestedWords) \\
%        \semS{\hlif{b}{s_1}{s_2}}{\rho} =&~ \begin{cases}
%            \semS{s_1}{\rho} & \text{if } \semB{b}{\rho} = \top \\
%            \semS{s_2}{\rho} & \text{otherwise}
%        \end{cases} \\
%            \semS{\hlyield{o}}{\rho} =&~ (\rho', [ \, u \, ]) & \text{where } (\rho', u) = \semO{o}{\rho} \\
%            \semS{\hlreturn{o}}{\rho} =&~ \semO{o}{\rho} & \text{if } o \in \oexpr \\
%            \semS{\hlreturn{b}}{\rho} =&~ \semB{b}{\rho} & \text{if } b \in \bexpr \\
%            \semS{\hlletoutput{x}{o}{s}}{\rho} =&~ \semS{s}{\rho'[x \mapsto u]} & \text{where } (\rho',u) = \semO{o}{\rho} \\
%            \semS{\hlletboolean{b}{s}}{\rho} =&~ \semS{s}{\rho[b \mapsto \bot]} \\
%            \semS{\hlsettrue{b}}{\rho} =&~ (\rho[b \mapsto \top], \varepsilon) \\
%            \semS{\hlfor{(i,x)}{o}{s}}{\rho} =&~ \semS{s}{\rho'[i \mapsto 0, x \mapsto u_1]} ; \ldots ; \semS{s}{\rho'[i \mapsto n, x \mapsto u_n]} \\
%                                              &     &\text{where } (\rho', [u_1, \dots, u_n]) = \semA{o}{\rho} \\
%            \\
%            \semS{\hlforRev{(i,x)}{o}{s}}{\rho} =&~ \semC{s}{\rho[i \mapsto p, x \mapsto \semO{o}{\rho}]} \\
%            \semS{\hlseq{s_1}{s_2}}{\rho} =&~ \semS{s_2}{\semS{s_1}{\rho}}
%    \end{align*}
%    \caption{The semantics of \kl{control statements}.
%    We write $\Env$ for the set of all possible evaluation environments
%    and $\varepsilon$ for the empty word.}
%    \label{fig:semantics-control}
%\end{figure}
%
%\begin{figure}
%    \centering
%    \begin{align*}
%        \semF{\hlfundef{f}{a_1, \dots, a_n}{s}}{\rho} =&~ \lambda x_1, \dots, x_n. \semS{s}{\rho[x_1 \mapsto \semO{a_1}{\rho}, \dots, x_n \mapsto \semO{a_n}{\rho}]}
%    \end{align*}
%    \caption{The semantics of functions.}
%    \label{fig:semantics-functions}
%\end{figure}

\begin{figure}
\begin{align*}
    \mathsf{arg} ::=&~ (\TOut[n],\ell) & \ell \in \Nat \\
    \mathsf{fun} ::=&~ 
           \mathsf{arg}_1 \times \cdots \times \mathsf{arg}_k \to \TBool \\
    \mid&~ \mathsf{arg}_1 \times \cdots \times \mathsf{arg}_k \to \TOut[n] 
\end{align*}
\caption{Possible types of \kl{for-programs} and their functions.}
\label{fig:typing-for-programs}
\end{figure}

% Basic typing judgments
\begin{figure}[h]
    \begin{prooftree}
    \AxiomC{}
    \RightLabel{(T-True)}
    \UnaryInfC{$\Gamma \vdash \btrue : \TBool$}
    \end{prooftree}

    \begin{prooftree}
    \AxiomC{}
    \RightLabel{(T-False)}
    \UnaryInfC{$\Gamma \vdash \bfalse : \TBool$}
    \end{prooftree}

    % Boolean expressions
    \begin{prooftree}
    \AxiomC{$\Gamma \vdash e : \TBool$}
    \RightLabel{(T-Not)}
    \UnaryInfC{$\Gamma \vdash \bnot{e} : \TBool$}
    \end{prooftree}

    \begin{prooftree}
    \AxiomC{$\Gamma \vdash e_1 : \TBool$}
    \AxiomC{$\Gamma \vdash e_2 : \TBool$}
    \RightLabel{(T-BBin)}
    \BinaryInfC{$\Gamma \vdash \bbin{e_1}{op}{e_2} : \TBool$}
    \end{prooftree}

    % Position comparisons
    \begin{prooftree}
    \AxiomC{$\Gamma \vdash i :  \TPos[o_i]$}
    \AxiomC{$\Gamma \vdash j : \TPos[o_j]$}
    \AxiomC{$o_i = o_j$}
    \RightLabel{(T-PComp)}
    \TrinaryInfC{$\Gamma \vdash \bcomp{i}{op}{j} : \TBool$}
    \end{prooftree}

    \caption{Typing rules for boolean expressions.}
    \label{fig:typing-bool}
\end{figure}

\begin{figure}[h]
    % Output expressions
    \begin{prooftree}
    \AxiomC{}
    \RightLabel{(T-OVar)}
    \UnaryInfC{$\Gamma, x : \TOut[n] \vdash x : \TOut[n]$}
    \end{prooftree}

    \begin{prooftree}
        \AxiomC{$\Gamma \vdash e_i : \TOut[n]$ for all $i$}
    \RightLabel{(T-OList)}
    \UnaryInfC{$\Gamma \vdash \olist{e_1,\ldots,e_n} : \TOut[n+1]$}
    \end{prooftree}

    \caption{Typing rules for \kl{list expressions} and \kl{constant expressions}.}
    \label{fig:typing-output}
\end{figure}

\begin{figure}
% Statements
\begin{prooftree}
\AxiomC{$\Gamma \vdash e : \TBool$}
\AxiomC{$\Gamma \vdash s_1 : \tau$}
\AxiomC{$\Gamma \vdash s_2 : \tau$}
\RightLabel{(T-If)}
\TrinaryInfC{$\Gamma \vdash \hlif{e}{s_1}{s_2} : \tau$}
\end{prooftree}

\begin{prooftree}
\AxiomC{$\Gamma \vdash e : \TOut[n]$}
\RightLabel{(T-Yield)}
\UnaryInfC{$\Gamma \vdash \hlyield{e} : \TOut[n+1]$}
\end{prooftree}

\begin{prooftree}
\AxiomC{$\Gamma \vdash e : \TOut[n]$}
\RightLabel{(T-Return)}
\UnaryInfC{$\Gamma \vdash \hlreturn{e} : \TOut[n]$}
\end{prooftree}

\begin{prooftree}
\AxiomC{$\Gamma \vdash e : \TBool$}
\RightLabel{(T-Return)}
\UnaryInfC{$\Gamma \vdash \hlreturn{e} : \TBool$}
\end{prooftree}

% Let bindings
\begin{prooftree}
\AxiomC{$\Gamma \vdash e : \TOut[n]$}
\AxiomC{$\Gamma, x:\TOut[n] \vdash s : \tau$}
\RightLabel{(T-LetOut)}
\BinaryInfC{$\Gamma \vdash \hlletoutput{x}{e}{s} : \tau$}
\end{prooftree}

\begin{prooftree}
\AxiomC{$\Gamma, x:\TBool \vdash s : \tau$}
\RightLabel{(T-LetBool)}
\UnaryInfC{$\Gamma \vdash \hlletboolean{x}{s} : \tau$}
\end{prooftree}

% For loops
\begin{prooftree}
\AxiomC{$\Gamma \vdash o : \TOut[n]$}
\AxiomC{$\Gamma, i:\TPos[o], x:\TOut[n] \vdash s : \tau$}
\AxiomC{$n > 0$}
\RightLabel{(T-For)}
\TrinaryInfC{$\Gamma \vdash \hlfor{(i,x)}{o}{s} : \tau$}
\end{prooftree}

\begin{prooftree}
\AxiomC{$\Gamma \vdash o : \TOut[n]$}
\AxiomC{$\Gamma, i:\TPos[o], x:\TOut \vdash s : \tau$}
\AxiomC{$n > 0$}
\RightLabel{(T-ForRev)}
\TrinaryInfC{$\Gamma \vdash \hlforRev{(i,x)}{o}{s} : \tau$}
\end{prooftree}

\begin{prooftree}
\AxiomC{}
\RightLabel{(T-SetTrue)}
\UnaryInfC{$\Gamma \vdash \hlsettrue{x} : \tau$}
\end{prooftree}

% Sequence
\begin{prooftree}
\AxiomC{$\Gamma \vdash s_1 : \tau$}
\AxiomC{$\Gamma \vdash s_2 : \tau$}
\RightLabel{(T-Seq)}
\BinaryInfC{$\Gamma \vdash \hlseq{s_1}{s_2} : \tau$}
\end{prooftree}
\caption{Typing rules for control statements.}
\label{fig:typing-control}
\end{figure}

\begin{figure}
\begin{prooftree}
    % aexpressions 
    \AxiomC{$\Gamma \vdash o : \TOut[n]$}
    \AxiomC{$\Gamma \vdash p_i : \TPos[o]$ for all $1 \leq i \leq n$}
    \RightLabel{(T-AExpr)}
    \BinaryInfC{$\Gamma \vdash (o, p_1, \dots, p_n) : (\TOut[n], n)$}
\end{prooftree}

% Function definition
\begin{prooftree}
\AxiomC{$\Gamma \vdash a_i : \tau_i$ for all $i$}
\AxiomC{$\Gamma \vdash s : \tau$}
\RightLabel{(T-Fun)}
\BinaryInfC{$\Gamma \vdash \hlfundef{f}{a_1, \dots, a_n}{s} : (\tau_1,\ldots,\tau_n) \to \tau$}
\end{prooftree}

\begin{prooftree}
    \AxiomC{$\Gamma \vdash f : (\tau_1,\ldots,\tau_n) \to \tau$}
    \AxiomC{$\Gamma \vdash a_i : \tau_i$ for all $1 \leq i \leq n$}
    \RightLabel{(T-App)}
    \BinaryInfC{$\Gamma \vdash f(a_1,\ldots,a_n) : \tau$}
\end{prooftree}

% Program
\begin{prooftree}
\AxiomC{$\Gamma, \seqof{f_j : \tau_j}[j < i] \vdash f_i : \tau_i$ for all $1 \leq i \leq n$}
\AxiomC{$f = f_j$ for some $1 \leq j \leq n$}
\RightLabel{(T-Prog)}
\BinaryInfC{$\Gamma \vdash ([f_1, \dots, f_n], f) : \tau_j$}
\end{prooftree}
\caption{Typing rules of \kl{for-programs}.}
\label{fig:typing-high-level}
\end{figure}


\begin{figure}
    \centering

\begin{prooftree}
    \AxiomC{$\Gamma' \vdash s : \TBool$}
    \AxiomC{$\Gamma' \subseteq \Gamma$}
    \AxiomC{$\Gamma'$ contains no boolean variables}
    \RightLabel{(B-Gen)}
    \TrinaryInfC{$\Gamma \vdash \bgen{ s } \colon \TBool$}
\end{prooftree}

\begin{prooftree}
    \AxiomC{$\Gamma' \vdash s : \TOut[n]$}
    \AxiomC{$\Gamma' \subseteq \Gamma$}
    \AxiomC{$\Gamma'$ contains no boolean variables}
    \RightLabel{(L-Gen)}
    \TrinaryInfC{$\Gamma \vdash \ogen{ s } \colon \TOut[n]$}
\end{prooftree}

    \caption{The syntax of \kl{generator expressions}.}
    \label{fig:generators}
\end{figure}



\end{document}
