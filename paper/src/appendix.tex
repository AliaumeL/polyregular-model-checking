\section{Proofs}

\begin{lemma}
    \label{lem:fo-emptiness}
    \proofref{lem:fo-emptiness}
    The \kl{emptiness} problem for the \kl{first-order logic on words} is decidable for the infinite alphabet $\mathcal{D}$.
\end{lemma}
\begin{proof}
    Take a formula $\varphi$ and observe that is contains only a finite number of constants from $\mathcal{D}$ -- call this set $A$.
    It is not hard to see that the truth value of $\varphi$ is \kl{supported} by $A$: for every function $f : \mathcal{D} \to \mathcal{D}$
    that does not touch 
    elements of $A$, the truth value of $\varphi$ is the same for $w$ and $f^*(w)$. (Remember that 
    $f^*$ is the pointwise application of $f$). 
    Let $\mathtt{blank} \in \mathcal{D}$ be a letter that does not appear in $A$,
    and observe that the formula $\varphi$ is satisfied for some word in $\mathcal{D}^*$ if and only if it is satisfied by
    some word in $(A \cup \{\mathtt{blank}\})^*$. Indeed, if we take a function $g: \mathcal{D} \to \mathcal{D}$ that does not touch elements of $A$
    and maps all other letters to $\mathtt{blank}$, we can use it to map $\mathcal{D}^*$ to $(A \cup \{\mathtt{blank}\})^*$ in a way 
    that preserves the truth value of $\varphi$.
    This finishes the proof of the lemma, as we have reduced the general problem to a finite alphabet $A \cup \{\mathtt{blank}\}$.
\end{proof}

\section{Encoding Equality of Subwords}

\begin{figure}
    \centering
\begin{Shaded}
\begin{Highlighting}[]
\KeywordTok{def}\NormalTok{ eq(u, v):}
    \ControlFlowTok{for}\NormalTok{ (i, ui) }\KeywordTok{in} \BuiltInTok{enumerate}\NormalTok{(u):}
        \ControlFlowTok{for}\NormalTok{ (j, vj) }\KeywordTok{in} \BuiltInTok{enumerate}\NormalTok{(v):}
            \ControlFlowTok{if}\NormalTok{ i }\OperatorTok{==}\NormalTok{ j }\KeywordTok{and}\NormalTok{ ui }\OperatorTok{!=}\NormalTok{ vj:}
                \ControlFlowTok{return} \VariableTok{False}
    \ControlFlowTok{return} \VariableTok{True}
\end{Highlighting}
\end{Shaded}
\caption{Encoding the equality of two words $u$ and $v$ in Python,
using a comparison between indices of two different lists.}
\label{fig:eq-def-different-indices}
\end{figure}

\begin{figure}
    \centering
\begin{Shaded}
\begin{Highlighting}[]
\KeywordTok{def}\NormalTok{ switch(b, u, v):}
    \ControlFlowTok{if}\NormalTok{ b:}
        \ControlFlowTok{return}\NormalTok{ u}
    \ControlFlowTok{else}\NormalTok{:}
        \ControlFlowTok{return}\NormalTok{ v}

\KeywordTok{def}\NormalTok{ eq(u, v):}
\NormalTok{    b }\OperatorTok{=} \VariableTok{False}
    \ControlFlowTok{for}\NormalTok{ (i, ui) }\KeywordTok{in} \BuiltInTok{enumerate}\NormalTok{(switch(b, u, v)):}
\NormalTok{        b }\OperatorTok{=} \VariableTok{True}
        \ControlFlowTok{for}\NormalTok{ (j, vj) }\KeywordTok{in} \BuiltInTok{enumerate}\NormalTok{(switch(b, u, v)):}
            \ControlFlowTok{if}\NormalTok{ i }\OperatorTok{==}\NormalTok{ j }\KeywordTok{and}\NormalTok{ ui }\OperatorTok{!=}\NormalTok{ vj:}
                \ControlFlowTok{return} \VariableTok{False}
    \ControlFlowTok{return} \VariableTok{True}
\end{Highlighting}
\end{Shaded}
\caption{Encoding the equality of two words $u$ and $v$ in Python,
using a function taking a boolean as input.}
\label{fig:eq-def-boolean}
\end{figure}

\begin{figure}
    \centering
\begin{Shaded}
\begin{Highlighting}[]
\KeywordTok{def}\NormalTok{ eq(u, v):}
\NormalTok{    w }\OperatorTok{=}\NormalTok{ u}
    \ControlFlowTok{for}\NormalTok{ (i, ui) }\KeywordTok{in} \BuiltInTok{enumerate}\NormalTok{(w):}
\NormalTok{        w }\OperatorTok{=}\NormalTok{ v}
        \ControlFlowTok{for}\NormalTok{ (j, vj) }\KeywordTok{in} \BuiltInTok{enumerate}\NormalTok{(w):}
            \ControlFlowTok{if}\NormalTok{ i }\OperatorTok{==}\NormalTok{ j }\KeywordTok{and}\NormalTok{ ui }\OperatorTok{!=}\NormalTok{ vj:}
                \ControlFlowTok{return} \VariableTok{False}
    \ControlFlowTok{return} \VariableTok{True}
\end{Highlighting}
\end{Shaded}
\caption{Encoding the equality of two words $u$ and $v$ in Python,
using the shadowing of a variable to switch between two lists.}
\label{fig:eq-def-shadowing}
\end{figure}

\section{Subwords Containing the Substring $ab$}

\begin{figure}
    \centering
\begin{Shaded}
\begin{Highlighting}[numbers=left]
\KeywordTok{def}\NormalTok{ getBetween( l : }\DataTypeTok{[Char]} \KeywordTok{with}\NormalTok{ (i,j) ) : }\DataTypeTok{[Char]}\NormalTok{ := }
\KeywordTok{    for}\NormalTok{ (k,c) }\KeywordTok{in} \KeywordTok{enumerate}\NormalTok{(l) }\KeywordTok{do}
        \KeywordTok{if}\NormalTok{ i \textless{}= k }\KeywordTok{and}\NormalTok{ k \textless{}= j }\KeywordTok{then}
            \KeywordTok{yield}\NormalTok{ c}
        \KeywordTok{endif}
    \KeywordTok{done}

\KeywordTok{def}\NormalTok{ containsAB(w : }\DataTypeTok{[Char]}\NormalTok{) : }\DataTypeTok{Bool}\NormalTok{ := }
    \KeywordTok{let} \KeywordTok{mut}\NormalTok{ seen\_a := }\KeywordTok{False} \KeywordTok{in} 
\KeywordTok{    for}\NormalTok{ (i, c) }\KeywordTok{in} \KeywordTok{enumerate}\NormalTok{(w) }\KeywordTok{do}
    \KeywordTok{if}\NormalTok{ c === }\StringTok{\textquotesingle{}a\textquotesingle{} }\KeywordTok{then}
\NormalTok{            seen\_a := }\KeywordTok{True}
    \KeywordTok{else} \KeywordTok{if}\NormalTok{ c === }\StringTok{\textquotesingle{}b\textquotesingle{} }\KeywordTok{and}\NormalTok{ seen\_a }\KeywordTok{then}
            \KeywordTok{return} \KeywordTok{True}
        \KeywordTok{endif} \KeywordTok{endif}
    \KeywordTok{done}
    \KeywordTok{return} \KeywordTok{False}

\KeywordTok{def}\NormalTok{ subwordsWithAB(w : }\DataTypeTok{[Char]}\NormalTok{) : }\DataTypeTok{[[Char]]}\NormalTok{ := }
\KeywordTok{    for}\NormalTok{ (i,c) }\KeywordTok{in} \KeywordTok{enumerate}\NormalTok{(w) }\KeywordTok{do}
\KeywordTok{        for}\NormalTok{ (j,d) }\KeywordTok{in} \KeywordTok{reversed}\NormalTok{(}\KeywordTok{enumerate}\NormalTok{(w)) }\KeywordTok{do}
            \KeywordTok{let}\NormalTok{ s := getBetween(w }\KeywordTok{with}\NormalTok{ (i,j)) }\KeywordTok{in}
            \KeywordTok{if}\NormalTok{ containsAB(s) }\KeywordTok{then}
                \KeywordTok{yield}\NormalTok{ s}
            \KeywordTok{endif}
        \KeywordTok{done}
    \KeywordTok{done}

\KeywordTok{def}\NormalTok{ main (w : }\DataTypeTok{[Char]}\NormalTok{) : }\DataTypeTok{[Char]}\NormalTok{ := }
    \KeywordTok{let}\NormalTok{ subwrds := subwordsWithAB(w) }\KeywordTok{in}
\KeywordTok{    for}\NormalTok{ (j,s) }\KeywordTok{in} \KeywordTok{enumerate}\NormalTok{(subwrds) }\KeywordTok{do}
\KeywordTok{        for}\NormalTok{ (i,c) }\KeywordTok{in} \KeywordTok{enumerate}\NormalTok{(s) }\KeywordTok{do}
            \KeywordTok{yield}\NormalTok{ c}
        \KeywordTok{done}
        \KeywordTok{yield}\StringTok{ \textquotesingle{}\#\textquotesingle{}}
    \KeywordTok{done}
\end{Highlighting}
\end{Shaded}
\caption{The \kl{for-program} computing all subwords of a word containing the substring $ab$,
corresponding to the Python code in \cref{fig:python-example-nested}.}
\label{fig:high-level-example-nested}
\end{figure}

\begin{figure}
    \centering
\begin{Shaded}
\begin{Highlighting}[numbers=left]
\KeywordTok{for}\NormalTok{ i }\KeywordTok{in} \KeywordTok{input} \KeywordTok{do}
    \KeywordTok{for}\NormalTok{ j }\KeywordTok{in} \KeywordTok{reversed}\NormalTok{(}\KeywordTok{input}\NormalTok{) }\KeywordTok{do}
        \KeywordTok{let}\NormalTok{ b2, b3, b4 := }\KeywordTok{false} \KeywordTok{in}
        \KeywordTok{for}\NormalTok{ k }\KeywordTok{in} \KeywordTok{input} \KeywordTok{do}
            \KeywordTok{if}\NormalTok{ (i <= k) and (k <= j) }\KeywordTok{then}
                \KeywordTok{if}\NormalTok{ label(k) == \textquotesingle{}a\textquotesingle{} }\KeywordTok{then}
\NormalTok{                    b4 := }\KeywordTok{true}
                \KeywordTok{else}
                    \KeywordTok{if}\NormalTok{ (label(k) == \textquotesingle{}b\textquotesingle{}) and (b4) }\KeywordTok{then}
                        \KeywordTok{if}\NormalTok{ b3 }\KeywordTok{then}
                            \KeywordTok{skip}
                        \KeywordTok{else}
\NormalTok{                            b3 := }\KeywordTok{true}
\NormalTok{                            b2 := }\KeywordTok{true}
                        \KeywordTok{endif}
                    \KeywordTok{else}
                        \KeywordTok{skip}
                    \KeywordTok{endif}
                \KeywordTok{endif}
            \KeywordTok{else}
                \KeywordTok{skip}
            \KeywordTok{endif}
        \KeywordTok{done}
        \KeywordTok{if}\NormalTok{ b2 }\KeywordTok{then}
            \KeywordTok{for}\NormalTok{ l }\KeywordTok{in} \KeywordTok{input} \KeywordTok{do}
                \KeywordTok{if}\NormalTok{ (i <= l) and (l <= j) }\KeywordTok{then}
                    \KeywordTok{print}\NormalTok{ label(l)}
                \KeywordTok{else}
                    \KeywordTok{skip}
                \KeywordTok{endif}
            \KeywordTok{done}
            \KeywordTok{print}\NormalTok{ \textquotesingle{}\#\textquotesingle{}}
        \KeywordTok{else}
            \KeywordTok{skip}
        \KeywordTok{endif}
     \KeywordTok{done}
\KeywordTok{done}
\end{Highlighting}
\end{Shaded}
\caption{The \kl{simple for-program} computing all subwords of a word containing the substring $ab$,
corresponding to the Python code in \cref{fig:python-example-nested}, and obtained
by compiling \cref{fig:high-level-example-nested}.}
\label{fig:low-level-example-nested}
\end{figure}



\section{Syntax and Semantics of High Level Programs}

\subsection{Typing}


\begin{figure}
\begin{align*}
    \mathsf{arg} ::=&~ (\TOut[n],\ell) & \ell \in \Nat \\
    \mathsf{fun} ::=&~ 
           \mathsf{arg}_1 \times \cdots \times \mathsf{arg}_k \to \TBool \\
    \mid&~ \mathsf{arg}_1 \times \cdots \times \mathsf{arg}_k \to \TOut[n] 
\end{align*}
\caption{Typing of \kl{for-programs}.}
\label{fig:typing-for-programs}
\end{figure}

\subsection{Syntax} 



\begin{figure}[h]
    \centering
    \begin{align*}
        \intro*\bBinOp := &~ \land ~|~ \lor ~|~ \Rightarrow ~|~ \Leftrightarrow \\
        \intro*\pCmpOp := &~ = ~|~ \neq ~|~ < ~|~ \leq ~|~ > ~|~ \geq \\
        \intro*\bexpr :=&~ \intro*\btrue ~|~ \intro*\bfalse ~|~ \bnot{\bexpr} \\
               |&~ \bbin{\bexpr}{\bBinOp}{\bexpr}   \\
               |&~ \bcomp{i}{\pCmpOp}{j} & i,j \in \PVars \\
               |&~ \bapp{f}{\bexpr} & f \in \FunVars \\
               |&~ \bliteq{\cexpr}{\oexpr}
    \end{align*}
    \caption{Boolean expressions.}
    \label{fig:bool-expr}
\end{figure}


\begin{figure}[h]
    \centering
    \begin{align*}
        \intro*\cexpr :=&~ \mathsf{char} \; c & c \in \Sigma \\
               |&~ \mathsf{list}(\cexpr_1, \ldots, \cexpr_n)
    \end{align*}
    \caption{Constant expressions.}
    \label{fig:const-expr}
\end{figure}

\begin{figure}[h]
    \centering
    \begin{align*}
        \intro*\oexpr :=&~ x & x \in \OVars \\
               |&~ \cexpr \\
               |&~ \olist{\oexpr_1, \dots,  \oexpr_n}  \\
               |&~ f(\aexpr_1, \dots, \aexpr_n) & f \in \FunVars \\
    \end{align*}
    \caption{Output expressions.}
    \label{fig:out-expr}
\end{figure}

\begin{figure}[h]
    \centering
    \AP
    \begin{align*}
        \intro*\hlstmt :=&~ 
                   \intro*\hlif{\bexpr}{\hlstmt}{\hlstmt} \\
               |&~ \intro*\hlyield{\oexpr} \\
               |&~ \intro*\hlreturn{\oexpr} \\
               |&~ \intro*\hlletoutput{x}{\oexpr}{\hlstmt} & x \in \OVars \\
               |&~ \intro*\hlletboolean{x}{\hlstmt} & x \in \BVars \\
               |&~ \intro*\hlsettrue{x} & x \in \BVars \\
               |&~ \intro*\hlfor{(i,x)}{\oexpr}{\hlstmt} & (i,x) \in \PVars \times \OVars \\
               |&~ \intro*\hlforRev{(i,x)}{\oexpr}{\hlstmt} & (i,x) \in \PVars \times \OVars \\
               |&~ \intro*\hlseq{\hlstmt}{\hlstmt}
    \end{align*}
    \caption{Control statements.}
    \label{fig:high-level-stmt}
\end{figure}

\begin{figure}[h]
    \centering
    \begin{align*}
        \aexpr :=&~ (\oexpr, p_1, \dots, p_n) & \forall 1 \leq i \leq n, p_i \in \PVars \\
        \hlfun :=&~ \hlfundef{f}{\aexpr_1, \dots, \aexpr_n}{\hlstmt} & f \in \FunVars \\
        \hlprogram :=&~ ([\hlfun_1, \dots, \hlfun_n], f) & f \in \FunVars \\
    \end{align*}
    \caption{High Level For Programs.}
    \label{fig:high-level-program}
\end{figure}


\subsection{Semantics}

\begin{figure}[h]
    \centering
    \begin{align*}
        \semB{ \cdot }{\rho}               \colon&~ \bexpr \to \Bools \\
        \semB{\btrue}{\rho}               =&~ \top \\
        \semB{\bfalse}{\rho}              =&~ \bot \\
        \semB{b}{\rho}                    =&~ \rho(b) & b \in \BVars \\
        \semB{\bnot{b}}{\rho}             =&~ \neg \semB{b}{\rho} \\
        \semB{\bbin{b_1}{op}{b_2}}{\rho}  =&~ \semB{b_1}{\rho} \mathbin{op} \semB{b_2}{\rho} \\
        \semB{\bcomp{i}{op}{j}}{\rho}     =&~ \rho(i) \mathbin{op} \rho(j) \\
        \semB{\bapp{f}{b}}{\rho}          =&~ \semF{f}{\rho}(\semB{b}{\rho}) \\
        \semB{\bliteq{c}{o}}{\rho}        =&~ \semC{c}{\rho} = \semO{o}{\rho}
    \end{align*}
    \begin{align*}
        \semC{ \cdot }                  \colon&~ \cexpr \to \OutputType \\
        \semC{\cchar{c}}                =&~ c \\
        \semC{\clist{c_1, \ldots, c_n}} =&~ [c_1, \ldots, c_n]
    \end{align*}
    \begin{align*}
        \semO{ \cdot }{\rho}       \colon&~ \oexpr \to \OutputType \\
        \semO{x}{\rho}                  =&~ \rho(x) & x \in \OVars \\
        \semO{c}{\rho}             =&~ \semC{c} & c \in \cexpr \\
        \semO{\olist{o_1, \ldots, o_n}}{\rho} =&~ [\semO{o_1}{\rho}, \ldots, \semO{o_n}{\rho}] \\
        \semO{f(\aexpr_1, \dots, \aexpr_n)}{\rho} =&~ \semF{f}{\rho}(\semO{\aexpr_1}{\rho}, \dots, \semO{\aexpr_n}{\rho})
    \end{align*}
    \caption{Semantics of \kl{boolean expressions}, \kl{constant expressions}
        and \kl{list expressions}, in a given an \kl{evaluation environment} $\rho$
    mapping variables to $\Bools$, $\NestedWords$, $\Nat$, or a function, depending 
    on their type.}
    \label{fig:semantics-expr}
\end{figure}

\begin{figure}
    \centering
    \begin{align*}
        \semC{\cdot}{\rho} \colon&~ \hlstmt \to \rho \to \rho \\
    \end{align*}
    \caption{Semantics of control statements.}
    \label{fig:semantics-control}
\end{figure}

\begin{figure}
    \centering
    \begin{align*}
        \semF{\cdot}{\rho} \colon&~ \hlfun \to \rho \to \rho \\
        \semF{\hlfundef{f}{\aexpr_1, \dots, \aexpr_n}{\hlstmt}}{\rho} =&~ \lambda x_1, \dots, x_n. \semC{\hlstmt}{\rho[x_1 \mapsto \semO{\aexpr_1}{\rho}, \dots, x_n \mapsto \semO{\aexpr_n}{\rho}]}
    \end{align*}
    \caption{Semantics of functions.}
    \label{fig:semantics-functions}
\end{figure}

While the semantics are standard, let us briefly go
through how \kl{for-programs} are executed by providing a denotational
semantics. To that end, let us map types to their set of values:
$\intro*\semT{\TBool} = \Bools$, $\reintro*\semT{\TPos[o]} = \Nat$ for all $o
\in \oexpr$, and $\reintro*\semT{\TOut[n]} = \NestedWords[n]$. Then, for
arguments to functions, we let $\reintro*\semT{(\TOut[n],\ell)} =
\NestedWords[n] \times \Nat^\ell$ for all $\ell \in \Nat$,
$\semT{\mathsf{arg}_1 \times \cdots \times \mathsf{arg}_k \to \TBool}$ will be
defined as functions from $\semT{\mathsf{arg}_1} \times \cdots \times
\semT{\mathsf{arg}_k}$ to $\Bools$, and similarly for functions with output
type $\TOut[n]$ for some $n \in \Nat$.

\subsection{Typing Rules}

% Basic typing judgments
\begin{figure}[h]
    \begin{prooftree}
    \AxiomC{}
    \RightLabel{(T-True)}
    \UnaryInfC{$\Gamma \vdash \btrue : \TBool$}
    \end{prooftree}

    \begin{prooftree}
    \AxiomC{}
    \RightLabel{(T-False)}
    \UnaryInfC{$\Gamma \vdash \bfalse : \TBool$}
    \end{prooftree}

    % Boolean expressions
    \begin{prooftree}
    \AxiomC{$\Gamma \vdash e : \TBool$}
    \RightLabel{(T-Not)}
    \UnaryInfC{$\Gamma \vdash \bnot{e} : \TBool$}
    \end{prooftree}

    \begin{prooftree}
    \AxiomC{$\Gamma \vdash e_1 : \TBool$}
    \AxiomC{$\Gamma \vdash e_2 : \TBool$}
    \RightLabel{(T-BBin)}
    \BinaryInfC{$\Gamma \vdash \bbin{e_1}{\odot}{e_2} : \TBool$}
    \end{prooftree}

    % Position comparisons
    \begin{prooftree}
    \AxiomC{$\Gamma(i) = \TPos[o_i]$}
    \AxiomC{$\Gamma(j) = \TPos[o_j]$}
    \AxiomC{$o_i = o_j$}
    \RightLabel{(T-PComp)}
    \TrinaryInfC{$\Gamma \vdash \bcomp{i}{\odot}{j} : \TBool$}
    \end{prooftree}

    % Function application in boolean expressions
    \begin{prooftree}
    \AxiomC{$\Delta(f) = (\tau_1,\ldots,\tau_n) \to \TBool$}
    \AxiomC{$\Gamma \vdash e : \tau_1$}
    \RightLabel{(T-BApp)}
    \BinaryInfC{$\Gamma \vdash \bapp{f}{e} : \TBool$}
    \end{prooftree}
    \caption{Typing rules for boolean expressions.}
    \label{fig:typing-bool}
\end{figure}

\begin{figure}[h]
    % Output expressions
    \begin{prooftree}
    \AxiomC{$\Gamma(x) = \TOut[n]$}
    \RightLabel{(T-OVar)}
    \UnaryInfC{$\Gamma \vdash x : \TOut[n]$}
    \end{prooftree}

    \begin{prooftree}
        \AxiomC{$\Gamma \vdash e_i : \TOut[n]$ for all $i$}
    \RightLabel{(T-OList)}
    \UnaryInfC{$\Gamma \vdash \olist{e_1,\ldots,e_n} : \TOut[n+1]$}
    \end{prooftree}
    \caption{Typing rules for \kl{list expressions} and \kl{constant expressions}.}
    \label{fig:typing-output}
\end{figure}

\begin{figure}
% Statements
\begin{prooftree}
\AxiomC{$\Gamma \vdash e : \TBool$}
\AxiomC{$\Gamma \vdash s_1 : \tau$}
\AxiomC{$\Gamma \vdash s_2 : \tau$}
\RightLabel{(T-If)}
\TrinaryInfC{$\Gamma \vdash \hlif{e}{s_1}{s_2} : \tau$}
\end{prooftree}

\begin{prooftree}
\AxiomC{$\Gamma \vdash e : \TOut[n]$}
\RightLabel{(T-Yield)}
\UnaryInfC{$\Gamma \vdash \hlyield{e} : \TOut[n+1]$}
\end{prooftree}

\begin{prooftree}
\AxiomC{$\Gamma \vdash e : \TOut[n]$}
\RightLabel{(T-Return)}
\UnaryInfC{$\Gamma \vdash \hlreturn{e} : \TOut[n]$}
\end{prooftree}

\begin{prooftree}
\AxiomC{$\Gamma \vdash e : \TBool$}
\RightLabel{(T-Return)}
\UnaryInfC{$\Gamma \vdash \hlreturn{e} : \TBool$}
\end{prooftree}

% Let bindings
\begin{prooftree}
\AxiomC{$\Gamma \vdash e : \TOut[n]$}
\AxiomC{$\Gamma, x:\TOut[n] \vdash s : \tau$}
\RightLabel{(T-LetOut)}
\BinaryInfC{$\Gamma \vdash \hlletoutput{x}{e}{s} : \tau$}
\end{prooftree}

\begin{prooftree}
\AxiomC{$\Gamma, x:\TBool \vdash s : \tau$}
\RightLabel{(T-LetBool)}
\UnaryInfC{$\Gamma \vdash \hlletboolean{x}{s} : \tau$}
\end{prooftree}

% For loops
\begin{prooftree}
\AxiomC{$\Gamma \vdash o : \TOut[n]$}
\AxiomC{$\Gamma, i:\TPos[o], x:\TOut[n] \vdash s : \tau$}
\AxiomC{$n > 0$}
\RightLabel{(T-For)}
\TrinaryInfC{$\Gamma \vdash \hlfor{(i,x)}{o}{s} : \tau$}
\end{prooftree}

\begin{prooftree}
\AxiomC{$\Gamma \vdash o : \TOut[n]$}
\AxiomC{$\Gamma, i:\TPos[o], x:\TOut \vdash s : \tau$}
\AxiomC{$n > 0$}
\RightLabel{(T-ForRev)}
\TrinaryInfC{$\Gamma \vdash \hlforRev{(i,x)}{o}{s} : \tau$}
\end{prooftree}

\begin{prooftree}
\AxiomC{}
\RightLabel{(T-SetTrue)}
\UnaryInfC{$\Gamma \vdash \hlsettrue{x} : \tau$}
\end{prooftree}

% Sequence
\begin{prooftree}
\AxiomC{$\Gamma \vdash s_1 : \tau$}
\AxiomC{$\Gamma \vdash s_2 : \tau$}
\RightLabel{(T-Seq)}
\BinaryInfC{$\Gamma \vdash \hlseq{s_1}{s_2} : \tau$}
\end{prooftree}
\caption{Typing rules for control statements.}
\label{fig:typing-control}
\end{figure}

\begin{figure}
% Function definition
\begin{prooftree}
\AxiomC{$\Gamma \vdash \aexpr_i : \tau_i$ for all $i$}
\AxiomC{$\Gamma \vdash s : \tau$}
\RightLabel{(T-Fun)}
\BinaryInfC{$\Delta \vdash \hlfundef{f}{(\aexpr_1,\ldots,\aexpr_n)}{s} : (\tau_1,\ldots,\tau_n) \to \tau$}
\end{prooftree}

% Program
\begin{prooftree}
\AxiomC{$\Delta \vdash \hlfun_i : \tau_i$ for all $i$}
\AxiomC{$f \in \FunVars$}
\RightLabel{(T-Prog)}
\BinaryInfC{$\vdash ([\hlfun_1,\ldots,\hlfun_n], f) : \tau_f$}
\end{prooftree}
\caption{Typing rules of \kl{for-programs}.}
\label{fig:typing-high-level}
\end{figure}

\section{Program Counter Examples}

\begin{lemma}
    \label{lem:umc-equality-nested-words}
    Allowing unrestricted equality checks between two \kl{nested words}
    results in the undecidability of the model checking problem.
\end{lemma}
\begin{proof}
        \begin{example}
            For every instance of the Post Correspondence Problem (PCP), we can 
            construct a function \texttt{f(x : list[list[Char]]) : Bool}
            in the \kl{high-level language} with unrestricted equality checks, such 
            that $f(x) = \btrue$ if and only if $x$ encodes a solution to the PCP instance.
            For example the PCP instance $\{ (ab, a), (b, aa), (ba, b) \}$ can be encoded 
            as the following function:
            \begin{verbatim}
                \input{programs/pcp_instance.py}
            \end{verbatim}
        \end{example}
\end{proof}
