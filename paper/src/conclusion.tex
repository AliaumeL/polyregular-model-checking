% LTeX: language=EN
\section{Conclusion}
\label{sec:conclusion}


\subsection{Further work}
\begin{enumerate}
    \item \textbf{Optimization.}
    \begin{enumerate}
        \item \emph{Simple-for programs.} The benchmarks indicate that the most promising optimization direction is minimizing the boolean depth.
        This can be approached in two ways: post-compilation optimization and improving the initial code generation. 
        We have already implemented basic post-compilation optimizations,
        such as constant propagation and dead code elimination,
        but there is significant potential for further improvements.
    
        So far, we have not implemented any optimizations in the translation process itself, but we believe that this could be a fruitful direction.
        Based on the benchmarks, we believe that \kl{elimination of Literal Equality} is a particularly costly step of the translation
        that introduces a large number of boolean variables. We should investigate whether this can be mitigated. Since, the boolean variables introduced
        in this step are used to simulate counters, it might be useful to introduce explicit counters and position successor operations (i.e. $x + 1$)
        in the language of simple-for programs.
    
        \item \emph{First-order interpretation.} At this level, optimization should focus on reducing quantifier depth.
        As with simple-for programs, this can be approached either by applying static optimizations
        to the generated formulas or by improving the initial translation process. 
        So far, we have not implemented static formula optimizations, relying instead on solvers' capabilities.
        However, it might be still beneficial to explore this direction.
    
        On the translation side, we have implemented one significant optimization: Instead of structuring sequential compositions in a left-associative sequence,
        we structure them in a balanced binary-tree fashion. For example, instead of \(\psi_1 (\psi_2, (\psi_3, \psi_4))\),
        we construct \(((\psi_1, \psi_2), (\psi_3, \psi_4))\), reducing  the overhead quantifier depth from \(n\) to \(\log n\).
        Additional potential optimizations include eliminating quantification while composing simple programs.
        For example, instead of computing the composition \( [[x := \text{true}]]; \psi \) using quantification,
        we could computing directly with the following substitution \(\psi\) with \(\psi[\text{In } x := \text{true}]\).
        We should also explore simplifying some formulas by using arithmetic operations such as successor, predecessor or addition.
    
        The final potential optimization that we would like to discuss considers the treatment of loops. It is based on the observation that 
        if a loop tracks $n$ variables and has already undergone $n$ updates, we do not need to need to verify \kl{completes} of the loop, 
        as no more state changes are possible. Based on this observation, we might be able to significantly reduce the number of universal 
        quantification in the generated formulas.
    \end{enumerate}

    \item \textbf{Understanding the solvers.} The performance of the solvers is highly dependent on the structure of the formulas.
    We would like to investigate how the solvers work internally and how we can structure the formulas to make them more efficient.
    For example, reducing the number of universal quantifier, described at the previous item might improve the performance of MONA.
    An interesting research direction would be to reduce the verification problem to emptiness of LTL formulas (rather than first-order formulas on
    words) and use LTL solvers.

    \item \textbf{Modular verification.} Expanding the for loops is a quite expensive step of the translation process. For this reason,
     it would be very helpful to verify statements of the form \texttt{for (i, e) in enumerate(f(x)) do s done}, based on the 
     specification of $f$ given as a Hoare triple, rather than by expanding the for loop. However, as for now our approach is not 
     modular, it is unclear whether this is possible. 

    \item \textbf{MSO and unrestricted booleans.} As mentioned in
        \cref{sec:high_level}, the version of \kl{high-level language} with
        unrestricted booleans is also decidable, but it requires the use of
        monadic second-order logic (MSO) for word instead of first-order logic.
        This logic can be handled by the MONA solver, and it might be
        interesting to implement and benchmark the unrestricted version of the
        language.

    \item \textbf{Richer type system.} For now, the the type system of the
        high-level language is very simple, with list being the only
        constructor for nested words. The language could be extended with more
        complex types, such as pairs and records. This would make our language
        more usable, making it feasible, to use it to implement transformations
        on more complex data structures (such as JSON). This would require
        extending the specification language to structured data types,
        bypassing the current limitation that we can only verify
        transformations that ultimately operate on words.
    
    \item \textbf{Integrating with existing tools.} For
        example, we could implement a translation from a subset of Python to
        our high-level language, allowing users to verify parts of their Python
        programs. Another interesting project, would be to
        integrate our tool with the Why3 platform, allowing the users
        to discharge the certification of suitable goals to our tool.

\end{enumerate}
