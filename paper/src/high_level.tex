% LTeX: language=EN
\section{High Level For Programs}
\label{sec:high-level}

\AP In this section, we introduce our \kl{high-level language} for
describing list-manipulating functions which can be seen as a subset of
\texttt{Python}, which we call \reintro{(high-level) for-programs}. Our goal is
to reason algorithmically about the programs written in this language, so it
needs to be highly restricted. To illustrate those restrictions, let us present
in \cref{fig:python-example-nested}
a comprehensive example written in a subset of \texttt{Python}.\footnote{The
corresponding program in the syntax accepted by our solver is given in
\cref{fig:high-level-example-nested}.}

\begin{figure}[h]
    \centering
    \begin{Shaded}
\begin{Highlighting}[numbers=left]
\KeywordTok{def}\NormalTok{ getBetween(l, i, j):}
    \CommentTok{""" Get elements between i and j """}
    \ControlFlowTok{for}\NormalTok{ (k, c) }\KeywordTok{in} \BuiltInTok{enumerate}\NormalTok{(l):}
        \ControlFlowTok{if}\NormalTok{ i }\OperatorTok{\textless{}=}\NormalTok{ k }\KeywordTok{and}\NormalTok{ k }\OperatorTok{\textless{}=}\NormalTok{ j:} \circled{1}{code-comparisons}
            \ControlFlowTok{yield}\NormalTok{ c} \circled{2}{code-yield}

\KeywordTok{def}\NormalTok{ containsAB(w):}
    \CommentTok{""" Contains "ab" as a subsequence """}
\NormalTok{    seen\_a }\OperatorTok{=} \VariableTok{False} \circled{3}{code-mut-bool}
    \ControlFlowTok{for}\NormalTok{ (x, c) }\KeywordTok{in} \BuiltInTok{enumerate}\NormalTok{(w):}
        \ControlFlowTok{if}\NormalTok{ c }\OperatorTok{==} \StringTok{"a"}\NormalTok{:} \circled{4}{code-string-comp}
    \NormalTok{            seen\_a }\OperatorTok{=} \VariableTok{True} \circled{5}{code-set-true}
        \ControlFlowTok{elif}\NormalTok{ seen\_a }\KeywordTok{and}\NormalTok{ c }\OperatorTok{==} \StringTok{"b"}\NormalTok{:}
            \ControlFlowTok{return} \VariableTok{True}
    \ControlFlowTok{return} \VariableTok{False}

\KeywordTok{def}\NormalTok{ subwordsWithAB(word):}
    \CommentTok{""" Get subwords that contain "ab" """}
    \ControlFlowTok{for}\NormalTok{ (i,c) }\KeywordTok{in} \BuiltInTok{enumerate}\NormalTok{(word):} \circled{6}{code-enumerate}
        \ControlFlowTok{for}\NormalTok{ (j,d) }\KeywordTok{in}\BuiltInTok{ reversed(}\BuiltInTok{enumerate}\NormalTok{(word)):} \circled{7}{code-enumerate-rev}
    \NormalTok{            s }\OperatorTok{=}\NormalTok{ getBetween(word, i, j)} \circled{8}{code-immutable-variable}
            \ControlFlowTok{if}\NormalTok{ containsAB(s):}
                \ControlFlowTok{yield}\NormalTok{ s}
\end{Highlighting}
\end{Shaded}


    \caption{A small Python program that
        outputs all subwords of a given word containing \texttt{ab}
        as a scattered subword}.
    \label{fig:python-example-nested}
\end{figure}

For the sake of readability, we implicitly coerce generators (created using the
\texttt{yield} keyword) to lists. Our programs will only deal with three kinds
of values: booleans ($\intro*\Bools$), non-negative integers ($\Nat$), and
\intro{(nested) words} ($\intro*\NestedWords$), i.e. characters
($\reintro*\NestedWords[0]$), words ($\NestedWords[1]$), lists of words
($\NestedWords[2]$), etc. 
These lists can be created by \emph{yielding} values in a loop, such
as in \circleref{2}{code:yield} of \cref{fig:python-example-nested}.
In order to ensure decidable verification of Hoare triples,\footnote{
    Using \kl{first-order logic on words} as a specification language.
} we
also will enforce the following conditions, which are satisfied in our example:
\begin{enumerate}[label=(\Roman*), ref=R. \Roman*]
    \item \textbf{Loop Constructions.}
        \label{item:loop-constructions}
        We only allow \texttt{for} loops iterating forward
        or backward over a list, as in 
        \circleref{6}{code:enumerate} and \circleref{7}{code:enumerate:rev}.
        In particular, \texttt{while} loops and recursive functions 
        are forbidden, which guarantees termination of our programs.

    \item \textbf{Mutable Variables.} 
        \label{item:mut-variables}
        The only mutable variables are booleans. The
        values of integer variables are introduced by the \texttt{for} loop
        as in \circleref{6}{code:enumerate},
        and their values are fixed during each iteration. Mutable integer
        variables could serve as unrestricted counters, resulting in
        undecidable verification. Similarly, we prohibit mutable list
        variables, as their lengths could be used as counters.
        However, we still allow the use of immutable
        list variables, as in \circleref{8}{code:immutable:variable}.

    \item \textbf{Equality Checks.}
        \label{item:equality-checks}
        We disallow equality
        checks between two \kl{nested words}, 
        unless one of them is a constant expression.
        This is what happens in point \circleref{4}{code:string:comp}
        of our \cref{fig:python-example-nested}.
        Without this restriction, verification would also be undecidable
        (\cref{lem:umc-equality-nested-words}).
        Note that more generally, classical string \emph{algorithms}
        (edit distance, string matching, longest common subsequence, etc.) should not 
        be expressible in our language, since one can easily derive an
        equality check from them.
        
    \item \textbf{Integer Comparisons.} 
        \label{item:integer-comparisons}
        The only allowed operations on integers
        are usual comparisons operators (equality, inequalities).
        However, we only
        allow comparisons between integers that are indices of the
        same list.
        Every integer is associated to a list expression.
        For instance, in points \circleref{6}{code:enumerate} and
        \circleref{7}{code:enumerate:rev} of our example, the variables
        $i$ and $j$ are associated to the same list variable \texttt{word}.
        Similarly, for the comparison 
        of point \circleref{1}{code:comparisons} to be valid,
        the variables $k$, $i$, and $j$ should all be associated to the same 
        list variable $l$.

        To ensure this compatibility, we designed the following type system,
        containing Booleans, \kl{nested words} of a given depth
        (characters are of depth $0$), and integers associated to a \kl{list
        expression} (the set of which is denoted by $\oexpr$, and will
        be defined in \cref{fig:out-expr}):
        \begin{align*}
            \tau ::=~ \intro*\TBool
            ~\mid~ \intro*\TPos[o] 
            ~\mid~ \intro*\TOut[n] 
            \quad 
            n \in \Nat, \,
            o \in \oexpr
            \quad .
        \end{align*}
        These types can be inferred from the context,
        except in the case of function arguments, in which case
        we explicitly specify to which list argument integer variables
        are associated, as shown in \cref{fig:high-level-example-nested}.

        Without this restriction, the equality predicate between two lists can
        be redefined (as shown in \cref{fig:eq-def-different-indices}).


    \item \textbf{Variable Shadowing.} 
        \label{item:variable-shadowing}
          We disallow shadowing of variable names, as it could
          be used to forge the origin of integers, leading to unrestricted comparisons
          (as shown in \cref{fig:eq-def-shadowing}).

    \item \textbf{Boolean Arguments.}
        \label{item:boolean-arguments}
        We disallow functions to take boolean arguments,
        as it
        would allow to forge the origin of integers,
        by considering the function \texttt{switch(b, l1, l2)} which
        returns either \texttt{l1} or \texttt{l2} 
        depending on the value of \texttt{b}
        (as in \cref{fig:eq-def-boolean}).

    \item \textbf{Boolean Updates.} 
        \label{item:boolean-updates}
        Boolean variables are initialized to \texttt{false}
        as in \circleref{3}{code:mutbool}, and
        once they are set to \texttt{true} as in 
        \circleref{5}{code:settrue},
        they cannot be reset to \texttt{false}. 
        We depart here from the semantics of Python by
        considering lexical scoping of variables; in
        particular a variable declared in a loop is not
        accessible outside this loop.

        This restriction allows us to reduce the verification problem to
        the satisfiability of a \kl{first-order formula} on finite words. This
        problem is not only decidable but also solvable by well-engineered
        existing tools, such as automata-based solvers (e.g., \kl{MONA}) and
        classical SMT solvers (e.g., \kl{Z3}, and \kl{CVC5}).
        Without this restriction, the problem would require the use the monadic
        second order logic on words which is still decidable but not supported
        by the SMT solvers. 

\end{enumerate}


\paragraph{Formal Syntax and Typing.} We extend the typing system to functions
by grouping input positions with the list they are associated to. For instance,
the function \texttt{getBetweenIndicesBeforeStop(l, i, j)} has type
$(\TOut[2],2) \to \TOut[2]$, that is, we are given an input list $l$ together
with two pointers to indices of this list. Similarly, the function
\texttt{containsAB(w)} has type $(\TOut[1], 0) \to \TBool$, while the function
\texttt{subwordsWithAB(word)} has type $(\TOut[1], 0) \to \TOut[2]$. We
implemented a linear-time algorithm for the type checking and inference
problems.

\AP The formal syntax of our language is given in
\cref{fig:bool-expr,fig:const-expr,fig:out-expr,fig:high-level-stmt,fig:high-level-program}.
They define the syntax of \intro{boolean expressions} ($\bexpr$),
\intro{constant expressions} ($\cexpr$), \intro{list expressions} ($\oexpr$),
and \intro{control statements} ($\hlstmt$). For readability, we distinguish
boolean variables $\intro*\BVars$ ($b, p, q, \dots$), position variables
$\intro*\PVars$ ($i,j, \dots$), list variables $\intro*\OVars$ ($x,y,u,v,w,
\dots$), and function variables $\intro*\FunVars$ ($f,g,h, \dots$). A
\intro{for-program} is a list of function definitions together with a
\emph{main} function of type $\TOut[1] \to \TOut[1]$.



\paragraph{Semantics.} \AP Given an \intro{evaluation context} that assigns
values (functions, positions, booleans, nested words) to variables, the
semantics of \kl{boolean expressions} into booleans $\Bools$, \kl{constant
expressions} into \kl{nested words}, and \kl{list expressions} into \kl{nested
words} pose no difficulty.
For the \kl{control statements}, there is a crucial design choice regarding
the semantics of backward iteration.
While the semantics of forward iteration is unambiguous, the backward iteration 
$\hlforRev{(i,x)}{l}{s}$
could be understood in two different ways,
provided that $l$ evaluates to a list $[x_0, \dots, x_k]$:
\begin{itemize}
    \item Executing the statement $s$ for pairs
        $(k, x_k), \dots, (0, x_0)$. This corresponds to
        Python's \texttt{for (i,x) in reversed(enumerate(l))}
        ;
    \item Executing the statement $s$ for pairs
        $(0, x_k), \dots, (k, x_0)$.
        This corresponds to
        Python's \texttt{for (i,x) in enumerate(reversed(l))}.
\end{itemize}
As shown in our example program,
we use the first interpretation (see \circleref{7}{code:enumerate:rev}). In fact,
the second interpretation would allow us to define the equality predicate
between two lists, leading to undecidable verification.
