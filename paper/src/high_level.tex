% LTeX: language=EN
\section{High Level For Programs}
\label{sec:high_level}

\AP In this section, we introduce our \intro{high-level language} for
describing list-manipulating functions which can be seen as a subset of
\texttt{Python}. Our goal is to reason algorithmically about the programs
written in this language, so it needs to be highly restricted. To illustrate
those restrictions, let us present a comprehensive example written in a subset
of \texttt{Python}. The corresponding program in the syntax accepted by our
solver is given in \cref{fig:high-level-example-nested}.

\begin{figure}[h]
    \centering
    \begin{Shaded}
\begin{Highlighting}[numbers=left]
\KeywordTok{def}\NormalTok{ getBetween(l, i, j):}
    \CommentTok{""" Get elements between i and j """}
    \ControlFlowTok{for}\NormalTok{ (k, c) }\KeywordTok{in} \BuiltInTok{enumerate}\NormalTok{(l):}
        \ControlFlowTok{if}\NormalTok{ i }\OperatorTok{\textless{}=}\NormalTok{ k }\KeywordTok{and}\NormalTok{ k }\OperatorTok{\textless{}=}\NormalTok{ j:} \circled{1}{code-comparisons}
            \ControlFlowTok{yield}\NormalTok{ c} \circled{2}{code-yield}

\KeywordTok{def}\NormalTok{ containsAB(w):}
    \CommentTok{""" Contains "ab" as a subsequence """}
\NormalTok{    seen\_a }\OperatorTok{=} \VariableTok{False} \circled{3}{code-mut-bool}
    \ControlFlowTok{for}\NormalTok{ (x, c) }\KeywordTok{in} \BuiltInTok{enumerate}\NormalTok{(w):}
        \ControlFlowTok{if}\NormalTok{ c }\OperatorTok{==} \StringTok{"a"}\NormalTok{:} \circled{4}{code-string-comp}
    \NormalTok{            seen\_a }\OperatorTok{=} \VariableTok{True} \circled{5}{code-set-true}
        \ControlFlowTok{elif}\NormalTok{ seen\_a }\KeywordTok{and}\NormalTok{ c }\OperatorTok{==} \StringTok{"b"}\NormalTok{:}
            \ControlFlowTok{return} \VariableTok{True}
    \ControlFlowTok{return} \VariableTok{False}

\KeywordTok{def}\NormalTok{ subwordsWithAB(word):}
    \CommentTok{""" Get subwords that contain "ab" """}
    \ControlFlowTok{for}\NormalTok{ (i,c) }\KeywordTok{in} \BuiltInTok{enumerate}\NormalTok{(word):} \circled{6}{code-enumerate}
        \ControlFlowTok{for}\NormalTok{ (j,d) }\KeywordTok{in}\BuiltInTok{ reversed(}\BuiltInTok{enumerate}\NormalTok{(word)):} \circled{7}{code-enumerate-rev}
    \NormalTok{            s }\OperatorTok{=}\NormalTok{ getBetween(word, i, j)} \circled{8}{code-immutable-variable}
            \ControlFlowTok{if}\NormalTok{ containsAB(s):}
                \ControlFlowTok{yield}\NormalTok{ s}
\end{Highlighting}
\end{Shaded}


    \caption{A small Python program that
        outputs all subwords of a given word containing \texttt{ab}
        as a scattered subword}.
    \label{fig:python-example-nested}
\end{figure}

For the sake of readability, we implicitly coerce generators (created using the
\texttt{yield} keyword) to lists. Our programs will only deal with three kinds
of values: booleans ($\intro*\Bools$), non-negative integers ($\Nat$), and
\intro{(nested) words} ($\intro*\NestedWords$), i.e. characters
($\reintro*\NestedWords[0]$), words ($\NestedWords[1]$), lists of words
($\NestedWords[2]$), etc. 
These lists can be created by \emph{yielding} values in a loop, such
as in \circleref{2}{code:yield}. 
In order to ensure decidable \kl{model checking}, we
also will enforce the following conditions, which are satisfied in our example:
\begin{enumerate}[label=(\Roman*), ref=R. \Roman*]
    \item \textbf{Loop Constructions.}
        \label{item:loop-constructions}
        We only allow \texttt{for} loops iterating forward
        or backward over a list, as in 
        \circleref{6}{code:enumerate} and \circleref{7}{code:enumerate:rev}.
        In particular, \texttt{while} loops and recursive functions 
        are forbidden, which guarantees termination of our programs.

    \item \textbf{Mutable Variables.} 
        \label{item:mut-variables}
        The only mutable variables are booleans. The
        values of integer variables are introduced by the \texttt{for} loop
        as in \circleref{6}{code:enumerate},
        and their values are fixed during each iteration. Mutable integer
        variables could serve as unrestricted counters, resulting in
        undecidable \kl{model checking}. Similarly, we prohibit mutable list
        variables, as their lengths could be used as counters.
        However, we still allow the use of immutable
        list variables, as in \circleref{8}{code:immutable:variable}.

    \item \textbf{Equality Checks.}
        \label{item:equality-checks}
        We disallow equality
        checks between two \kl{nested words}, 
        unless one of them is a constant expression.
        This is what happens in point \circleref{4}{code:string:comp}
        of our \cref{fig:python-example-nested}.
        Without this restriction, \kl{model checking} would be undecidable
        (\cref{lem:umc-equality-nested-words}).
        
    \item \textbf{Integer Comparisons.} 
        \label{item:integer-comparisons}
        The only allowed operations on integers
        are usual comparisons operators (equality, inequalities).
        However, we only
        allow comparisons between integers that are indices of the
        same list.
        Every integer is associated to a list expression.
        For instance, in points \circleref{6}{code:enumerate} and
        \circleref{7}{code:enumerate:rev} of our example, the variables
        $i$ and $j$ are associated to the same list variable \texttt{word}.
        Similarly, in for the comparison 
        of point \circleref{1}{code:comparisons} to be valid,
        the variables $k$,$i$, and $j$ should all be associated to the same 
        list variable $l$.

        To ensure this compatibility, we designed the following type system,
        containing Booleans, \kl{nested words} of a given depth
        (characters are of depth $0$), and integers associated to a list
        expression (the set of which is denoted by $\oexpr$, and will
        be defined later on):
        \begin{align*}
            \tau ::=~ \TBool
            ~\mid~ \TPos[o] 
            ~\mid~ \TOut[n] 
            \quad 
            n \in \Nat, \,
            o \in \oexpr
            \quad .
        \end{align*}
        These types can be inferred from the context,
        except in the case of function arguments, in which case
        we explicitly specify to which list argument integer variables
        are associated.

        Without this restriction, the equality predicate can be 
        defined between two lists, leading to an undecidable
        \kl{model checking problem}.


    \item \textbf{Variable Shadowing.} 
        \label{item:variable-shadowing}
        We disallow shadowing of variable names, as it could
          be used to forge the origin of integers, leading to unrestricted comparisons.
          \begin{verbatim}
            \input{programs/eq_from_idxc_shadowing.py}
          \end{verbatim} 

    \item \textbf{Boolean Arguments.}
        \label{item:boolean-arguments}
        Allowing functions to take boolean arguments
        would allow to forge the origin of integers,
        by considering the function \texttt{switch(l1, l2, b)} which
        returns either \texttt{l1} or \texttt{l2} 
        depending on the value of \texttt{b}.
        \begin{verbatim}
            \input{programs/eq_from_idxc_shadowing.py} 
        \end{verbatim} 

    \item \textbf{Boolean Updates.} 
        Boolean variables are initialized to \texttt{false}
        as in \circleref{3}{code:mutbool}, and
        once they are set to \texttt{true} as in 
        \circleref{5}{code:settrue},
        they cannot be reset to \texttt{false}. 
        We depart here from the semantics of Python by
        considering lexical scoping of variables; in
        particular a variable declared in a loop is not
        accessible outside this loop.

        This restriction allows us to reduce the \kl{model checking problem} to
        the satisfiability of a \kl{first-order formula} on finite words. This
        problem is not only decidable but also solvable by well-engineered
        existing tools, such as automata-based solvers (e.g., \kl{MONA}) and
        classical SMT solvers (e.g., \kl{Alt-Ergo}, \kl{Z3}, and \kl{CVC5}).
        Without this restriction, the problem would require the use the monadic
        second order logic on words which is still decidable but not supported
        by the SMT solvers. 

\end{enumerate}


\paragraph{Formal Syntax and Typing.} 
In order to give a type to our programs, it remains to give a type 
to functions. Because every integer variable is associated to a list
expression, we simplify the syntax of function calls and definitions
by syntactically grouping positions and lists. In the type system,
this is reflected by the creation of a type for \emph{function arguments}
as follows:
\begin{align*}
    \mathsf{arg} ::=&~ (\TOut[n],\ell) & \ell \in \Nat \\
    \mathsf{fun} ::=&~ 
           \mathsf{arg}_1 \times \cdots \times \mathsf{arg}_k \to \TBool \\
    \mid&~ \mathsf{arg}_1 \times \cdots \times \mathsf{arg}_k \to \TOut[n] 
\end{align*}

For instance, the function \texttt{getBetweenIndicesBeforeStop(l, i, j)}
has type $(\TOut[2],2) \to \TOut[2]$, that is, we are given an input list $l$ together
with two pointers to indices of this particular list. 
Similarly, the function \texttt{containsAB(w)} has type $(\TOut[1], 0) \to \TBool$,
while the function \texttt{subwordsWithAB(word)} has type $(\TOut[1], 0) \to \TOut[2]$.

\AP The rest of the syntax of the high-level language is straightforward:
boolean expressions ($\bexpr$), constant expressions ($\cexpr$), list
expressions ($\oexpr$), and control statements ($\hlstmt$) are defined in
\cref{fig:bool-expr,fig:const-expr,fig:out-expr,fig:high-level-stmt}.
For readability, we distinguish boolean variables $\intro*\BVars$ ($b, p, q,
\dots$), position variables $\intro*\PVars$ ($i,j, \dots$), list variables
$\intro*\OVars$ ($x,y,u,v,w, \dots$), and function variables $\intro*\FunVars$
($f,g,h, \dots$). A \intro{high-level program} is a list of function
definitions together with a \emph{main} function. 

\AP We provide a full type system for the \kl{high-level language} in appendix,
that ensures non-shadowing and non-forgeability of the origin of integer
variables.

\begin{lemma}
    \label{lem:type-checking}
    Type checking of \kl{high-level programs} is decidable in linear time, and 
    inference can be performed in linear time.
\end{lemma}


\paragraph{Semantics.} While the semantics are standard, let us briefly go
through how \kl{high-level programs} are executed by providing a denotational
semantics. To that end, let us map types to their set of values:
$\intro*\semT{\TBool} = \Bools$, $\reintro*\semT{\TPos[o]} = \Nat$ for all $o
\in \oexpr$, and $\reintro*\semT{\TOut[n]} = \NestedWords[n]$. Then, for
arguments to functions, we let $\reintro*\semT{(\TOut[n],\ell)} =
\NestedWords[n] \times \Nat^\ell$ for all $\ell \in \Nat$,
$\semT{\mathsf{arg}_1 \times \cdots \times \mathsf{arg}_k \to \TBool}$ will be
defined as functions from $\semT{\mathsf{arg}_1} \times \cdots \times
\semT{\mathsf{arg}_k}$ to $\Bools$, and similarly for functions with output
type $\TOut[n]$ for some $n \in \Nat$.

\AP The evaluation of boolean expressions ($\bexpr$), constant expressions
($\cexpr$), and \kl{list expressions} ($\oexpr$) pose no difficulty and are
described in \cref{fig:semantics-expr}. We therefore have a denotational
semantics where the evaluation of such expressions in an \intro{evaluation
environment} $\rho$ are functions $\semB{\cdot}{\rho} : \bexpr \to \Bools$,
$\semC{\cdot} : \cexpr \to \NestedWords$, and $\semO{\cdot}{\rho} : \oexpr
\to \NestedWords$ respectively.

For the statements, the key point of the semantics
is that the for loops are taken to be of the form
$\hlfor{(i,x)}{o}{s}$ (in the case of a forward iteration) and
$\hlforRev{(i,x)}{o}{s}$ (in the case of a backward iteration).
While the forward iteration is defined as expected, the backward iteration 
could be understood in two ways, assuming that $o$ evaluates to 
a list $(x_0, \dots, x_k)$:
\begin{itemize}
    \item Executing the statement $s$ for every pair 
        $(k, x_k), \dots, (0, x_0)$, that is,
        the Python fragment \texttt{for (i,x) in reversed(enumerate(l))}
        ;
    \item Executing the statement $s$ for every pair
        $(0, x_k), \dots, (k, x_0)$,
        that is, 
        the Python fragment \texttt{for (i,x) in enumerate(reversed(l))}.
\end{itemize}
In our example program \cref{fig:python-example-nested}, 
we used the first interpretation (see \circleref{7}{code:enumerate:rev}). In fact,
the second interpretation would allow us to define the equality predicate
between two input \kl{list expressions}, leading to an undecidable \kl{model checking problem}.

\AP Another technical decision we made is regarding \kl{list expressions} and
the evaluation of statements. If a function $f$ has output type $\TOut[0]$,
then it can only use the statement $\hlreturn{x}$ for some $x$ of type
$\TOut[0]$. However, when a function has an output type $\TOut[n]$ for some $n
> 0$, then it can use both $\hlyield{x}$ and $\hlreturn{y}$ to produce its
output, in which case the output will be the concatenation of all the yields
before the first return, concatenated with the return value.

