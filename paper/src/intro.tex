% LTeX: language=EN
\section{Introduction}
\label{sec:intro}

The goal of this paper is to define a programming language that is expressive
enough to capture a wide range of functions, and yet simple enough to be
amenable to formal verification. Specifically, we want to be able to verify
Hoare triples of the form $\hoaretriple{P}{\texttt{code}}{Q}$, where $P$ and
$Q$ are predicates and \texttt{code} is a program, meaning that whenever the
input satisfies property $P$, the output of the program satisfies property $Q$.

To design our language, we leverage the theory of \kl{polyregular functions}
that originated in the extension of automata theory to string-to-string
functions of polynomial growth --- that is, $|f(w)|$ is bounded by a polynomial
in $|w|$ --- \cite{ENMA02,bojanczyk2018polyregular}. For such functions, it is
decidable whether a given Hoare triple holds, when $P$ and $Q$ are given
as formulas in the monadic second order logic on words \cite[Theorem
1.7]{bojanczyk2018polyregular}. However, this result is of theoretical nature
(no implementation or complexity bounds are given), and writing programs using
this model is cumbersome and error-prone. Furthermore, relying on monadic
second order logic implies that one cannot use the vast majority of SMT
solvers, which only handle first-order logic.

\paragraph{Contributions.} In this paper our contributions are threefold.
First, we introduce a programming language that corresponds to a rich subset of
\texttt{Python}, which we argue is expressive enough to be usable in practice.
Second, we demonstrate that this language can be compiled into a certain type
of \kl{polyregular functions}. Third, we prove that for these polyregular
functions, the verification of \kl{Hoare triples} (specified using
\kl{first-order logic} on words) effectively reduces to a satisfiability
problem of a first-order formula on finite words. While this last step was
known to be theoretically possible, an efficient and effective implementation
was lacking. Because we are using \kl{first-order logic} as a target language,
we are not restricted to using automata based tools like \intro{MONA}
\cite{MONA01}, but can also employ general purpose SMT solvers like \intro{Z3}
\cite{z3} and \intro{CVC5} \cite{cvc5}, generating proof
obligations in the \texttt{SMT-LIB} format \cite{BARRETT17}.

We implemented all of these conversions in a \texttt{Haskell} program, and
tested it on a number of examples with encouraging results.\footnote{An anonymized
    version of our code is available at \repositoryUrl.}


\paragraph{Outline.} The structure of the paper is as follows. We introduce our
\kl{high-level language} in \cref{sec:high-level}. In \cref{sec:polyregular},
we recall the theory of \kl{polyregular functions} by introducing them in terms
of \kl{simple for-programs} and \kl{$\FO$-interpretations}. We will also
provide an efficient reduction of the verification of Hoare triples to the
satisfiability of a \kl{first-order formula on words} in \cref{sec:pullback}.
In order to verify \kl{for-programs}, we compile them into \kl{simple
for-programs} in \cref{sec:htl}, and then compile \kl{simple for-programs} into
\kl{$\FO$-interpretations} in \cref{sec:low-level}. 
%These steps 
%are summarized in the following diagram:
%\begin{center}
%    \begin{tikzpicture}[
%        syntaxNode/.style={
%                    rectangle, draw, 
%                    text width=5em, 
%                    text centered, 
%                    rounded corners, 
%                    minimum height=4em}
%    ]
%        %
%        % Write a tikz picture with nodes explaining the different 
%        % steps of the rewriting system 
%        % 
%        % (a) high-level language
%        % (b) simple for-programs
%        % (c) first-order string-to-string interpretations
%        % (c1) precondition
%        % (c2) postcondition
%        % (d) first-order formula
%        \node[syntaxNode] (a) {\kl{For-program}};
%        \node[syntaxNode, right=of a] (b)  {\kl{Simple for-program}};
%        \node[syntaxNode, right=of b] (c)  {\kl{$\FO$ interpretation}};
%        \node[syntaxNode, right=of c] (d) {\kl{First-order formula}};
%
%        \draw[->] (a) -- node[above,rotate=90,xshift=2.3em,yshift=-0.3em] {\cref{sec:htl}} (b);
%        \draw[->] (b) -- node[above,rotate=90,xshift=2.3em,yshift=-0.3em] {\cref{sec:low-level}} (c);
%        \draw[->] (c) -- node[above,rotate=90,xshift=2.7em,yshift=-0.3em] {\cref{sec:pullback}} (d);
%    \end{tikzpicture}
%\end{center}
Then, in \cref{sec:benchmarks}, we present
benchmarks of our implementation on various examples, discussing
the complexity of the transformations and the main bottlenecks of our approach.
Finally, we conclude in \cref{sec:conclusion} by discussing potential
optimizations and future work.

\paragraph{Related work.} The correspondence between subsets of string to
string functions and first order logic dates back to the origins of automata
theory and the seminal results of \cite{PEPI86,SCHU65,MNPA71}, establishing the
equivalence between \emph{star-free languages}, \emph{first order definable
languages}, and \emph{counter free automata}. Extensions of this correspondence
to functions has been an active area of research \cite{CADA15,MUSC19}, which
we leverage in this work via the theory of \intro{polyregular functions}
\cite{ENMA02,bojanczyk2018polyregular,bojanczyk2019string,bojanczyk2023growth}.

