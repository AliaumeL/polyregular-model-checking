\section{Low Level For Programs and Interpretations}
\label{sec:low-level}

In this section we compile low level For programs to first order interpretations,
allowing the verification of Hoare triples over low level For Programs.

\begin{itemize}
    \item Recall that this was already known, but without any bounds on the 
        complexity of the process.
    \item The key idea is that every print of the program is going to be a 
        \kl{transduction tag}.
    \item The main issue is understanding how boolean variables are modified
        by the program during execution.
\end{itemize}

\subsection{Program Formulas}

\AP In order to compile a low level For program to a first order
interpretation, our main issue is compositionality. To that end, let us
introduce an intermediate \emph{compositional} language talking about the state
of the program. A \intro{program formula} is a tuple $(I, \varphi, O)$ where $I
\subseteq \BVars \cup \PVars \cup \OVars$, and $O \subseteq \BVars$.

\begin{itemize}
    \item A program formula models a \emph{relation} between the input variables
        and output variables. We only care about \emph{functions}.
    \item There is a nice diagrammatic representation of such 
        \kl{program formulas}, where input variables are on the left,
        output variables are on the right.
\end{itemize}

\AP There are two main constructions needed for programs formulas: the
composition (sequential) and the iteration (loop).

