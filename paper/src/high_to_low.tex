% LTeX: language=EN
\section{From High Level to Low Level For Programs}
\label{sec:htl}

\AP In this section, we provide a compilation from \kl{high-level for-programs}
to \kl{simple for-programs}. To smoothen the conversion, we introduce
\intro{generator expressions} to the language, as a way to inline 
function calls. We distinguish between \kl{nested-word} generators $\intro*\ogen{s}$
and boolean generators $\intro*\bgen{s}$.

\paragraph{Generator Expressions.} Let us briefly discuss the new typing rules
and semantics of these \kl{generator expressions}. The meaning of 
$\ogen{s}$ is to evaluate the statement $s$ in the current context and collect its output.
For instance, $\ogen{\hlreturn{x}}$ is equivalent to $x$,
and $\ogen{\hlseq{\hlyield{x}}{\hlyield{y}}}$ is equivalent
to $\olist{x,y}$. Similarly, $\bgen{s}$ is used to evaluate a boolean
statement and return its value. The type of a \kl{generator expression}
is equal to the type of the statement $s$ it contains. Importantly, 
when evaluating the statement $s$ in a generator,
we hide all boolean variables from the \kl{evaluation context}.
In particular, $\hlletboolean{b}{\hlreturn{\bgen{ \hlreturn{b} }}}$ is an \emph{invalid
program}, because the variable $b$ is undefined in the context of the generator
expression $\bgen{b}$. The formal typing rules of \kl{generator expressions}
can be found in
\iflongversion
  \cref{fig:generators}. 
\else
  the full version of this paper.
\fi

Hiding the booleans from the context, ensures that the evaluation order of 
the expressions is irrelevant, allowing us to freely substitute expressions
during the compilation process. 

\paragraph{Rewriting Steps.} We will convert \kl{for-programs} to
\kl{simple for-programs} by a series of rewriting steps listed below.
While most of the steps make can be applied to any \kl{for-program},
some of them only apply to programs of type $\TOut[1] \to \TOut[1]$.
\begin{enumerate}[label=(\Alph*), ref=Step \Alph*]
    \item \label{item:lit_eq_elim} \intro{Elimination of Literal
        Equalities}, i.e., of expressions $\bliteq{c}{o}$ where $c \in \cexpr$
        and $o \in \oexpr$. This is done by replacing those tests with a call
        to a function that checks for equality with the constant $c$ by
        traversing its input. We define these functions by induction on $c$.
        Note that this is only possible because equalities are always
        between a variable and a constant (\ref{item:equality-checks}).

    \item \label{item:lit_elim} \intro{Elimination of Literal
        Productions}, i.e., of constant expressions in the construction of
        $\oexpr$, except single characters. This is done by replacing a
        constant $c$ by a function call. For instance, $\clist{\cchar{a_1}, \cchar{a_2}}$
        is replaced by a call to a function with body
        $\hlseq{\hlyield{\cchar{a_1}}}{\hlyield{\cchar{a_2}}}$.

    \item \label{item:fun_elim} \intro{Elimination of Function Calls},
        by replacing them with \kl{generator expressions}. Given a function $f$
        with body $s$ and arguments $x_1, \dots, x_n$, we replace a call
        $f(a_1, \dots, a_n)$ by $\ogen{ s[ a_1/x_1, \dots, a_n/x_n ] }$
        (or $\bgen{ \cdots }$ for boolean functions).
        This is valid because functions do not take booleans as arguments
        (\ref{item:boolean-arguments}).

    \item \label{item:bool_elim} \intro{Elimination of Boolean
        Generators}. Note that $\bgen{s}$ can only appear in a conditional test, 
        and let us illustrate this step on an example.
        Consider the following statement:
        $\hlif{\bgen{s_1}}{s_2}{s_3}$. We replace it by $\hlletboolean{b_1}{
        (\hlseq{s_1'}{\hlif{b_1}{s_2}{s_3}}) }$, where $s_1'$ is obtained by
        replacing boolean return statements $(\hlreturn{b})$ by assignments of
        the form $(\hlif{b}{\hlsettrue{b_1}}{\hlskip{}})$.
    \item \label{item:let_output_elim} \intro{Elimination of Let
        Output Statements}, i.e., of statements of the form
        $\hlletoutput{x}{e}{s}$. This is done by textually replacing
        $\hlletoutput{x}{e}{s}$ by $s[x \mapsto e]$.

    \item \label{item:return_elim} \intro{Elimination of Return
        Statements} for \kl{list expressions}. 
        First, to make sure that the program does not produce 
        any output after the first return statement, we
        introduce a boolean variable \texttt{has\_returned}, 
        and guard every yield statement by a check on this variable.
        Then, we replace every statement $\hlreturn{e}$ by
        a for loop $\hlfor{(i,x)}{e}{\hlyield{x}}$.
        This is not possible if the return statement is of type $\TOut[0]$,
        and for this edge case, we refer the readers to our implementation.

    \item \label{item:for_loop_exp} \intro{Expansion of For Loops},
        ensuring that every for loop iterates over a single list
        variable. This is the key step of the compilation, and it will be 
        thoroughly explained later in this section.

    \item \label{item:let_bools_top} 
        \intro{Defining booleans at the beginning of for loops}.
        This is a technical step that ensures that all boolean variables 
        are defined at the beginning of the program or at the beginning of a for loop.
        Thanks to the no-shadowing rule (\ref{item:variable-shadowing}), we can 
        safely move all boolean definitions to the top of their scopes.
\end{enumerate}

\begin{theorem}
    \label{thm:rewriting-termination}
    The rewriting steps (\ref{item:lit_eq_elim}--- \ref{item:let_bools_top})
    all terminate and preserve typing. Moreover, normalized \kl{for-programs}
    of type $ \TOut[1] \to \TOut[1]$ are isomorphic to \kl{simple for-programs}.
\end{theorem}

\paragraph{Forward For Loop Expansion.} We now focus on the \kl{expansion of for
loops}, that is, \ref{item:for_loop_exp}. The case of forward
iterations is simpler and will illustrate a first difficulty. We replace each loop of
the form $\hlfor{(i,x)}{\ogen{s_1}}{s_2}$ by the statement $s_1$ where every
statement $\hlyield{e}$ is replaced by $s_2[x \mapsto e]$. This
rewriting is problematic because it leaves the variable $i$ undefined in $s_2$.
The key observation allowing us to circumvent this issue is that the variable
$i$ can only be used in \emph{comparisons}, and can only be compared with 
variables $j$ that are iterating over $\ogen{s_1}$
(thanks to \ref{item:integer-comparisons}). It is therefore sufficient 
to order the outputs of $s_1$ to effectively remove the variable $i$ from the program.

One can recover the ordering between outputs of $s_1$ by storing
the position of the $\hlyield{e}$ responsible for the output,
together with all position variables visible at that point.
Let us illustrate this in a simple example:
\begin{equation*}
    \hlseq{
    (\hlforRev{(j\tikzmark{yieldIndex},y)}{e}{
        (\hlseq{\tikzmark{yield1}\hlyield{y}}
               {\tikzmark{yield2}\hlyield{\cchar{a}}}
        )})
    }{\tikzmark{yield3}\hlyield{\cchar{b}}}
\end{equation*}
\begin{tikzpicture}[overlay, remember picture]
    % Y1 below = 0.5cm, left 0.3cm of pic cs:yield1
    \node (Y1) at ([yshift=-0.5cm, xshift=0.2cm]pic cs:yield1) {$p_1(j)$};
    \node (Y2) at ([yshift=-0.5cm, xshift=0.2cm]pic cs:yield2) {$p_2(j)$};
    \node (Y3) at ([yshift=-0.5cm, xshift=0.2cm]pic cs:yield3) {$p_3$};
    \node (YI) at ([yshift=0cm, xshift=-0.2cm]pic cs:yieldIndex) {};

    \draw[dashed, A2, thick] (Y1) edge[->, bend left] (YI);
    \draw[dashed, A2, thick] (Y2) edge[->, bend left] (YI);
\end{tikzpicture}
\vspace{1em}

In this example, there are three yield statements at
positions $p_1$, $p_2$ and $p_3$. We can compute
the \emph{happens (strictly) before} relation between outputs 
of the various yield statements:
\begin{center}
    $\mathsf{before}(p_1(j), p_2(j)) = \btrue$ \hspace{1em}
    $\mathsf{before}(p_2(j), p_3) = \btrue$ \hspace{1em}
    $\mathsf{before}(p_1(j), p_3) = \btrue$ 
    \\[1em]
    $\mathsf{before}(p_1(j), p_1(j')) = j > j'$ \hspace{2em}
    $\mathsf{before}(p_2(j), p_2(j')) = j > j'$
    \\[1em]
    $\mathsf{before}(p_1(j), p_2(j')) = j \geq j'$
\end{center}
In the case of $j = j'$, the 
output of $p_1(j)$ happens before the output of $p_2(j')$,
because $p_1$ is the first yield statement in the loop.
When $j > j'$, the output of $p_1(j)$ happens
before the output of $p_2(j')$ because the loop
is iterating in reverse order.

\paragraph{Backward For Loop Expansion.} The case of backward iterations adds a
new layer of complexity, namely to perform a non-reversible computation $s$ in
a reversed order: indeed, in the for loop $\hlforRev{(i,x)}{\ogen{s_1}}{s_2}$,
$s_1$ can contain the command $\hlsettrue{b}$ which cannot be reversed.

Let us consider as an example
$\hlforRev{(i,x)}{\ogen{s}}{\hlyield{x}}$, where the statement $s$ is defined
to print all elements of a list $u$ except the first one, namely:
\[ s \ :=\ \hlletboolean{b}{\hlfor{(j,y)}{ u }{\hlif{b}{\hlyield{y}}{\hlsettrue{b}}}}\] 
The semantics of $\hlforRev{(i,x)}{\ogen{s}}{\hlyield{x}}$ is to print all
elements of $u$ in reverse order, skipping the last loop iteration. To compute
this new statement, we will use the following \emph{trick} that can be traced
back to \cite[Lemma 8.1 and Figure 6, p. 68]{bojanczyk2018polyregular}: we will
use two versions of the statement $s$, the first one $s_\mathsf{rev}$, will be
$s$ where all boolean introductions are removed, if statements
$\hlif{e}{s_1}{s_2}$ are replaced by sequences $\hlseq{s_1}{s_2}$, every loop
direction is swapped, and every sequence of statements is reversed. Its
intended semantics is to reach all possible yield statements of $s$ in the
reversed order. In our case:
\begin{wrapfigure}{r}{0.5\textwidth}
    \centering
\begin{tikzpicture}[overlay, remember picture]
    \draw[fill=D4hint,draw=none] 
                     ([yshift=1em, xshift=-0.5em]pic cs:revloopexpstart) 
                     rectangle 
                     ([yshift=-1.3em, xshift=4.5em]pic cs:revloopexpend);

    \draw[fill=D3hint,draw=none] 
                     ([yshift=1em, xshift=-0.5em]pic cs:revloopexpinnerstart) 
                     rectangle 
                     ([yshift=-0.3em, xshift=4.5em]pic cs:revloopexpend);

    \draw[fill=D2hint,draw=none] 
                     ([yshift=1em, xshift=-0.5em]pic cs:revloopifstart)
                     rectangle 
                     ([yshift=-0.3em, xshift=4.4em]pic cs:revloopifend);

     \node[anchor=east,D4] at ([xshift=1.5em,yshift=2em]pic cs:revloopexpstart) {$s_{\mathsf{rev}}$};
     \node[anchor=west,D3,rotate=90] at ([xshift=11em]pic cs:revloopexpinnerstart) {$s$};
     \node[anchor=west,D2,rotate=90] at ([xshift=9em,yshift=-3em]pic cs:revloopifstart) {yield guard};
\end{tikzpicture}
\begin{align*}
    &\tikzmark{revloopexpstart}\mathsf{for}^{\leftarrow}~(j',y')~\mathsf{in}~ u ~\mathsf{do} \\
    &\quad \tikzmark{revloopexpinnerstart}\mathsf{for}^{\rightarrow}~(j, y)~\mathsf{in}~ u ~\mathsf{do} \\
    &\quad \quad \mathsf{let}~\mathsf{mut}~b = \mathsf{false}~\mathsf{in} \\
    &\quad \quad \mathsf{if} \; b \; \mathsf{then} \\
    &\quad \quad \quad \tikzmark{revloopifstart}\mathsf{if} \; j = j' \; \mathsf{then} \\
    &\quad \quad \quad \quad \mathsf{yield}~y\tikzmark{revloopifend} \\
    &\quad \quad \mathsf{else} \\
    &\quad \quad \quad b \leftarrow \mathsf{true}\tikzmark{revloopexpend} \\
\end{align*}
\caption{Backward for loop expansion.}
\label{fig:revloopexp}
\end{wrapfigure}
\begin{equation*}
    s_\mathsf{rev} \ := \ \hlforRev{(j',y')}{ u }{\hlyield{y'}}
\end{equation*}
Some yield statements are reachable in
$s_\mathsf{rev}$, but not when iterating over $s$ in reverse order.
To ensure that we only output correct elements,
we replace every $\hlyield{\cdot}$ statement in $s_{\mathsf{rev}}$ by a
copy of $s$, leading to the programs $s' = s_{\mathsf{rev}}[ \hlyield{\cdot} \mapsto s ]$.
In our case:
\begin{align*}
    s' \ :=\ \hlforRev{(j',y')}{ u }{ s }
\end{align*}
It is now possible to replace every yield statement in this new program
by a conditional check ensuring that the output would actually be 
produced by the original program $s$.
\begin{equation*}
    s'' = s'[ \hlyield{e} \mapsto \hlifnoelse{\bcomp{i}{=}{j'}}{\hlyield{e}}]
\end{equation*}
In our case, the final program is described in \cref{fig:revloopexp}.
This rewriting can be generalised to any program of the form
$\hlforRev{(i,x)}{\ogen{s_1}}{s_2}$ combining the construction illustrated here
with the one taking care of position variables in the case of forward loops.
