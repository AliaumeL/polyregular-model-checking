\section{Benchmarks}
\label{sec:benchmarks}

We implemented all the transformations expressed in this paper in a
\texttt{Haskell} program. In particular, we transformed \kl{high-level
for-programs} into \kl{first-order interpretations} which where then used to
verify \kl{first-order Hoare triples} using an SMT solver or the \kl{MONA}
program. In this section, we will discuss the practical aspects of this
implementation.

\paragraph{Methodology.} We selected a number of programs in the \kl{high-level
for-program} language and for each of them we computed its corresponding
\kl{simple for-program} and its corresponding \kl{first-order interpretation}.
To measure the complexity of these transformations, we associated to a
\kl{high-level for-program} the following parameters: its \emph{size} (number
of control flow statements), its \emph{loop depth} (the maximum number of
nested loops), and its \emph{boolean depth} (the maximum number of boolean
variables visible at any point in the program). We compute the same parameters
for the corresponding \kl{simple for-program}. In the case of \kl{first-order
interpretations}, we only compute its \emph{size} (number of nodes in the
formula) and its \kl{quantifier rank}. The results are presented in
\cref{tab:benchmarks}.

Then, we used several existing solvers to verify basic \kl{first-order Hoare
triples} for these programs. Because the goal of this paper was to showcase the
usage of the theory of \kl{polyregular functions}, we did not focus on
optimizing the input for a particular solver: all solvers (except \kl{MONA})
were given an input in the \texttt{SMT-LIB} format. We list in
\cref{tab:timings} the answer and the time it took for each solver to answer
the different queries. This table illustrates the complementarity of the
various solvers, and in particular, justifies that for some specific queries,
the \kl{MONA} program (which is complete for \kl{first-order logic on words})
does not provide an answer (either by timeout, or by memory exhaustion).

\begin{table}
    \caption{Benchmark results for the transformations. 
        Here \kl[for program]{FP} is a \kl{high-level program},
        \kl[simple for-program]{S.FP} is a \kl{simple for-program},
        and \kl[first-order interpretation]{$\FO$-I} is a \kl{first-order interpretation}.
        The columns \textbf{size}, \textbf{l.d.}, and \textbf{b.d.} stand for the size, loop depth, and boolean depth respectively. While the columns \textbf{size} and \textbf{q.r.} stand for the size and \kl{quantifier rank} of the 
        \kl{first-order interpretation} respectively.
    }
    \label{tab:benchmarks}
    \centering
    %          | high-level                         | low-level                         | interpretation
% filename | size | loop depth | boolean depth  | size | loop depth | boolean depth | size | qr 
% increase table column spacing to 2mm
\setlength{\tabcolsep}{2mm}
\begin{tabular}{l|rrr|rrr|rrr}
    \toprule
     & \multicolumn{3}{c|}{\kl[for program]{FP}} & \multicolumn{3}{c|}{\kl[simple for-program]{S.FP}} & \multicolumn{2}{c}{\kl[first-order interpretation]{$\FO$-I}} \\
    \textbf{filename} & \textbf{size} & \textbf{l.d.} & \textbf{b.d.} & \textbf{size} & \textbf{l.d.} & \textbf{b.d.} & \textbf{size} & \textbf{q.r.} \\
    \midrule
            \texttt{identity.pr} &
        $ 3 $ & $ 1 $ & $ 0 $ &
            $ 2    $ & $ 2  $ & $ 0 $  &
         $ 1 $ & $ 0 $  \\
            \texttt{reverse.pr} &
        $ 3 $ & $ 1 $ & $ 0 $ &
            $ 2    $ & $ 2  $ & $ 0 $  &
         $ 1 $ & $ 0 $  \\
            \texttt{subwords\_ab.pr} &
        $ 24 $ & $ 2 $ & $ 1 $ &
            $ 15    $ & $ 4  $ & $ 3 $  &
         $ 956 $ & $ 14 $  \\
            \texttt{map\_reverse.pr} &
        $ 36 $ & $ 2 $ & $ 1 $ &
            $ 18    $ & $ 4  $ & $ 1 $  &
         $ 285 $ & $ 5 $  \\
            \texttt{prefixes.pr} &
        $ 6 $ & $ 2 $ & $ 0 $ &
            $ 5    $ & $ 3  $ & $ 0 $  &
         $ 2 $ & $ 0 $  \\
            \texttt{get\_last\_word.pr} &
        $ 18 $ & $ 1 $ & $ 1 $ &
            $ 23    $ & $ 4  $ & $ 2 $  &
         $ 8553 $ & $ 15 $  \\
            \texttt{get\_first\_word.pr} &
        $ 22 $ & $ 1 $ & $ 1 $ &
            $ 5    $ & $ 2  $ & $ 0 $  &
         $ 103 $ & $ 4 $  \\
            \texttt{compress\_as.pr} &
        $ 12 $ & $ 1 $ & $ 1 $ &
            $ 12    $ & $ 3  $ & $ 2 $  &
         $ 209 $ & $ 10 $  \\
            \texttt{litteral\_test.pr} &
        $ 29 $ & $ 1 $ & $ 1 $ &
            $ 129    $ & $ 3  $ & $ 12 $  &
         $ 3.2 \times 10^4 $ & $ 82 $  \\
            \texttt{bibtex.pr} &
        $ 110 $ & $ 2 $ & $ 1 $ &
            $ 802    $ & $ 6  $ & $ 29 $  &
         $13.7 \times 10^6$ & $136$  \\
        \bottomrule
\end{tabular}

\end{table}

\begin{table}
    \caption{timings}
    \label{tab:timings}
    \centering
    % latex timings
% 
%      "Right": {
%        "bsmtFormula": {
%          "bsBoolDepth": 0,
%          "bsCtn": "",
%          "bsDepth": 6,
%          "bsSize": 89
%        },
%        "bsmtName": "test-bench/prefixes.pr",
%        "bsmtPost": "assets/Formulas/endsWithA.fo",
%        "bsmtPre": "assets/Formulas/endsWithA.fo",
%        "bsmtResponses": {
%          "AltErgoSingle": {
%            "contents": "32.51886 ms",
%            "tag": "SmtQQWithTime"
%          },
%          "Mona": { "contents": "22.72391 ms", "tag": "SmtOkWithTime" },
%          "SMTLibAltErgo": {
%            "contents": "20.94126 ms",
%            "tag": "SmtOkWithTime"
%          },
%          "SMTLibCVC5": { "contents": "69.29064 ms", "tag": "SmtOkWithTime" },
%          "SMTLibYices": { "contents": "6.820917 ms", "tag": "SmtQQWithTime" },
%          "SMTLibZ3": { "tag": "SmtTimeout" }
%        }
%      }
%
% name | pre | post | depth | size | mona | alt-ergo | yices | cvc5 | z3
% ---- | --- | ---- | ----- | ---- | ---- | -------- | ----- | ---- | --

\begin{tabular}{l|l|l|r|r|r|r|r|r|r}
    \toprul
\textbf{name} & \textbf{pre} & \textbf{post} & \textbf{depth} & \textbf{size} & \textbf{mona} &  \textbf{cvc5} & \textbf{z3} \\
\midrule
prefixes.pr & endsWithA.fo & endsWithA.fo & 6 & 89 & OK & 
OK & timeout \\
prefixes.pr & endsWithA.fo & containsAB.fo & 9 & 221 & KO & 
UK & KO \\
prefixes.pr & endsWithA.fo & false.fo & 2 & 11 & KO & 
UK & KO \\
prefixes.pr & endsWithA.fo & containsA.fo & 3 & 39 & OK & 
OK & OK \\
prefixes.pr & endsWithA.fo & true.fo & 2 & 11 & OK & 
OK & OK \\
prefixes.pr & endsWithA.fo & startsWithA.fo & 6 & 89 & KO & 
timeout & timeout \\
prefixes.pr & endsWithA.fo & containsAA.fo & 9 & 221 & KO & 
UK & KO \\
get\_last\_word.pr & endsWithA.fo & endsWithA.fo & 23 & 18767 & KO & 
UK & KO \\
get\_last\_word.pr & endsWithA.fo & containsAB.fo & 27 & 28274 & timeout & 
UK & KO \\
get\_last\_word.pr & endsWithA.fo & false.fo & 2 & 11 & KO & 
UK & KO \\
get\_last\_word.pr & endsWithA.fo & containsA.fo & 19 & 9372 & KO & 
UK & KO \\
get\_last\_word.pr & endsWithA.fo & true.fo & 2 & 11 & OK & 
OK & OK \\
get\_last\_word.pr & endsWithA.fo & startsWithA.fo & 23 & 18767 & KO & 
UK & KO \\
get\_last\_word.pr & endsWithA.fo & containsAA.fo & 27 & 28274 & timeout & 
UK & KO \\
get\_first\_word.pr & endsWithA.fo & endsWithA.fo & 8 & 279 & KO & 
timeout & timeout \\
get\_first\_word.pr & endsWithA.fo & containsAB.fo & 10 & 447 & KO & 
UK & timeout \\
get\_first\_word.pr & endsWithA.fo & false.fo & 2 & 11 & KO & 
UK & KO \\
get\_first\_word.pr & endsWithA.fo & containsA.fo & 6 & 143 & KO & 
timeout & timeout \\
get\_first\_word.pr & endsWithA.fo & true.fo & 2 & 11 & OK & 
OK & OK \\
get\_first\_word.pr & endsWithA.fo & startsWithA.fo & 8 & 279 & KO & 
timeout & timeout \\
get\_first\_word.pr & endsWithA.fo & containsAA.fo & 10 & 447 & KO & 
UK & timeout \\
map\_reverse.pr & endsWithA.fo & endsWithA.fo & 13 & 801 & timeout & 
UK & timeout \\
map\_reverse.pr & endsWithA.fo & containsAB.fo & 17 & 1378 & UK & 
UK & KO \\
map\_reverse.pr & endsWithA.fo & false.fo & 2 & 11 & KO & 
UK & KO \\
map\_reverse.pr & endsWithA.fo & containsA.fo & 9 & 378 & OK & 
UK & timeout \\
map\_reverse.pr & endsWithA.fo & true.fo & 2 & 11 & OK & 
OK & OK \\
map\_reverse.pr & endsWithA.fo & startsWithA.fo & 13 & 801 & timeout & 
UK & timeout \\
map\_reverse.pr & endsWithA.fo & containsAA.fo & 17 & 1378 & UK & 
UK & KO \\
compress\_as.pr & endsWithA.fo & endsWithA.fo & 14 & 483 & OK & 
UK & timeout \\
compress\_as.pr & endsWithA.fo & containsAB.fo & 16 & 753 & KO & 
UK & timeout \\
compress\_as.pr & endsWithA.fo & false.fo & 2 & 11 & KO & 
UK & KO \\
compress\_as.pr & endsWithA.fo & containsA.fo & 12 & 245 & OK & 
UK & timeout \\
compress\_as.pr & endsWithA.fo & true.fo & 2 & 11 & OK & 
OK & OK \\
compress\_as.pr & endsWithA.fo & startsWithA.fo & 14 & 483 & KO & 
UK & timeout \\
compress\_as.pr & endsWithA.fo & containsAA.fo & 16 & 753 & KO & 
UK & timeout \\
subwords\_ab.pr & endsWithA.fo & endsWithA.fo & 22 & 2095 & timeout & 
UK & timeout \\
subwords\_ab.pr & endsWithA.fo & containsAB.fo & 26 & 3266 & timeout & 
UK & KO \\
subwords\_ab.pr & endsWithA.fo & false.fo & 2 & 11 & KO & 
UK & KO \\
subwords\_ab.pr & endsWithA.fo & containsA.fo & 18 & 1034 & timeout & 
UK & KO \\
subwords\_ab.pr & endsWithA.fo & true.fo & 2 & 11 & OK & 
OK & OK \\
subwords\_ab.pr & endsWithA.fo & startsWithA.fo & 22 & 2095 & timeout & 
UK & timeout \\
subwords\_ab.pr & endsWithA.fo & containsAA.fo & 26 & 3266 & timeout & 
UK & KO \\
prefixes.pr & containsAB.fo & endsWithA.fo & 6 & 99 & KO & 
timeout & timeout \\
prefixes.pr & containsAB.fo & containsAB.fo & 9 & 231 & OK & 
OK & timeout \\
prefixes.pr & containsAB.fo & false.fo & 3 & 21 & KO & 
UK & KO \\
prefixes.pr & containsAB.fo & containsA.fo & 3 & 49 & OK & 
OK & OK \\
prefixes.pr & containsAB.fo & true.fo & 3 & 21 & OK & 
OK & OK \\
prefixes.pr & containsAB.fo & startsWithA.fo & 6 & 99 & KO & 
timeout & timeout \\
prefixes.pr & containsAB.fo & containsAA.fo & 9 & 231 & KO & 
UK & timeout \\
get\_last\_word.pr & containsAB.fo & endsWithA.fo & 23 & 18777 & KO & 
UK & KO \\
get\_last\_word.pr & containsAB.fo & containsAB.fo & 27 & 28284 & timeout & 
UK & KO \\
get\_last\_word.pr & containsAB.fo & false.fo & 3 & 21 & KO & 
UK & KO \\
get\_last\_word.pr & containsAB.fo & containsA.fo & 19 & 9382 & KO & 
UK & KO \\
get\_last\_word.pr & containsAB.fo & true.fo & 3 & 21 & OK & 
OK & OK \\
get\_last\_word.pr & containsAB.fo & startsWithA.fo & 23 & 18777 & KO & 
UK & KO \\
get\_last\_word.pr & containsAB.fo & containsAA.fo & 27 & 28284 & timeout & 
UK & KO \\
get\_first\_word.pr & containsAB.fo & endsWithA.fo & 8 & 289 & KO & 
timeout & timeout \\
get\_first\_word.pr & containsAB.fo & containsAB.fo & 10 & 457 & KO & 
timeout & timeout \\
get\_first\_word.pr & containsAB.fo & false.fo & 3 & 21 & KO & 
UK & KO \\
get\_first\_word.pr & containsAB.fo & containsA.fo & 6 & 153 & KO & 
timeout & timeout \\
get\_first\_word.pr & containsAB.fo & true.fo & 3 & 21 & OK & 
OK & OK \\
get\_first\_word.pr & containsAB.fo & startsWithA.fo & 8 & 289 & KO & 
timeout & timeout \\
get\_first\_word.pr & containsAB.fo & containsAA.fo & 10 & 457 & KO & 
UK & timeout \\
map\_reverse.pr & containsAB.fo & endsWithA.fo & 13 & 811 & timeout & 
UK & timeout \\
map\_reverse.pr & containsAB.fo & containsAB.fo & 17 & 1388 & UK & 
UK & KO \\
map\_reverse.pr & containsAB.fo & false.fo & 3 & 21 & KO & 
UK & KO \\
map\_reverse.pr & containsAB.fo & containsA.fo & 9 & 388 & OK & 
UK & timeout \\
map\_reverse.pr & containsAB.fo & true.fo & 3 & 21 & OK & 
OK & OK \\
map\_reverse.pr & containsAB.fo & startsWithA.fo & 13 & 811 & timeout & 
UK & timeout \\
map\_reverse.pr & containsAB.fo & containsAA.fo & 17 & 1388 & UK & 
UK & KO \\
compress\_as.pr & containsAB.fo & endsWithA.fo & 14 & 493 & KO & 
UK & timeout \\
compress\_as.pr & containsAB.fo & containsAB.fo & 16 & 763 & OK & 
UK & timeout \\
compress\_as.pr & containsAB.fo & false.fo & 3 & 21 & KO & 
UK & KO \\
compress\_as.pr & containsAB.fo & containsA.fo & 12 & 255 & OK & 
UK & timeout \\
compress\_as.pr & containsAB.fo & true.fo & 3 & 21 & OK & 
OK & OK \\
compress\_as.pr & containsAB.fo & startsWithA.fo & 14 & 493 & KO & 
UK & timeout \\
compress\_as.pr & containsAB.fo & containsAA.fo & 16 & 763 & KO & 
UK & timeout \\
subwords\_ab.pr & containsAB.fo & endsWithA.fo & 22 & 2105 & timeout & 
UK & timeout \\
subwords\_ab.pr & containsAB.fo & containsAB.fo & 26 & 3276 & timeout & 
UK & timeout \\
subwords\_ab.pr & containsAB.fo & false.fo & 3 & 21 & KO & 
UK & KO \\
subwords\_ab.pr & containsAB.fo & containsA.fo & 18 & 1044 & timeout & 
UK & OK \\
subwords\_ab.pr & containsAB.fo & true.fo & 3 & 21 & OK & 
OK & OK \\
subwords\_ab.pr & containsAB.fo & startsWithA.fo & 22 & 2105 & timeout & 
UK & timeout \\
subwords\_ab.pr & containsAB.fo & containsAA.fo & 26 & 3276 & timeout & 
UK & KO \\
prefixes.pr & false.fo & endsWithA.fo & 6 & 81 & OK & 
OK & OK \\
prefixes.pr & false.fo & containsAB.fo & 9 & 213 & OK & 
OK & OK \\
prefixes.pr & false.fo & false.fo & 0 & 3 & OK & 
OK & OK \\
prefixes.pr & false.fo & containsA.fo & 3 & 31 & OK & 
OK & OK \\
prefixes.pr & false.fo & true.fo & 0 & 3 & OK & 
OK & OK \\
prefixes.pr & false.fo & startsWithA.fo & 6 & 81 & OK & 
OK & OK \\
prefixes.pr & false.fo & containsAA.fo & 9 & 213 & OK & 
OK & OK \\
get\_last\_word.pr & false.fo & endsWithA.fo & 23 & 18759 & OK & 
OK & OK \\
get\_last\_word.pr & false.fo & containsAB.fo & 27 & 28266 & OK & 
OK & OK \\
get\_last\_word.pr & false.fo & false.fo & 0 & 3 & OK & 
OK & OK \\
get\_last\_word.pr & false.fo & containsA.fo & 19 & 9364 & OK & 
OK & OK \\
get\_last\_word.pr & false.fo & true.fo & 0 & 3 & OK & 
OK & OK \\
get\_last\_word.pr & false.fo & startsWithA.fo & 23 & 18759 & OK & 
OK & OK \\
get\_last\_word.pr & false.fo & containsAA.fo & 27 & 28266 & OK & 
OK & OK \\
get\_first\_word.pr & false.fo & endsWithA.fo & 8 & 271 & OK & 
OK & OK \\
get\_first\_word.pr & false.fo & containsAB.fo & 10 & 439 & OK & 
OK & OK \\
get\_first\_word.pr & false.fo & false.fo & 0 & 3 & OK & 
OK & OK \\
get\_first\_word.pr & false.fo & containsA.fo & 6 & 135 & OK & 
OK & OK \\
get\_first\_word.pr & false.fo & true.fo & 0 & 3 & OK & 
OK & OK \\
get\_first\_word.pr & false.fo & startsWithA.fo & 8 & 271 & OK & 
OK & OK \\
get\_first\_word.pr & false.fo & containsAA.fo & 10 & 439 & OK & 
OK & OK \\
map\_reverse.pr & false.fo & endsWithA.fo & 13 & 793 & OK & 
OK & OK \\
map\_reverse.pr & false.fo & containsAB.fo & 17 & 1370 & OK & 
OK & OK \\
map\_reverse.pr & false.fo & false.fo & 0 & 3 & OK & 
OK & OK \\
map\_reverse.pr & false.fo & containsA.fo & 9 & 370 & OK & 
OK & OK \\
map\_reverse.pr & false.fo & true.fo & 0 & 3 & OK & 
OK & OK \\
map\_reverse.pr & false.fo & startsWithA.fo & 13 & 793 & OK & 
OK & OK \\
map\_reverse.pr & false.fo & containsAA.fo & 17 & 1370 & OK & 
OK & OK \\
compress\_as.pr & false.fo & endsWithA.fo & 14 & 475 & OK & 
OK & OK \\
compress\_as.pr & false.fo & containsAB.fo & 16 & 745 & OK & 
OK & OK \\
compress\_as.pr & false.fo & false.fo & 0 & 3 & OK & 
OK & OK \\
compress\_as.pr & false.fo & containsA.fo & 12 & 237 & OK & 
OK & OK \\
compress\_as.pr & false.fo & true.fo & 0 & 3 & OK & 
OK & OK \\
compress\_as.pr & false.fo & startsWithA.fo & 14 & 475 & OK & 
OK & OK \\
compress\_as.pr & false.fo & containsAA.fo & 16 & 745 & OK & 
OK & OK \\
subwords\_ab.pr & false.fo & endsWithA.fo & 22 & 2087 & OK & 
OK & OK \\
subwords\_ab.pr & false.fo & containsAB.fo & 26 & 3258 & OK & 
OK & OK \\
subwords\_ab.pr & false.fo & false.fo & 0 & 3 & OK & 
OK & OK \\
subwords\_ab.pr & false.fo & containsA.fo & 18 & 1026 & OK & 
OK & OK \\
subwords\_ab.pr & false.fo & true.fo & 0 & 3 & OK & 
OK & OK \\
subwords\_ab.pr & false.fo & startsWithA.fo & 22 & 2087 & OK & 
OK & OK \\
subwords\_ab.pr & false.fo & containsAA.fo & 26 & 3258 & OK & 
OK & OK \\
prefixes.pr & containsA.fo & endsWithA.fo & 6 & 84 & KO & 
timeout & timeout \\
prefixes.pr & containsA.fo & containsAB.fo & 9 & 216 & KO & 
UK & KO \\
prefixes.pr & containsA.fo & false.fo & 1 & 6 & KO & 
UK & KO \\
prefixes.pr & containsA.fo & containsA.fo & 3 & 34 & OK & 
OK & OK \\
prefixes.pr & containsA.fo & true.fo & 1 & 6 & OK & 
OK & OK \\
prefixes.pr & containsA.fo & startsWithA.fo & 6 & 84 & KO & 
timeout & timeout \\
prefixes.pr & containsA.fo & containsAA.fo & 9 & 216 & KO & 
UK & KO \\
get\_last\_word.pr & containsA.fo & endsWithA.fo & 23 & 18762 & KO & 
UK & KO \\
get\_last\_word.pr & containsA.fo & containsAB.fo & 27 & 28269 & timeout & 
UK & KO \\
get\_last\_word.pr & containsA.fo & false.fo & 1 & 6 & KO & 
UK & KO \\
get\_last\_word.pr & containsA.fo & containsA.fo & 19 & 9367 & KO & 
UK & KO \\
get\_last\_word.pr & containsA.fo & true.fo & 1 & 6 & OK & 
OK & OK \\
get\_last\_word.pr & containsA.fo & startsWithA.fo & 23 & 18762 & KO & 
UK & KO \\
get\_last\_word.pr & containsA.fo & containsAA.fo & 27 & 28269 & timeout & 
UK & KO \\
get\_first\_word.pr & containsA.fo & endsWithA.fo & 8 & 274 & KO & 
timeout & timeout \\
get\_first\_word.pr & containsA.fo & containsAB.fo & 10 & 442 & KO & 
UK & timeout \\
get\_first\_word.pr & containsA.fo & false.fo & 1 & 6 & KO & 
UK & KO \\
get\_first\_word.pr & containsA.fo & containsA.fo & 6 & 138 & KO & 
timeout & timeout \\
get\_first\_word.pr & containsA.fo & true.fo & 1 & 6 & OK & 
OK & OK \\
get\_first\_word.pr & containsA.fo & startsWithA.fo & 8 & 274 & KO & 
timeout & timeout \\
get\_first\_word.pr & containsA.fo & containsAA.fo & 10 & 442 & KO & 
UK & timeout \\
map\_reverse.pr & containsA.fo & endsWithA.fo & 13 & 796 & timeout & 
UK & timeout \\
map\_reverse.pr & containsA.fo & containsAB.fo & 17 & 1373 & UK & 
UK & KO \\
map\_reverse.pr & containsA.fo & false.fo & 1 & 6 & KO & 
UK & KO \\
map\_reverse.pr & containsA.fo & containsA.fo & 9 & 373 & OK & 
UK & timeout \\
map\_reverse.pr & containsA.fo & true.fo & 1 & 6 & OK & 
OK & OK \\
map\_reverse.pr & containsA.fo & startsWithA.fo & 13 & 796 & timeout & 
UK & timeout \\
map\_reverse.pr & containsA.fo & containsAA.fo & 17 & 1373 & UK & 
UK & KO \\
compress\_as.pr & containsA.fo & endsWithA.fo & 14 & 478 & KO & 
UK & timeout \\
compress\_as.pr & containsA.fo & containsAB.fo & 16 & 748 & KO & 
UK & KO \\
compress\_as.pr & containsA.fo & false.fo & 1 & 6 & KO & 
UK & KO \\
compress\_as.pr & containsA.fo & containsA.fo & 12 & 240 & OK & 
UK & timeout \\
compress\_as.pr & containsA.fo & true.fo & 1 & 6 & OK & 
OK & OK \\
compress\_as.pr & containsA.fo & startsWithA.fo & 14 & 478 & KO & 
UK & timeout \\
compress\_as.pr & containsA.fo & containsAA.fo & 16 & 748 & KO & 
UK & KO \\
subwords\_ab.pr & containsA.fo & endsWithA.fo & 22 & 2090 & timeout & 
UK & timeout \\
subwords\_ab.pr & containsA.fo & containsAB.fo & 26 & 3261 & timeout & 
UK & KO \\
subwords\_ab.pr & containsA.fo & false.fo & 1 & 6 & KO & 
UK & KO \\
subwords\_ab.pr & containsA.fo & containsA.fo & 18 & 1029 & timeout & 
UK & KO \\
subwords\_ab.pr & containsA.fo & true.fo & 1 & 6 & OK & 
OK & OK \\
subwords\_ab.pr & containsA.fo & startsWithA.fo & 22 & 2090 & timeout & 
UK & timeout \\
subwords\_ab.pr & containsA.fo & containsAA.fo & 26 & 3261 & timeout & 
UK & KO \\
prefixes.pr & true.fo & endsWithA.fo & 6 & 81 & KO & 
UK & KO \\
prefixes.pr & true.fo & containsAB.fo & 9 & 213 & KO & 
UK & KO \\
prefixes.pr & true.fo & false.fo & 0 & 3 & KO & 
UK & KO \\
prefixes.pr & true.fo & containsA.fo & 3 & 31 & KO & 
UK & KO \\
prefixes.pr & true.fo & true.fo & 0 & 3 & OK & 
OK & OK \\
prefixes.pr & true.fo & startsWithA.fo & 6 & 81 & KO & 
UK & KO \\
prefixes.pr & true.fo & containsAA.fo & 9 & 213 & KO & 
UK & KO \\
get\_last\_word.pr & true.fo & endsWithA.fo & 23 & 18759 & KO & 
UK & KO \\
get\_last\_word.pr & true.fo & containsAB.fo & 27 & 28266 & timeout & 
UK & KO \\
get\_last\_word.pr & true.fo & false.fo & 0 & 3 & KO & 
UK & KO \\
get\_last\_word.pr & true.fo & containsA.fo & 19 & 9364 & KO & 
UK & KO \\
get\_last\_word.pr & true.fo & true.fo & 0 & 3 & OK & 
OK & OK \\
get\_last\_word.pr & true.fo & startsWithA.fo & 23 & 18759 & KO & 
UK & KO \\
get\_last\_word.pr & true.fo & containsAA.fo & 27 & 28266 & timeout & 
UK & KO \\
get\_first\_word.pr & true.fo & endsWithA.fo & 8 & 271 & KO & 
UK & KO \\
get\_first\_word.pr & true.fo & containsAB.fo & 10 & 439 & KO & 
UK & KO \\
get\_first\_word.pr & true.fo & false.fo & 0 & 3 & KO & 
UK & KO \\
get\_first\_word.pr & true.fo & containsA.fo & 6 & 135 & KO & 
UK & KO \\
get\_first\_word.pr & true.fo & true.fo & 0 & 3 & OK & 
OK & OK \\
get\_first\_word.pr & true.fo & startsWithA.fo & 8 & 271 & KO & 
UK & KO \\
get\_first\_word.pr & true.fo & containsAA.fo & 10 & 439 & KO & 
UK & KO \\
map\_reverse.pr & true.fo & endsWithA.fo & 13 & 793 & timeout & 
UK & KO \\
map\_reverse.pr & true.fo & containsAB.fo & 17 & 1370 & UK & 
UK & KO \\
map\_reverse.pr & true.fo & false.fo & 0 & 3 & KO & 
UK & KO \\
map\_reverse.pr & true.fo & containsA.fo & 9 & 370 & KO & 
UK & KO \\
map\_reverse.pr & true.fo & true.fo & 0 & 3 & OK & 
OK & OK \\
map\_reverse.pr & true.fo & startsWithA.fo & 13 & 793 & timeout & 
UK & KO \\
map\_reverse.pr & true.fo & containsAA.fo & 17 & 1370 & UK & 
UK & KO \\
compress\_as.pr & true.fo & endsWithA.fo & 14 & 475 & KO & 
UK & KO \\
compress\_as.pr & true.fo & containsAB.fo & 16 & 745 & KO & 
UK & KO \\
compress\_as.pr & true.fo & false.fo & 0 & 3 & KO & 
UK & KO \\
compress\_as.pr & true.fo & containsA.fo & 12 & 237 & KO & 
UK & KO \\
compress\_as.pr & true.fo & true.fo & 0 & 3 & OK & 
OK & OK \\
compress\_as.pr & true.fo & startsWithA.fo & 14 & 475 & KO & 
UK & KO \\
compress\_as.pr & true.fo & containsAA.fo & 16 & 745 & KO & 
UK & KO \\
subwords\_ab.pr & true.fo & endsWithA.fo & 22 & 2087 & timeout & 
UK & KO \\
subwords\_ab.pr & true.fo & containsAB.fo & 26 & 3258 & timeout & 
UK & KO \\
subwords\_ab.pr & true.fo & false.fo & 0 & 3 & KO & 
UK & KO \\
subwords\_ab.pr & true.fo & containsA.fo & 18 & 1026 & timeout & 
UK & KO \\
subwords\_ab.pr & true.fo & true.fo & 0 & 3 & OK & 
OK & OK \\
subwords\_ab.pr & true.fo & startsWithA.fo & 22 & 2087 & timeout & 
UK & KO \\
subwords\_ab.pr & true.fo & containsAA.fo & 26 & 3258 & timeout & 
UK & KO \\
prefixes.pr & startsWithA.fo & endsWithA.fo & 6 & 89 & KO & 
timeout & timeout \\
prefixes.pr & startsWithA.fo & containsAB.fo & 9 & 221 & KO & 
UK & KO \\
prefixes.pr & startsWithA.fo & false.fo & 2 & 11 & KO & 
UK & KO \\
prefixes.pr & startsWithA.fo & containsA.fo & 3 & 39 & OK & 
OK & OK \\
prefixes.pr & startsWithA.fo & true.fo & 2 & 11 & OK & 
OK & OK \\
prefixes.pr & startsWithA.fo & startsWithA.fo & 6 & 89 & KO & 
timeout & timeout \\
prefixes.pr & startsWithA.fo & containsAA.fo & 9 & 221 & KO & 
UK & KO \\
get\_last\_word.pr & startsWithA.fo & endsWithA.fo & 23 & 18767 & KO & 
UK & KO \\
get\_last\_word.pr & startsWithA.fo & containsAB.fo & 27 & 28274 & timeout & 
UK & KO \\
get\_last\_word.pr & startsWithA.fo & false.fo & 2 & 11 & KO & 
UK & KO \\
get\_last\_word.pr & startsWithA.fo & containsA.fo & 19 & 9372 & KO & 
UK & KO \\
get\_last\_word.pr & startsWithA.fo & true.fo & 2 & 11 & OK & 
OK & OK \\
get\_last\_word.pr & startsWithA.fo & startsWithA.fo & 23 & 18767 & KO & 
UK & KO \\
get\_last\_word.pr & startsWithA.fo & containsAA.fo & 27 & 28274 & timeout & 
UK & KO \\
get\_first\_word.pr & startsWithA.fo & endsWithA.fo & 8 & 279 & KO & 
timeout & timeout \\
get\_first\_word.pr & startsWithA.fo & containsAB.fo & 10 & 447 & KO & 
UK & timeout \\
get\_first\_word.pr & startsWithA.fo & false.fo & 2 & 11 & KO & 
UK & KO \\
get\_first\_word.pr & startsWithA.fo & containsA.fo & 6 & 143 & OK & 
OK & OK \\
get\_first\_word.pr & startsWithA.fo & true.fo & 2 & 11 & OK & 
OK & OK \\
get\_first\_word.pr & startsWithA.fo & startsWithA.fo & 8 & 279 & OK & 
OK & timeout \\
get\_first\_word.pr & startsWithA.fo & containsAA.fo & 10 & 447 & KO & 
UK & timeout \\
map\_reverse.pr & startsWithA.fo & endsWithA.fo & 13 & 801 & timeout & 
UK & timeout \\
map\_reverse.pr & startsWithA.fo & containsAB.fo & 17 & 1378 & UK & 
UK & KO \\
map\_reverse.pr & startsWithA.fo & false.fo & 2 & 11 & KO & 
UK & KO \\
map\_reverse.pr & startsWithA.fo & containsA.fo & 9 & 378 & OK & 
UK & timeout \\
map\_reverse.pr & startsWithA.fo & true.fo & 2 & 11 & OK & 
OK & OK \\
map\_reverse.pr & startsWithA.fo & startsWithA.fo & 13 & 801 & timeout & 
UK & timeout \\
map\_reverse.pr & startsWithA.fo & containsAA.fo & 17 & 1378 & UK & 
UK & KO \\
compress\_as.pr & startsWithA.fo & endsWithA.fo & 14 & 483 & KO & 
UK & timeout \\
compress\_as.pr & startsWithA.fo & containsAB.fo & 16 & 753 & KO & 
UK & timeout \\
compress\_as.pr & startsWithA.fo & false.fo & 2 & 11 & KO & 
UK & KO \\
compress\_as.pr & startsWithA.fo & containsA.fo & 12 & 245 & OK & 
UK & timeout \\
compress\_as.pr & startsWithA.fo & true.fo & 2 & 11 & OK & 
OK & OK \\
compress\_as.pr & startsWithA.fo & startsWithA.fo & 14 & 483 & OK & 
UK & timeout \\
compress\_as.pr & startsWithA.fo & containsAA.fo & 16 & 753 & KO & 
UK & timeout \\
subwords\_ab.pr & startsWithA.fo & endsWithA.fo & 22 & 2095 & timeout & 
UK & timeout \\
subwords\_ab.pr & startsWithA.fo & containsAB.fo & 26 & 3266 & timeout & 
UK & KO \\
subwords\_ab.pr & startsWithA.fo & false.fo & 2 & 11 & KO & 
UK & KO \\
subwords\_ab.pr & startsWithA.fo & containsA.fo & 18 & 1034 & timeout & 
UK & KO \\
subwords\_ab.pr & startsWithA.fo & true.fo & 2 & 11 & OK & 
OK & OK \\
subwords\_ab.pr & startsWithA.fo & startsWithA.fo & 22 & 2095 & timeout & 
UK & timeout \\
subwords\_ab.pr & startsWithA.fo & containsAA.fo & 26 & 3266 & timeout & 
UK & KO \\
prefixes.pr & containsAA.fo & endsWithA.fo & 6 & 99 & KO & 
timeout & timeout \\
prefixes.pr & containsAA.fo & containsAB.fo & 9 & 231 & KO & 
UK & KO \\
prefixes.pr & containsAA.fo & false.fo & 3 & 21 & KO & 
UK & KO \\
prefixes.pr & containsAA.fo & containsA.fo & 3 & 49 & OK & 
OK & OK \\
prefixes.pr & containsAA.fo & true.fo & 3 & 21 & OK & 
OK & OK \\
prefixes.pr & containsAA.fo & startsWithA.fo & 6 & 99 & KO & 
timeout & timeout \\
prefixes.pr & containsAA.fo & containsAA.fo & 9 & 231 & OK & 
OK & timeout \\
get\_last\_word.pr & containsAA.fo & endsWithA.fo & 23 & 18777 & KO & 
UK & timeout \\
get\_last\_word.pr & containsAA.fo & containsAB.fo & 27 & 28284 & timeout & 
UK & KO \\
get\_last\_word.pr & containsAA.fo & false.fo & 3 & 21 & KO & 
UK & KO \\
get\_last\_word.pr & containsAA.fo & containsA.fo & 19 & 9382 & KO & 
UK & timeout \\
get\_last\_word.pr & containsAA.fo & true.fo & 3 & 21 & OK & 
OK & OK \\
get\_last\_word.pr & containsAA.fo & startsWithA.fo & 23 & 18777 & KO & 
UK & timeout \\
get\_last\_word.pr & containsAA.fo & containsAA.fo & 27 & 28284 & timeout & 
UK & KO \\
get\_first\_word.pr & containsAA.fo & endsWithA.fo & 8 & 289 & KO & 
timeout & timeout \\
get\_first\_word.pr & containsAA.fo & containsAB.fo & 10 & 457 & KO & 
UK & timeout \\
get\_first\_word.pr & containsAA.fo & false.fo & 3 & 21 & KO & 
UK & KO \\
get\_first\_word.pr & containsAA.fo & containsA.fo & 6 & 153 & KO & 
timeout & timeout \\
get\_first\_word.pr & containsAA.fo & true.fo & 3 & 21 & OK & 
OK & OK \\
get\_first\_word.pr & containsAA.fo & startsWithA.fo & 8 & 289 & KO & 
timeout & timeout \\
get\_first\_word.pr & containsAA.fo & containsAA.fo & 10 & 457 & KO & 
timeout & timeout \\
map\_reverse.pr & containsAA.fo & endsWithA.fo & 13 & 811 & timeout & 
UK & timeout \\
map\_reverse.pr & containsAA.fo & containsAB.fo & 17 & 1388 & UK & 
UK & KO \\
map\_reverse.pr & containsAA.fo & false.fo & 3 & 21 & KO & 
UK & KO \\
map\_reverse.pr & containsAA.fo & containsA.fo & 9 & 388 & OK & 
UK & timeout \\
map\_reverse.pr & containsAA.fo & true.fo & 3 & 21 & OK & 
OK & OK \\
map\_reverse.pr & containsAA.fo & startsWithA.fo & 13 & 811 & timeout & 
UK & timeout \\
map\_reverse.pr & containsAA.fo & containsAA.fo & 17 & 1388 & UK & 
UK & timeout \\
compress\_as.pr & containsAA.fo & endsWithA.fo & 14 & 493 & KO & 
UK & timeout \\
compress\_as.pr & containsAA.fo & containsAB.fo & 16 & 763 & KO & 
UK & timeout \\
compress\_as.pr & containsAA.fo & false.fo & 3 & 21 & KO & 
UK & KO \\
compress\_as.pr & containsAA.fo & containsA.fo & 12 & 255 & OK & 
UK & timeout \\
compress\_as.pr & containsAA.fo & true.fo & 3 & 21 & OK & 
OK & OK \\
compress\_as.pr & containsAA.fo & startsWithA.fo & 14 & 493 & KO & 
UK & timeout \\
compress\_as.pr & containsAA.fo & containsAA.fo & 16 & 763 & KO & 
UK & timeout \\
subwords\_ab.pr & containsAA.fo & endsWithA.fo & 22 & 2105 & timeout & 
UK & timeout \\
subwords\_ab.pr & containsAA.fo & containsAB.fo & 26 & 3276 & timeout & 
UK & KO \\
subwords\_ab.pr & containsAA.fo & false.fo & 3 & 21 & KO & 
UK & KO \\
subwords\_ab.pr & containsAA.fo & containsA.fo & 18 & 1044 & timeout & 
UK & timeout \\
subwords\_ab.pr & containsAA.fo & true.fo & 3 & 21 & OK & 
OK & OK \\
subwords\_ab.pr & containsAA.fo & startsWithA.fo & 22 & 2105 & timeout & 
UK & timeout \\
subwords\_ab.pr & containsAA.fo & containsAA.fo & 26 & 3276 & timeout & 
UK & KO \\
    \bottomrule
\end{tabular}

\end{table}

\paragraph{Observations on the Compilation Process.} Let us highlight that in
the simplest cases of
\cref{tab:benchmarks},
our compilation procedure is able to eliminate all boolean variables, thus
producing a \emph{quantifier-free} formula. This is the case for the
\texttt{identity.pr}, \texttt{reverse.pr} and \texttt{prefixes.pr} programs,
respectively computing the identity function, the word reversal function, and
the list of all prefixes.

We can observe that the \kl{boolean depth} of the \kl{simple for-program} is a
good indicator of the \kl{quantifier rank} of the generated \kl{first-order
interpretation}. Furthermore, we demonstrate the \kl{elimination of literals}
is responsible for a significant increase in the complexity of the formulas, as
one can see in the \texttt{assets/HighLevel/litteral\_test.pr} benchmark. This
is explained by the fact that the elimination of literals introduces a
\emph{sequential} structure in the program which is simulated by introducing
boolean variables.

We found it interesting that our compilation process is non-symmetric when
transforming \kl{high-level for-programs} into \kl{simple for-programs}: this
can be seen in the \texttt{get\_first\_word.pr} and
\texttt{get\_last\_word.pr}, leading to vastly different \kl{quantifier ranks}
for the resulting \kl{first-order interpretations}, although these programs are
very similar in their \kl{high-level representation}.


\paragraph{Observations on the Solver Performance.} While we did not use
extensive testing, we did witness that ... on the examples X and Y, results in
table XX. So the use of different solvers really is interesting.


