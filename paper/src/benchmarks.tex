% LTeX: language=EN
\section{Implementation}
\label{sec:benchmarks}

\AP We implemented all the transformations expressed in this paper in a
\texttt{Haskell} program. To measure the complexity of these transformations,
we associated to a \kl{high-level for-program} the following parameters: its
\emph{size} (number of control flow statements), its \intro{loop depth} (the
maximum number of nested loops), and its \intro{boolean depth} (the maximum
number of boolean variables visible at any point in the program). We compute
the same parameters for the corresponding \kl{simple for-program}. In the case
of \kl{first-order interpretations}, we only compute its \emph{size} (number of
nodes in the formula) and its \kl{quantifier rank}. This allowed us to
estimate the complexity of our transformations on a small set of programs that we present in
\cref{tab:benchmarks}. Then, we used several existing solvers to verify basic
\kl{first-order Hoare triples} for these programs. We
illustrate in \cref{tab:timings} the behaviour of the solvers on various
verification tasks, with a timeout of $5$ seconds for every solver.
These test offer only initial insight into the performance of our implementation,
so developing our implementation into an actual tool would require systematic 
benchmarks and comparison with already existing tools.

% Because the goal of this
% paper was to showcase the usage of the theory of \kl{polyregular functions}, we
% did not focus on optimizing the input for a particular solver: all solvers
% (except \kl{MONA}) were given an input in the \texttt{SMT-LIB} format.

\begin{table}[t]
    \caption{Results for the transformations. 
        Here \kl[for-program]{FP} is a \kl{for-program},
        \kl[simple for-program]{S.FP} is a \kl{simple for-program},
        and \kl[first-order interpretation]{$\FO$-I} is a \kl{first-order interpretation}.
        The columns \textbf{l.d.}, \textbf{b.d.} and \textbf{q.r.}
        stand respectively for the \kl{loop depth}, 
        \kl{boolean depth} and \kl{quantifier rank}.
    }
    \label{tab:benchmarks}
    \centering
    %          | high-level                         | low-level                         | interpretation
% filename | size | loop depth | boolean depth  | size | loop depth | boolean depth | size | qr 
% increase table column spacing to 2mm
\setlength{\tabcolsep}{2mm}
\begin{tabular}{l|rrr|rrr|rrr}
    \toprule
     & \multicolumn{3}{c|}{\kl[for program]{FP}} & \multicolumn{3}{c|}{\kl[simple for-program]{S.FP}} & \multicolumn{2}{c}{\kl[first-order interpretation]{$\FO$-I}} \\
    \textbf{filename} & \textbf{size} & \textbf{l.d.} & \textbf{b.d.} & \textbf{size} & \textbf{l.d.} & \textbf{b.d.} & \textbf{size} & \textbf{q.r.} \\
    \midrule
            \texttt{identity.pr} &
        $ 3 $ & $ 1 $ & $ 0 $ &
            $ 2    $ & $ 2  $ & $ 0 $  &
         $ 1 $ & $ 0 $  \\
            \texttt{reverse.pr} &
        $ 3 $ & $ 1 $ & $ 0 $ &
            $ 2    $ & $ 2  $ & $ 0 $  &
         $ 1 $ & $ 0 $  \\
            \texttt{subwords\_ab.pr} &
        $ 24 $ & $ 2 $ & $ 1 $ &
            $ 15    $ & $ 4  $ & $ 3 $  &
         $ 956 $ & $ 14 $  \\
            \texttt{map\_reverse.pr} &
        $ 36 $ & $ 2 $ & $ 1 $ &
            $ 18    $ & $ 4  $ & $ 1 $  &
         $ 285 $ & $ 5 $  \\
            \texttt{prefixes.pr} &
        $ 6 $ & $ 2 $ & $ 0 $ &
            $ 5    $ & $ 3  $ & $ 0 $  &
         $ 2 $ & $ 0 $  \\
            \texttt{get\_last\_word.pr} &
        $ 18 $ & $ 1 $ & $ 1 $ &
            $ 23    $ & $ 4  $ & $ 2 $  &
         $ 8553 $ & $ 15 $  \\
            \texttt{get\_first\_word.pr} &
        $ 22 $ & $ 1 $ & $ 1 $ &
            $ 5    $ & $ 2  $ & $ 0 $  &
         $ 103 $ & $ 4 $  \\
            \texttt{compress\_as.pr} &
        $ 12 $ & $ 1 $ & $ 1 $ &
            $ 12    $ & $ 3  $ & $ 2 $  &
         $ 209 $ & $ 10 $  \\
            \texttt{litteral\_test.pr} &
        $ 29 $ & $ 1 $ & $ 1 $ &
            $ 129    $ & $ 3  $ & $ 12 $  &
         $ 3.2 \times 10^4 $ & $ 82 $  \\
            \texttt{bibtex.pr} &
        $ 110 $ & $ 2 $ & $ 1 $ &
            $ 802    $ & $ 6  $ & $ 29 $  &
         $13.7 \times 10^6$ & $136$  \\
        \bottomrule
\end{tabular}

\end{table}

\begin{table}
    \caption{Verification of \kl{first-order Hoare triples} over 
        sample \kl{for-programs}. We specify the preconditions and postconditions
        as regular languages, writing $\mathcal{L}_{ab}$ as a shorthand
        for $\mathcal{D}^*ab\mathcal{D}^*$,
        and similarly for $\mathcal{L}_{aa}$, $\mathcal{L}_{ba}$, etc.
        In the columns corresponding to the solvers, a checkmark
        indicates a positive reply,
        a cross mark indicates a negative reply,
        and a question mark indicates a timeout
        or a memory exhaustion.
        We indicate the size and the \kl{quantifier rank} (\textbf{q.r.})
        of the \kl{first-order formulas} that are fed to the solvers.
    }
    \label{tab:timings}
    \centering
    % latex timings
% 
%      "Right": {
%        "bsmtFormula": {
%          "bsBoolDepth": 0,
%          "bsCtn": "",
%          "bsDepth": 6,
%          "bsSize": 89
%        },
%        "bsmtName": "test-bench/prefixes.pr",
%        "bsmtPost": "assets/Formulas/endsWithA.fo",
%        "bsmtPre": "assets/Formulas/endsWithA.fo",
%        "bsmtResponses": {
%          "AltErgoSingle": {
%            "contents": "32.51886 ms",
%            "tag": "SmtQQWithTime"
%          },
%          "Mona": { "contents": "22.72391 ms", "tag": "Smt\OKWithTime" },
%          "SMTLibAltErgo": {
%            "contents": "20.94126 ms",
%            "tag": "Smt\OKWithTime"
%          },
%          "SMTLibCVC5": { "contents": "69.29064 ms", "tag": "Smt\OKWithTime" },
%          "SMTLibYices": { "contents": "6.820917 ms", "tag": "SmtQQWithTime" },
%          "SMTLibZ3": { "tag": "SmtUK" }
%        }
%      }
%
% name | pre | post | depth | size | mona | alt-ergo | yices | cvc5 | z3
% ---- | --- | ---- | ----- | ---- | ---- | -------- | ----- | ---- | --
\setlength{\tabcolsep}{2mm}
\begin{tabular}{l|l|l|r|r|r|r|r}
    \toprule
\textbf{Name} & \textbf{Pre.} & \textbf{Post.} & \textbf{q.r.} & \textbf{size} & \textbf{MONA} &  \textbf{CVC5} & \textbf{Z3} \\
\midrule
\texttt{compress\_as.pr} & $\mathcal{L}_{ab}$ & $\mathcal{L}_{ab}$ & 16 & 763 & \OK & 
\UK & \UK \\
\texttt{reverse\_add\_hash.pr} & $\mathcal{L}_{ab}$ & $\mathcal{L}_{ba}$ & 9 & 380 & \UK & 
\OK & \UK \\
\texttt{get\_last\_word.pr} & $\mathcal{D}^*a$ & $\mathcal{L}_{aa}$ & 27 & 28274 & \UK & 
\UK & \KO \\
\texttt{subwords\_ab.pr} & $\mathcal{L}_{ab}$ & $\mathcal{L}_{ab}$ & 26 & 3276 & \textcolor{C3}{\UK} & 
\UK & \UK \\
\texttt{map\_reverse.pr} & $\mathcal{D}^*a$ & $a \mathcal{D}^*$& 13 & 801 & \UK & 
\UK & \UK \\
    \bottomrule
\end{tabular}

\end{table}

\paragraph{Compilation to $\FO$-formulas.}
Looking at \cref{tab:benchmarks}, we observe that the generated simple for programs 
have reasonable size and boolean depth. The generated \kl{first-order interpretations}
still have reasonable quantifier ranks, but their size grows significantly.
In the simplest cases of \cref{tab:benchmarks}, our compilation procedure is able
to eliminate all boolean variables, thus producing a \emph{quantifier-free}
formula. This is the case for \texttt{identity.pr}, \texttt{reverse.pr} and
\texttt{prefixes.pr}. Moreover, we observe that the \kl{boolean depth} of the
\kl{simple for-program} is a good indicator of the \kl{quantifier rank}
of the generated \kl{first-order interpretation}. Furthermore, the tests indicate that
\kl{elimination of literals} is responsible for a significant increase of the formulas 
size and quantifier rank (\texttt{literal\_test.pr} and \texttt{bibtex.pr}).  
This is explained by the fact that the elimination of literals introduces
(non-cyclic) counters, simulated by a number of boolean variables. Finally,
we observe that the size of the generated formulas differs significantly
for the programs \texttt{get\_first\_word.pr} and \texttt{get\_last\_word.pr}.
This is somewhat surprising, as the two programs are symmetric with respect to 
reversing the input words, and indicates some room for improvement in handling 
the reversed iteration.

\paragraph{Solver Performance.}
We can observe in \cref{tab:timings} that the different solvers are complementary.
This might seem surprising, as the \kl{MONA} solver is a complete decision procedure. 
However, since it solves a problem that is tower complete,
it is understandable that it underperforms the SMT solvers on some instances,
even though we use them with the undecidable \texttt{UFDTLIA} theory.
Let us justify this choice of the SMTLib theory: 
(a) \emph{Uninterpreted Functions} (UF) are used to represent the word, which 
is treated as a function from positions to characters,
(b) \emph{Data Types} (DT) is used to represent finite sets of tags and characters, 
(c) \emph{Linear Integer Arithmetic} (LIA) is used to deal with the order of the 
positions in the word. This choice might not be optimal, but we believe that it
is a good trade off between ease-of-use and performance for our proof-of-concept
implementation. We can also observe that no solvers was able to deal with 
\texttt{subwords\_ab.pr} and \texttt{map\_reverse.pr} within the $5$ seconds timeout.
Understanding the complexities that arise in those cases, might be helpful
for improving the performance of our implementation.