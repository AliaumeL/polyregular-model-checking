%! TEX program = pdflatex
% WARNING: this is a generated file.
%
% Please do not edit this file directly. 
% - If you want to update the medatata of the paper (title, authors, abstract), please
%   edit the `paper-meta.yaml` file in the root of the repository.
% - If you want to update the content of the paper, please edit the latex files
%   in the `src` directory.
% - If you want to update the template itself (e.g., change the layout), please
%   edit the `templates/lncs/lncs.tex` file instead.
\documentclass[runningheads]{llncs}

\usepackage[utf8]{inputenc}
\usepackage[T1]{fontenc}


\newif\iflongversion
\longversionfalse

\usepackage{lineno}
\nolinenumbers


% babel for language settings
\usepackage[english]{babel}

% microtype for better typography
\usepackage{microtype}


% math packages
\usepackage{amssymb,amsmath,stmaryrd,thmtools,upgreek}


% graphics packages
\usepackage{graphicx}
\usepackage[obeyclassoptions,mode=tex]{standalone}
\usepackage{tikz}
\usetikzlibrary{backgrounds}
\usetikzlibrary{shapes.geometric}
\usetikzlibrary{positioning}
\usetikzlibrary{automata}
\usetikzlibrary{tikzmark}
\usetikzlibrary{patterns}
\usetikzlibrary{arrows}
\tikzset{every state/.style={minimum size=1pt}}
\usepackage{tikz-cd}


% links inside the document
\usepackage{hyperref}
\usepackage[capitalise,noabbrev,nameinlink]{cleveref}
\usepackage[electronic,hyperref,xcolor,cleveref]{knowledge}
\knowledgeconfigure{notion}
\knowledge{notion}
 | kl-usage

% Tables 
\usepackage{booktabs}
\usepackage{varwidth}

% Packages for macro definitions
\usepackage{xparse}
\usepackage{xpatch}
\usepackage{tokcycle}
\usepackage{ifthen}

% Proof trees
\usepackage{bussproofs}

% Colors 
\usepackage{ensps-colorscheme}
% set nice colors for the hyperlinks of knowledge
\knowledgestyle{intro notion}{color=A5, emphasize}
\knowledgestyle{notion}{color=B1}

% we include whatever the user wants to include in the header

% we include libraries (tex files) usually written in the `lib` directory
\input{lib/pandoc}
\input{lib/aliaume}

% High Level For Programs Syntax and semantics
\NewDocumentCommand{\PVars}{}{\mathbb{V}_{\text{pos}}}
\NewDocumentCommand{\FunVars}{}{\mathbb{V}_{\text{fun}}}
\NewDocumentCommand{\BVars}{}{\mathbb{V}_{\text{bool}}}
\NewDocumentCommand{\OVars}{}{\mathbb{V}_{\text{out}}}

\NewDocumentCommand{\Bools}{}{\mathbb{B}}
\NewDocumentCommand{\OutputType}{O{\Sigma}}{\mathcal{C}_{#1}}

\NewDocumentCommand{\hlprogram}{}{\mathsf{Program}}
\NewDocumentCommand{\hlfun}{}{\mathsf{Fun}}
\NewDocumentCommand{\hlstmt}{}{\mathsf{Stmt}}
\NewDocumentCommand{\bexpr}{}{\mathsf{BExpr}}
\NewDocumentCommand{\oexpr}{}{\mathsf{OExpr}}
\NewDocumentCommand{\cexpr}{}{\mathsf{CExpr}}
\NewDocumentCommand{\aexpr}{}{\mathsf{AExpr}}

\NewDocumentCommand{\bBinOp}{}{\mathsf{BBin}}
\NewDocumentCommand{\pCmpOp}{}{\mathsf{PComp}}

\NewDocumentCommand{\hlif}{m m m}{\mathsf{if} \; #1 \; \mathsf{then} \; #2 \; \mathsf{else} \; #3}
\NewDocumentCommand{\hlyield}{m}{\mathsf{yield} \; #1}
\NewDocumentCommand{\hlreturn}{m}{\mathsf{return} \; #1}
\NewDocumentCommand{\hlletoutput}{m m m}{\mathsf{let} \; #1 = #2 \; \mathsf{in} \; #3}
\NewDocumentCommand{\hlletboolean}{m m}{\mathsf{let~mut} \; #1 = \bfalse \; \mathsf{in} \; #2}
\NewDocumentCommand{\hlsettrue}{m}{\; #1 \; \leftarrow \; \mathsf{true}}
\NewDocumentCommand{\hlfor}{m m m}{\mathsf{for}^{\rightarrow} \; #1 \; \mathsf{in} \; #2 \; \mathsf{do} \; #3}
\NewDocumentCommand{\hlforRev}{m m m}{\mathsf{for}^{\leftarrow} \; #1 \; \mathsf{in} \; #2 \; \mathsf{do} \; #3}
\NewDocumentCommand{\hlseq}{m m }{ #1 \; ; \; #2}


\NewDocumentCommand{\btrue}{}{\mathsf{true}}
\NewDocumentCommand{\bfalse}{}{\mathsf{false}}
\NewDocumentCommand{\bnot}{m}{\neg #1}
\NewDocumentCommand{\bbin}{ m m m }{#1 \; #2 \; #3}
\NewDocumentCommand{\bcomp}{m m m}{#1 \; #2 \; #3}
\NewDocumentCommand{\bapp}{m m}{\mathop{#1}(#2)}
\NewDocumentCommand{\bliteq}{m m}{#1 \; = \; #2}
\NewDocumentCommand{\bgen}{m}{\langle #1 \rangle}


\NewDocumentCommand{\ogen}{m}{\langle #1 \rangle}
\NewDocumentCommand{\cchar}{m}{\mathsf{char}(#1)}
\NewDocumentCommand{\clist}{m}{\mathsf{list}(#1)}
\NewDocumentCommand{\olist}{m}{\mathsf{list}(#1)}

\NewDocumentCommand{\hlfundef}{m m m}{\mathsf{def} \; #1(#2) \; \{ #3 \}}

\input{knowledges.kl}

\newcommand{\repositoryUrl}{\url{https://github.com/AliaumeL/polyregular-model-checking}}


\usepackage[firstpage]{draftwatermark} % free badge placement
\SetWatermarkAngle{0}
%%%%%%%%% Use only one of the four following blocks
% use this block if you received the the "reusable" badge
\SetWatermarkText{\raisebox{18.5cm}{
\hspace{0.1cm}
\href{https://doi.org/10.5281/zenodo.15210967}{\includegraphics{1-available}}
\hspace{8cm}
\includegraphics{3-reusable}
}}


\begin{document}
%
\title{Polyregular Model Checking}

\author{
        Aliaume Lopez\inst{1}\orcidID{0000-0002-4205-327X}%
    \thanks{Aliaume Lopez was supported by the Polish National Science Centre (NCN) grant ``Polynomial finite state computation'' (2022/46/A/ST6/00072).}%
     \and
        Rafał Stefański\inst{1}\orcidID{0000-0002-8439-4056}%
    \thanks{Rafał Stefański was supported by the European Research Council (ERC) under the European Union's Horizon 2020 innovation program (grant PROCONTRA-885666).}%
    }
\authorrunning{A. Lopez and R. Stefański}
\institute{University of Warsaw}

\date{\today}

%
\maketitle              % typeset the header of the contribution
%
\begin{abstract}
    We introduce a high-level language with Python-like syntax for string-to-string, polyregular, first-order definable transductions. This language features function calls, boolean variables, and nested for-loops. We devise and implement a complete decision procedure for the verification of such programs against a first-order specification. The decision procedure reduces the verification problem to the decidable first-order theory of finite words (extensively studied in automata theory), which we discharge using either complete tools specific to this theory (MONA), or to general-purpose SMT solvers (Z3, CVC5).
\end{abstract}

\klogo\ This document uses \href{https://ctan.org/pkg/knowledge}{knowledge}:
\kl[kl-usage]{notion} points to its \intro[kl-usage]{definition}. 

% Include the content of the paper
% LTeX: language=EN
\section{Introduction}
\label{sec:intro}

The goal of this paper is to define a programming language that is expressive
enough to capture a wide range of \emph{string-to-string} functions, and yet
simple enough to be amenable to formal verification. Specifically, we want to
be able to verify Hoare triples of the form
$\hoaretriple{P}{\texttt{code}}{Q}$, where $P$ and $Q$ are predicates and
\texttt{code} is a program, meaning that whenever the input satisfies property
$P$, the output of the program satisfies property $Q$.

\paragraph{Regularity preserving programs.}
\begin{itemize}
  \item we are interested in precondition and postcondition
    that are regular languages, which is a fairly general 
    framework.
  \item define regularity preserving.
  \item this is at the core of several decision procedures.
  \item one of them is path feasibility analysis, where 
    we are interested in compositions of regularity preserving functions.
  \item there are uncomputable functions that are
    regularity preserving, and one has to select a computation model.
\end{itemize}

\paragraph{String-to-string transducer models.}
\begin{itemize}
  \item While most automata models capture the same 
    class of languages (regular languages), there is zoo of models for string-to-string transducers
    in the litterature. 
  \item Notably, in POPL'11, X and Y have devised an effective pullback procedure
    for SSTs (linear regular functions, equivalent to 2DFTs), showing a PSPACE
    complete complexity.
  \item In POPL'19, they
    rely on the effective pullback procedure of
    linear regular functions. 

  \item In CAV'23, they use infinite input alphabets (also known as atoms/nominal sets/data words),
    and in this setting, the weaker class of rational functions is used 
    because full 2DFTs not regularity preserving in this nominal setting.

  \item Recently, the theory of \emph{polyregular functions} has gained a lot of traction.
    It goes beyond the previous models, by allowing polynomial growth of the output.
    One of the equivalent definitions of this model dates back to \cite{ENMA02}, and 
    several other characterizations have been proposed since then
    \cite{bojanczyk2018polyregular,bojanczyk2019string,bojanczyk2023growth}, 
    showing that it is a robust class of functions.
    It is known that they are regularity preserving [cite],

  \item However, this result is of theoretical nature
    (no implementation or complexity bounds are given), and writing programs using
    any of the existing equivalent definitions of this 
    model is cumbersome and error-prone. Furthermore, relying on monadic
    second order logic implies that one cannot use the vast majority of SMT
    solvers, which only handle first-order logic.

  \item Because polyregular functions are closed under compositions,
    one can reduce POPL'19 to a single model checking problem. Furthermore,
    polyregular fuctions contain linear regular functions. 

  \item Beware that \kl{simple for-programs} can encode any FO formula 
    with a linear blowup, hence their model checking is as hard as
    the emptiness problem for FO formulas, which is known to be
    TOWER-complete (stockmeyer).

  \item Let us also mention that the study of exponential growth functions is on the way.
\end{itemize}



\paragraph{MSO vs FO.} 
\begin{itemize}
  \item Usually, one considers monadic second order logic (MSO) to express properties of
    regular languages.
  \item In order to simplify the presentation, and allow for an encoding into
    SMTLib, we will use first-order logic (FO) instead.
  \item 
The correspondence between subsets of string to string
functions and first order logic dates back to the origins of automata theory
and the seminal results of \cite{PEPI86,SCHU65,MNPA71}, establishing the
equivalence between \emph{star-free languages}, \emph{first order definable
languages}, and \emph{counter free automata}.
  \item Extensions of this correspondence
to functions has been an active area of research \cite{CADA15,MUSC19}, which we
leverage in this work via the theory of \intro{polyregular functions}
\cite{ENMA02,bojanczyk2018polyregular,bojanczyk2019string,bojanczyk2023growth}.
\end{itemize}


\paragraph{Contributions.} 
\begin{itemize}
  \item In this paper our contributions are threefold.
  \item 
First, we introduce a programming language that corresponds to a rich subset of
\texttt{Python}, which we argue is expressive enough to be usable in practice.

\item Second, we demonstrate that this language can be compiled into a certain type
of \kl{polyregular functions}. 

\item Third, we prove that for these polyregular
functions, the verification of \kl{Hoare triples} (specified using
\kl{first-order logic} on words) effectively reduces to a satisfiability
problem of a first-order formula on finite words. 

\item While this last step was
known to be theoretically possible, an efficient and effective implementation
was lacking. 

\item Because we are using \kl{first-order logic} as a target language,
we are not restricted to using automata based tools like \intro{MONA}
\cite{MONA01}, but can also employ general purpose SMT solvers like \intro{Z3}
\cite{z3} and \intro{CVC5} \cite{cvc5}, generating proof
obligations in the \texttt{SMT-LIB} format \cite{BARRETT17}.

\item We implemented all of these conversions in a \texttt{Haskell} program, and
tested it on a number of examples with encouraging results.\footnote{An
anonymized version of our code is available at \repositoryUrl.}

\item That being said, we are not a tool paper, and the implementation should
  be seen as a proof of concept.

\item Our initial tests, while promising, are not fully-fledged benchmarks

\item more benchmarks are needed to assess the viability of our approach.

\item and we are aware that there is room for 
optimizations both in the implementation and the way encoding used
to discharge the proof obligations when using SMT solvers.

\end{itemize}

\hline

\paragraph{Outline.} The structure of the paper is as follows. We introduce our
\kl{high-level language} in \cref{sec:high-level}. In \cref{sec:polyregular},
we recall the theory of \kl{polyregular functions} by introducing them in terms
of \kl{simple for-programs} and \kl{$\FO$-interpretations}. We will also
provide an efficient reduction of the verification of Hoare triples to the
satisfiability of a \kl{first-order formula on words} in \cref{sec:pullback}.
In order to verify \kl{for-programs}, we compile them into \kl{simple
for-programs} in \cref{sec:htl}, and then compile \kl{simple for-programs} into
\kl{$\FO$-interpretations} in \cref{sec:low-level}. 
%These steps 
%are summarized in the following diagram:
%\begin{center}
%    \begin{tikzpicture}[
%        syntaxNode/.style={
%                    rectangle, draw, 
%                    text width=5em, 
%                    text centered, 
%                    rounded corners, 
%                    minimum height=4em}
%    ]
%        %
%        % Write a tikz picture with nodes explaining the different 
%        % steps of the rewriting system 
%        % 
%        % (a) high-level language
%        % (b) simple for-programs
%        % (c) first-order string-to-string interpretations
%        % (c1) precondition
%        % (c2) postcondition
%        % (d) first-order formula
%        \node[syntaxNode] (a) {\kl{For-program}};
%        \node[syntaxNode, right=of a] (b)  {\kl{Simple for-program}};
%        \node[syntaxNode, right=of b] (c)  {\kl{$\FO$ interpretation}};
%        \node[syntaxNode, right=of c] (d) {\kl{First-order formula}};
%
%        \draw[->] (a) -- node[above,rotate=90,xshift=2.3em,yshift=-0.3em] {\cref{sec:htl}} (b);
%        \draw[->] (b) -- node[above,rotate=90,xshift=2.3em,yshift=-0.3em] {\cref{sec:low-level}} (c);
%        \draw[->] (c) -- node[above,rotate=90,xshift=2.7em,yshift=-0.3em] {\cref{sec:pullback}} (d);
%    \end{tikzpicture}
%\end{center}
Then, in \cref{sec:benchmarks}, we present
benchmarks of our implementation on various examples, discussing
the complexity of the transformations and the main bottlenecks of our approach.
Finally, we conclude in \cref{sec:conclusion} by discussing potential
optimizations and future work.


% LTeX: language=EN
\section{High Level For Programs}
\label{sec:high_level}

\AP In this section, we introduce our \intro{high-level language} for
describing list-manipulating functions which can be seen as a subset of
\texttt{Python}. Our goal is to reason algorithmically about the
programs written in this language, so it needs to be highly restricted.
To illustrate those restrictions, let us present a comprehensive
example written in a subset of \texttt{Python}. 

\begin{figure}[h]
    \centering
    \begin{Shaded}
\begin{Highlighting}[numbers=left]
\KeywordTok{def}\NormalTok{ getBetween(l, i, j):}
    \CommentTok{""" Get elements between i and j """}
    \ControlFlowTok{for}\NormalTok{ (k, c) }\KeywordTok{in} \BuiltInTok{enumerate}\NormalTok{(l):}
        \ControlFlowTok{if}\NormalTok{ i }\OperatorTok{\textless{}=}\NormalTok{ k }\KeywordTok{and}\NormalTok{ k }\OperatorTok{\textless{}=}\NormalTok{ j:} \circled{1}{code:comparisons}
            \ControlFlowTok{yield}\NormalTok{ c} \circled{2}{code:yield}

\KeywordTok{def}\NormalTok{ containsAB(w):}
    \CommentTok{""" Contains "ab" as a subsequence """}
\NormalTok{    seen\_a }\OperatorTok{=} \VariableTok{False} \circled{3}{code:mutbool}
    \ControlFlowTok{for}\NormalTok{ (x, c) }\KeywordTok{in} \BuiltInTok{enumerate}\NormalTok{(w):}
        \ControlFlowTok{if}\NormalTok{ c }\OperatorTok{==} \StringTok{"a"}\NormalTok{:} \circled{4}{code:string:comp}
    \NormalTok{            seen\_a }\OperatorTok{=} \VariableTok{True} \circled{5}{code:settrue} 
        \ControlFlowTok{elif}\NormalTok{ seen\_a }\KeywordTok{and}\NormalTok{ c }\OperatorTok{==} \StringTok{"b"}\NormalTok{:}
            \ControlFlowTok{return} \VariableTok{True}
    \ControlFlowTok{return} \VariableTok{False}

\KeywordTok{def}\NormalTok{ subwordsWithAB(word):}
    \CommentTok{""" Get subwords that contain "ab" """}
    \ControlFlowTok{for}\NormalTok{ (i,c) }\KeywordTok{in} \BuiltInTok{enumerate}\NormalTok{(word):} \circled{6}{code:enumerate}
        \ControlFlowTok{for}\NormalTok{ (j,d) }\KeywordTok{in}\BuiltInTok{ reversed(}\BuiltInTok{enumerate}\NormalTok{(word)):} \circled{7}{code:enumerate-rev}
    \NormalTok{            s }\OperatorTok{=}\NormalTok{ getBetween(word, i, j)} \circled{8}{code:immutable:variable}
            \ControlFlowTok{if}\NormalTok{ containsAB(s):}
                \ControlFlowTok{yield}\NormalTok{ s}
\end{Highlighting}
\end{Shaded}


    \caption{A small Python program that
        outputs all subwords of a given word containing \texttt{ab}
        as a scattered subword}.
    \label{fig:python-example-nested}
\end{figure}

For the sake of readability, we implicitly coerce generators (created using the
\texttt{yield} keyword) to lists. Our programs will only deal with three kinds
of values: booleans ($\intro*\Bools$), non-negative integers ($\Nat$), and
\intro{(nested) words} ($\intro*\NestedWords$), i.e. characters
($\reintro*\NestedWords[0]$), words ($\NestedWords[1]$), lists of words
($\NestedWords[2]$), etc. 
These lists can be created by \emph{yielding} values in a loop, such
as in \circleref{2}{code:yield}. 
In order to ensure decidable \kl{model checking}, we
also will enforce the following conditions, which are satisfied in our example:
\begin{enumerate}[label=\vspace{1em} Restriction \roman*:, ref=Rest. \roman*]
    \item \textbf{Loop Constructions.}
        \label{item:loop-constructions}
        We only allow \texttt{for} loops iterating forward
        or backward over a list, as in 
        \circleref{6}{code:enumerate} and \circleref{7}{code:enumerate:rev}.
        In particular, \texttt{while} loops and recursive functions 
        are forbidden, which guarantees termination of our programs.

    \item \textbf{Mutable Variables.} 
        \label{item:mut-variables}
        The only mutable variables are booleans. The
        values of integer variables are introduced by the \texttt{for} loop
        as in \circleref{6}{code:enumerate},
        and their values are fixed during each iteration. Mutable integer
        variables could serve as unrestricted counters, resulting in
        undecidable \kl{model checking}. Similarly, we prohibit mutable list
        variables, as their lengths could be used as counters.
        However, we still allow the use of immutable
        list variables, as in \circleref{8}{code:immutable:variable}.

    \item \textbf{Equality Checks.}
        \label{item:equality-checks}
        We disallow equality
        checks between two \kl{nested words}, 
        unless one of them is a constant expression.
        This is what happens in point \circleref{4}{code:string:comp}
        of our \cref{fig:python-example-nested}.
        Without this restriction, \kl{model checking} would be undecidable
        (\cref{lem:umc-equality-nested-words}).
        
    \item \textbf{Integer Comparisons.} 
        The only allowed operations on integers
        are usual comparisons operators (equality, inequalities).
        However, we only
        allow comparisons between integers that are indices of the
        same list.
        Every integer is associated to a list expression.
        For instance, in points \circleref{6}{code:enumerate} and
        \circleref{7}{code:enumerate:rev} of our example, the variables
        $i$ and $j$ are associated to the same list variable \texttt{word}.
        Similarly, in for the comparison 
        of point \circleref{1}{code:comparisons} to be valid,
        the variables $k$,$i$, and $j$ should all be associated to the same 
        list variable $l$.

        To ensure this compatibility, we designed the following type system,
        containing Booleans, \kl{nested words} of a given depth
        (characters are of depth $0$), and integers associated to a list
        expression (the set of which is denoted by $\oexpr$, and will
        be defined later on):
        \begin{align*}
            \tau ::=~ \TBool
            ~\mid~ \TPos[o] 
            ~\mid~ \TOut[n] 
            \quad 
            n \in \Nat, \,
            o \in \oexpr
            \quad .
        \end{align*}
        These types can be inferred from the context,
        except in the case of function arguments, in which case
        we explicitly specify to which list argument integer variables
        are associated.

        Without this restriction, the equality predicate can be 
        defined between two lists, leading to an undecidable
        \kl{model checking problem}.


    \item \textbf{Variable Shadowing.} 
        \label{item:variable-shadowing}
        We disallow shadowing of variable names, as it could
          be used to forge the origin of integers, leading to unrestricted comparisons.
          \begin{verbatim}
            \input{programs/eq_from_idxc_shadowing.py}
          \end{verbatim} 

    \item \textbf{Boolean Arguments.}
        \label{item:boolean-arguments}
        Allowing functions to take boolean arguments
        would allow to forge the origin of integers,
        by considering the function \texttt{switch(l1, l2, b)} which
        returns either \texttt{l1} or \texttt{l2} 
        depending on the value of \texttt{b}.
        \begin{verbatim}
            \input{programs/eq_from_idxc_shadowing.py} 
        \end{verbatim} 

    \item \textbf{Boolean Updates.} 
        Boolean variables are initialized to \texttt{false}
        as in \circleref{3}{code:mutbool}, and
        once they are set to \texttt{true} as in 
        \circleref{5}{code:settrue},
        they cannot be reset to \texttt{false}. 
        We depart here from the semantics of Python by
        considering lexical scoping of variables; in
        particular a variable declared in a loop is not
        accessible outside this loop.

        This restriction allows us to reduce the \kl{model checking problem} to
        the satisfiability of a \kl{first order formula} on finite words.
        This problem is not only decidable but also solvable by well-engineered
        existing tools, such as automata-based solvers (e.g., MONA) and
        classical SMT solvers (e.g., Alt-Ergo, Z3, CVC5). Without this
        restriction, the problem would require the use the MSO logic on words
        which is still decidable but not supported by the SMT solvers. 

\end{description}


\paragraph{Formal Syntax and Typing.} 
In order to give a type to our programs, it remains to give a type 
to functions. Because every integer variable is associated to a list
expression, we simplify the syntax of function calls and definitions
by syntactically grouping positions and lists. In the type system,
this is reflected by the creation of a type for \emph{function arguments}
as follows:
\begin{align*}
    \mathsf{arg} ::=&~ (\TOut[n],\ell) & \ell \in \Nat \\
    \mathsf{fun} ::=&~ 
           \mathsf{arg}_1 \times \cdots \times \mathsf{arg}_k \to \TBool \\
    \mid&~ \mathsf{arg}_1 \times \cdots \times \mathsf{arg}_k \to \TOut[n] 
\end{align*}

For instance, the function \texttt{getBetweenIndicesBeforeStop(l, i, j)}
has type $(\TOut[2],2) \to \TOut[2]$, that is, we are given an input list $l$ together
with two pointers to indices of this particular list. 
Similarly, the function \texttt{containsAB(w)} has type $(\TOut[1], 0) \to \TBool$,
while the function \texttt{subwordsWithAB(word)} has type $(\TOut[1], 0) \to \TOut[2]$.

\AP The rest of the syntax of the high-level language is straightforward:
boolean expressions ($\bexpr$), constant expressions ($\cexpr$), list
expressions ($\oexpr$), and control statements ($\hlstmt$) are defined in
\cref{fig:bool-expr,fig:const-expr,fig:out-expr,fig:high-level-stmt}.
For readability, we distinguish boolean variables $\intro*\BVars$ ($b, p, q,
\dots$), position variables $\intro*\PVars$ ($i,j, \dots$), list variables
$\intro*\OVars$ ($x,y,u,v,w, \dots$), and function variables $\intro*\FunVars$
($f,g,h, \dots$). A \intro{high-level program} is a list of function
definitions together with a \emph{main} function. 

\AP We provide a full type system for the \kl{high-level language} in appendix,
that ensures non-shadowing and non-forgeability of the origin of integer
variables.

\begin{lemma}
    \label{lem:type-checking}
    Type checking of \kl{high-level programs} is decidable in linear time, and 
    inference can be performed in linear time.
\end{lemma}


\paragraph{Semantics.} While the semantics are standard, let us briefly go
through how \kl{high-level programs} are executed by providing a denotational
semantics. To that end, let us map types to their set of values:
$\intro*\semT{\TBool} = \Bools$, $\reintro*\semT{\TPos[o]} = \Nat$ for all $o
\in \oexpr$, and $\reintro*\semT{\TOut[n]} = \NestedWords[n]$. Then, for
arguments to functions, we let $\reintro*\semT{(\TOut[n],\ell)} =
\NestedWords[n] \times \Nat^\ell$ for all $\ell \in \Nat$,
$\semT{\mathsf{arg}_1 \times \cdots \times \mathsf{arg}_k \to \TBool}$ will be
defined as functions from $\semT{\mathsf{arg}_1} \times \cdots \times
\semT{\mathsf{arg}_k}$ to $\Bools$, and similarly for functions with output
type $\TOut[n]$ for some $n \in \Nat$.

\AP
The evaluation of boolean expressions ($\bexpr$), constant expressions
($\cexpr$), and \kl{list expressions} ($\oexpr$) pose no difficulty and are
described in \textbf{TODO}. For the statements, the key point of the semantics
is that the for loops are taken to be of the form
$\hlfor{(i,x)}{o}{s}$ (in the case of a forward iteration) and
$\hlforRev{(i,x)}{o}{s}$ (in the case of a backward iteration).
While the forward iteration is defined as expected, the backward iteration 
could be understood in two ways, assuming that $o$ evaluates to 
a list $(x_0, \dots, x_k)$:
\begin{itemize}
    \item Executing the statement $s$ for every pair 
        $(k, x_k), \dots, (0, x_0)$, that is,
        the Python fragment \texttt{for (i,x) in reversed(enumerate(l))}
        ;
    \item Executing the statement $s$ for every pair
        $(0, x_k), \dots, (k, x_0)$,
        that is, 
        the Python fragment \texttt{for (i,x) in enumerate(reversed(l))}.
\end{itemize}
In our example program \cref{fig:python-example-nested}, 
we used the first interpretation (see \circleref{7}{code:enumerate:rev}). In fact,
the second interpretation would allow us to define the equality predicate
between two input \kl{list expressions}, leading to an undecidable \kl{model checking problem}.

\AP Another technical decision we made is regarding \kl{list expressions} and
the evaluation of statements. If a function $f$ has output type $\TOut[0]$,
then it can only use the statement $\hlreturn{x}$ for some $x$ of type
$\TOut[0]$. However, when a function has an output type $\TOut[n]$ for some $n
> 0$, then it can use both $\hlyield{x}$ and $\hlreturn{y}$ to produce its
output, in which case the output will be the concatenation of all the yields
before the first return, concatenated with the return value.


\newpage
\section{Polyregular Functions}
\label{sec:polyregular}

Our strategy for showing that the \kl{high level language} has decidable model checking,
is to show that is expressive power is captured by the class of \intro{first-order polyregular functions}
introduced in \cite{bojanczyk2018polyregular}. In this section we provide its two equivalent characterizations 
relevant for this paper:
their equivalent characterizations
that are relevant for this paper: \kl{first-order (simple) for programs}\footnote{
    In \cite{bojanczyk2018polyregular} the authors use the term \kl{first-order for-programs}. 
    In this paper we use the word `simple' to differentiate it from \kl{high-level for programs}. 
} \cite[p. 19]{bojanczyk2018polyregular} and \kl{first-order (string-to-string) interpretations}
\cite[Definition 4]{bojanczyk2019string}.

To make the models suitable for large alphabets (such as the Unicode characters)
present them in a symbolic framework (related to \cite{d2017power}, but not exactly the same) explained in the next section.
The non-symbolic versions models are proven in \cite{bojanczyk2018polyregular} to be equivalent.
In section \ref{sec:low_level} we show that the \kl{simple first-order language}
can be translated to a \kl{first-order string-to-string transduction}.
We believe that the other inclusion should also hold (with a similar proof as the one in \cite{bojanczyk2018polyregular}),
but it is out of this paper's scope.

\subsection{Symbolic transductions}
Consider the following program, that swaps all \texttt{a}'s to \texttt{b}.
programs.swapatob.py
Even though it operates on the entire Unicode alphabet, it only differentiates between three types of characters:
\texttt{a}, \texttt{b} and the rest. In order to formalize this idea, model the Unicode alphabet as an infinite set $\mathcal{D}$,
and we say that a transduction $T : \mathcal{D}^* \to \mathcal{D}^*$ is supported by a finite subset $A \subseteq \mathcal{D}$ if
for every function $f: \mathcal{D} \to \mathcal{D}$ that does not touch elements of $A$ (i.e. for every $a \in A$,  $f^{-1}(a) = {a}$),
it holds that: 
\[ \forall_{w} T(f^*(w)) = f^*(T(w))\texttt{,} \]

In the next two subsections we define symbolic versions of \kl{first-order simple for-programs} and \kl{first-order string-to-string transductions} which 
define (some of the) finitely supported functions of type $\mathcal{D}^* \to \mathcal{D}^*$. 


\subsection{First-Order Simple For-Programs}
\intro{First-order simple for-programs} is a simple programming language that can be be seen
as a simplified\footnote{Actually, its the \kl{high-level language} that is an expanded version of the \kl{first-order simple for-programs}.}
version of the \kl{high-level language}.
The main difference is that the \kl{simple for-programs} do not treat the (nested) words as first-class citizens -- there are no variables 
of type $\text{word}$ and one cannot define functions. In particular, the for-loops can only iterate over the input word (or the reversed input word),
so keeping track of the origin of the index variables is no longer required. Here is an example (see \cite{bojanczyk2018polyregular}
for a more detailed description):
%\emph{programs/reverse_words.spr}
It is not hard to see that programs written in this language define programs $f : \mathcal{D}^* \to \mathcal{D}^*$ that is supported by the 
finite set of letters used in the program (in this case $\{ \}$). 

\subsection{First-Order String-To-String Transductions}

\AP
A \intro{symbolic first-order string to string interpretation} is a logical way of defining transductions. It consists of: 

\begin{enumerate}
\item A finite set $T$ of tags.
\item A finite set $A$ of letters from $\mathcal{D}$.
\item A function $\text{arity} : T \to \Nat$ assigning an arity to each tag.
\item An output function function $\text{out} : T \to A + \{1, \ldots, \text{arity}(t)\}$. 
\item A domain formula $\varphi_{\text{dom}}^t(x_1,\ldots,x_{\text{arity}(t)})$ for every tag $t \in T$.
      The free variables represent positions in the input word, and can be queried by the following predicates: 
      $x_i = x_j$, $x_i < x_j$, and $x_i =_L `a'$ for $a \in A$.
\item An order formula $\varphi_{\leq}^{t,t'}((x_1,\ldots,x_{\text{arity}(t)}),(y_1,\ldots,y_{\text{arity}(t')}))$ for every pair of tags $t,t' \in T$. 
      It can use the same basic predicates as the domain formula, and it has to define a total order on elements of $(t : T) \times \mathbb{N}^{\mathsf{arity}(t)}$.
\end{enumerate}

Observe that the order formula defines a relation on the set $(t : T) \times \mathbb{N}^{\mathsf{arity}(t)}$ of tags 
equipped with position tuples of the appropriate arity. We say that a \kl{first-order interpretation} is valid if
the order formula defines a total order. Every valid interpretation defines a function $f : \mathcal{D}^* \to \mathcal{D}^*$
that is supported by $A$. The output for a word $w$ is obtained by the following process:
\begin{enumerate}
    \item Let $P = \{1, \ldots, |w|\}$ be the set of positions in $w$, and take the set $T(P) = (t : T) \times P^{\text{arity}(t)}$
           of all tags from $T$ equipped with position tuples of the appropriate arity.
    \item Filter out the elements that do not satisfy the domain formula.
    \item Sort the remaining elements according to the order formula.
    \item Assign a letter to each element according to the output function: 
          In order to choose the letter for an element $t(p_1, \ldots, p_k)$, we look at the 
          output of $\text{out}(t)$: if it is a letter $a \in A$ we simply choose it,
          and if it is $i \in \{1, \ldots, k\}$ we choose the $p_i$-th letter of the input word.
\end{enumerate}

For example, here is a first-order that interpretation defines the function \texttt{swapAsToBs}. 
It has two tags \texttt{printB} and \texttt{copy}, both of arity $1$. 
The element $\mathtt{printB}(x)$ outputs the letter \texttt{'a'} and the tag $\mathtt{copy}(x)$
outputs the letter of $x$-th position of the input word:
\[
\begin{tabular}{ccc}
    $\text{out}(\mathtt{printB}) = \mathtt{b}$ & \ \ \ & $\text{out}(\mathtt{copy}) = 1$ \\
\end{tabular}
\]
The element $\mathtt{printA}(x)$ is present in the output if $x$ is labelled with the letter \texttt{b}
in the input, otherwise the element $\mathtt{copy}(x)$ is present:
\[
\begin{tabular}{ccc}
    $\varphi_{\text{dom}}^{\mathtt{printA}}(x) : x =_L \mathtt{b}$ & \ \ \  &$\varphi_{\text{dom}}^{\mathtt{copy}}(x) : x \neq_L \mathtt{b}$ \\
\end{tabular}
\]
Finally the tags are sorted by their positions, with ties resolved in favour of \texttt{printA}:
\[ 
\begin{tabular}{c|cc}
    $\varphi_{\leq}$ & $\mathtt{printA}(x_1)$ & $\mathtt{copy}(x_1)$ \\
    \hline
    $\mathtt{printA}(x_2)$ & $x_1 \leq x_2$ & $x_1 < x_2$ \\
    $\mathtt{copy}(x_2)$ & $x_1 \leq x_2$ & $x_1 \leq x_2$ \\
\end{tabular}
\]
% LTeX: language=EN
\section{From High Level to Low Level For Programs}
\label{sec:htl}

\AP In this section, we provide a compilation from high level to low level For
programs. One of the main differences between the two languages is the lack of
functions in the latter. To smoothen the conversion between the two, we
introduce a new constructon to the language, that are \intro{generator
expressions}, representing the evaluation of a function on a given input.
Namely, we add the constructor $\intro*\ogen{s}$, and $\intro*\bgen{s}$ respectively to the
syntax of \kl{list expressions} and \kl{boolean expressions}. 

\paragraph{Generator Expressions.} Let us briefly discuss the new typing rules
and semantics of these \kl{generator expressions}. The intended meaning of a
generator expression $\ogen{s}$ is to evaluate the statement $s$ in the current
context and collect its output. For instance, $\ogen{\hlreturn{x}}$ is
equivalent to $x$, and $\ogen{\hlseq{\hlyield{x}}{\hlyield{y}}}$ is equivalent
to $\olist{x,y}$. The semantics are similarly defined for \kl{boolean
expressions}. It should therefore not be surprising that if a
\kl[hl]{statement} has type $\TOut[n]$, then the \kl{generator expression}
$\ogen{s}$ also has type $\TOut[n]$ (and similarly for \kl{boolean
expressions}). However, there is one key ingredient in the semantics of such
expressions: when evaluating the statement $s$, we hide all boolean variables
from the \kl{evaluation context}. In particular,
$\hlletboolean{b}{\hlreturn{\bgen{ \hlreturn{b} }}}$ is an \emph{invalid
program}, because the variable $b$ is undefined in the context of the generator
expression $\bgen{b}$. We refer the reader to the appendix and
\cref{fig:generators} for the formal definitions of the typing rules and
semantics of these expressions. The key lemma implied by this design choice is
that the evaluation order for expressions is irrelevant, which heavily relies
on the fact that only boolean variables are mutable
(\ref{item:mut-variables}).

\begin{lemma}
    \label{lem:gen-indep-bool}
    For every \kl{list expression} $e$, \kl{boolean expression} $b$,
    variable $x_e \in \OVars$, variable $x_b \in \BVars$, \kl{evaluation
    environment} $\rho$, and statement $s$, the following equality holds:
    \begin{equation*}
        \semS{ s[x_e \mapsto e, x_b \mapsto b] }{\rho}
        =
        \semS{ s }{\rho[x_e \mapsto \semO{e}{\rho}, x_b \mapsto \semB{b}{\rho}]}
    \end{equation*}
\end{lemma}

\paragraph{Rewriting Steps.} We will convert \kl{high level programs} to
\kl{low level programs} by a series of rewriting steps which we 
list hereafter, briefly explaining how they are performed.
Note that while most of the steps can be applied to any \kl{high level
program}, we will focus on programs
of type $(\TOut[1],0) \to (\TOut[1],0)$, since they ultimately will
be converted to \kl{simple for-programs}.
\begin{enumerate}[label=(\Alph*), ref=Step \Alph*]
    \item \label{item:lit_eq_elim} \intro{Elimination of Literal
        Equalities}, i.e., of expressions $\bliteq{c}{o}$ where $c \in \cexpr$
        and $o \in \oexpr$. This is done by replacing those tests with a call
        to a function that checks for equality with the constant $c$ by
        traversing its input. Such functions are defined in
        \cref{lem:constequality}.
        Note that this is only possible because equalities are always
        between a variable and a constant (\ref{item:equality-checks}).

    \item \label{item:lit_elim} \intro{Elimination of Literal
        Productions}, i.e., of constant expressions in the construction of
        $\oexpr$, except single characters. This is done by replacing a
        constant $c$ by a call to a function that produces the constant $c$.
        For instance, $\clist{\cchar{a_1}, \cchar{a_2}}$ is replaced by a call
        to a function with body
        $\hlseq{\hlyield{\cchar{a_1}}}{\hlyield{\cchar{a_2}}}$.


    \item \label{item:fun_elim} \intro{Elimination of Function Calls},
        by replacing them with \kl{generator expressions}. Given a function $f$
        with body $s$ and arguments $x_1, \dots, x_n$, we replace a call
        $f(a_1, \dots, a_n)$ by $\ogen{ s[ a_1/x_1, \dots, a_n/x_n ] }$
        (respectively, using $\bgen{ \cdots }$, for boolean functions). This is
        valid because function cannot take booleans as arguments
        (\ref{item:boolean-arguments}). A formal statement of this
        fact is available in \cref{lem:fungenexpr}.


    \item \label{item:bool_elim} \intro{Elimination of Boolean
        Generators}, i.e., of expressions $\bgen{s}$. Note that the only
        occurences of $\bgen{s}$ are in \kl{boolean expressions}, which are
        only used in the conditional tests. For instance, the conditional test
        $\hlif{\bgen{s_1}}{s_2}{s_3}$ is replaced by $\hlletboolean{b_1}{
        (\hlseq{s_1'}{\hlif{b_1}{s_2}{s_3}}) }$, where $s_1'$ is obtained by
        replacing boolean return statements $(\hlreturn{b)}$ by assignments of
        the form $(\hlif{b}{\hlsettrue{b_1}}{})$. This is valid because of
        \cref{lem:gen-indep-bool}.

    \item \label{item:let_output_elim} \intro{Elimination of Let
        Output Statements}, i.e., of statements of the form
        $\hlletoutput{x}{e}{s}$. This is done by textually replacing
        $\hlletoutput{x}{e}{s}$ by $s[x \mapsto e]$, which is valid because of
        \cref{lem:gen-indep-bool}.


    \item \label{item:return_elim} \intro{Elimination of Return
        Statements} for \kl{list expressions}. The idea is to replace returns
        by yield statements, and add a boolean variable \texttt{has\_returned}
        to prevent further outputs after the first return statement is reached.
        This works for expressions of type $\TOut[n]$ where $n > 0$, because
        one can replace $\hlreturn{e}$ by
        $\hlif{\bnot{\texttt{has\_returned}}}{\hlseq{\hlsettrue{\texttt{has\_returned}}}{s'}}$,
        where $s' = \hlfor{(i,x)}{e}{\hlyield{x}}$. The conversion is slightly
        more involved for returns of type $\TOut[0]$, and we refer the readers
        to our implementation for the details of this case. Its completeness
        relies on the fact that the final program is of type $(\TOut[1],0) \to
        (\TOut[1],0)$.


    \item \label{item:for_loop_exp} \intro{Expansion of For Loops},
        i.e., ensuring that every for loop iterates over a single \kl{list
        variable}. The most complex case being
        $\hlforRev{(i,x)}{\ogen{s_1}}{s_2}$. This rewriting step will be
        thouroughly explained in the upcoming section.

    \item \label{item:let_bools_top} 
        \intro{Defining booleans at the beginning of for loops}.
        Because there is no shadowing of variable
        names (\ref{item:variable-shadowing}),
        the following programs are equivalent:
        $(\hlseq{s_1}{\hlletboolean{b}{s_2}})$
        and
        $\hlletboolean{b}{(\hlseq{s_1}{s_2})}$. Similarly, for 
        the $\hlif{\cdot}{\cdot}{\cdot}$ construct.
        As a consequence, one can always move boolean definitions to the 
        first for loop in which they are defined.
\end{enumerate}

\begin{lemma}
    \label{lem:constequality}
    For every
    constant expression $c \in \cexpr$ of type $\TOut[n]$, there exists a \kl{high
    level function} $f_c$ of type $(\TOut[n], 0) \to \TBool$ such that
    $\semB{f_c(o)}{\rho} = \semB{\bliteq{c}{o}}{\rho}$ for every $o \in \oexpr$ and
    \kl{evaluation environment} $\rho$. Such functions are defined by induction on
    $c$, using a simple pattern matching.
\end{lemma}


\begin{theorem}
    \label{thm:rewriting-termination}
    The rewriting steps (\ref{item:lit_eq_elim}--- \ref{item:let_bools_top})
    all terminate and preserve typing, and a fully normalized \kl{high level for program}
    of type $(\TOut[1],0) \to (\TOut[1],0)$
    satisfies the following properties:
    \begin{itemize}
        \item It only iterates over its input or the reverse of its input;
        \item It contains no \kl{generator expressions};
        \item All introduced variables are boolean and introduced at the top level
            or at the beginning of a for loop;
        \item There are no return statements, and the only
            yield are of type character and contain either a constant
            or a variable.
    \end{itemize}
\end{theorem}

Now, it is clear that this sub-language of for-programs corresponds
to the simple-for-programs described in the previous section, hence
we have effectively obtained the following corollary:

\begin{corollary}
    Every \kl{high level program} of type $(\TOut[1],0) \to (\TOut[1],0)$
    is (effectively) computable by a \kl{simple for-program}.
\end{corollary}

\begin{corollary}
    Every \kl{high level program} has \kl{polynomial growth},
    and the \kl{model checking problem} is decidable for 
    \kl{high level programs}.
\end{corollary}


\paragraph{Forward For Loop Expansion.} We now focus on the \kl{expansion of for
loops}, that is, \ref{item:for_loop_exp}. The case of forward
iterations is simpler and will illustrate a first difficulty. We whish to
replace a loop form $\hlfor{(i,x)}{\ogen{s_1}}{s_2}$ by the statement $s_1$
where every statement $\hlyield{e}$ is replaced by $s_2[x \mapsto e]$. This
rewriting is problematic because it leaves the variable $i$ undefined in $s_2$.
The key observation allowing us to circumvent this issue is that the variable
$i$ can only be used in \emph{comparisons} of the form $\bcomp{i}{\pCmpOp}{j}$
inside \kl{boolean expressions}. Because of our restriction on the usage of
\kl{position variables} (\ref{item:integer-comparisons}), $i$ and $j$
can only be compared if originate from the same \kl{list expression}. In
particular, this means that $i$ can only be compared to a variable $j$ that is
iterating over $\ogen{s_1}$. This means that one only needs to be able to order
the outputs of $s_1$ to effectively replace the comparisions using the variable
$i$.

One can recover the ordering between outputs of a statement $s_1$ by storing
the position of the $\hlyield{e}$ responsible for the output, together with the
values of all position variables introduced by $s$ at that point. Let us show
how this can be done in a simple example:
\begin{equation*}
    \hlseq{
    (\hlforRev{(j\tikzmark{yieldIndex},y)}{e}{
        (\hlseq{\tikzmark{yield1}\hlyield{y}}
               {\tikzmark{yield2}\hlyield{\cchar{a}}}
        )})
    }{\tikzmark{yield3}\hlyield{\cchar{b}}}
\end{equation*}
\begin{tikzpicture}[overlay, remember picture]
    % Y1 below = 0.5cm, left 0.3cm of pic cs:yield1
    \node (Y1) at ([yshift=-0.5cm, xshift=0.2cm]pic cs:yield1) {$p_1(j)$};
    \node (Y2) at ([yshift=-0.5cm, xshift=0.2cm]pic cs:yield2) {$p_2(j)$};
    \node (Y3) at ([yshift=-0.5cm, xshift=0.2cm]pic cs:yield3) {$p_3$};
    \node (YI) at ([yshift=0cm, xshift=-0.2cm]pic cs:yieldIndex) {};

    \draw[dashed, A2, thick] (Y1) edge[->, bend left] (YI);
    \draw[dashed, A2, thick] (Y2) edge[->, bend left] (YI);
\end{tikzpicture}
\vspace{1em}

In this example, there are three yield statements at
positions $p_1$, $p_2$ and $p_3$. We can compute
the happens (strictly) before relation between outputs 
of the various yield statements:
\begin{itemize}
    \item $\mathsf{happensBefore}(p_1(j), p_3) = \btrue$;
    \item $\mathsf{happensBefore}(p_2(j), p_3) = \btrue$;
    \item $\mathsf{happensBefore}(p_1(j), p_2(j')) =
        j \geq j'$.
\end{itemize}
In the case of $j = j'$, the 
output of $p_1(j)$ happens before the output of $p_2(j')$,
because $p_1$ is the first yield statement in the loop.
When $j > j'$, the output of $p_1(j)$ happens
before the output of $p_2(j')$ because the loop
is iterating in reverse order.

\paragraph{Backward For Loop Expansion.} The case of backward iterations adds a
new layer of complexity, mainly to perform a non-reversible computation $s$ in
a reversed order: indeed $s$ can contain the command $\hlsettrue{b}$ which
cannot be reversed in our language. Let us consider as an example
$\hlforRev{(i,x)}{\ogen{s}}{\hlyield{x}}$, where the statement $s$ is defined
to print all elements of a list $u$ except the first one, namely:
$\hlletboolean{b}{\hlfor{(j,y)}{ u }{\hlif{b}{\hlyield{y}}{\hlsettrue{b}}}}$.
The semantics of $\hlforRev{(i,x)}{\ogen{s}}{\hlyield{x}}$ is to print all
elements of $u$ in reverse order, skipping the last loop iteration. To compute
this new statement, we will use the following \emph{trick} that can be traced
back to \cite[Lemma 8.1 and Figure 6, p. 68]{bojanczyk2018polyregular}: we will
use two versions of the statement $s$, the first one $s_\mathsf{rev}$, will be
$s$ where all boolean introductions are removed, if statements
$\hlif{e}{s_1}{s_2}$ are replaced by sequences $\hlseq{s_1}{s_2}$, every loop
direction is swapped, and every sequence of statements is reversed. Its
intended semantics is to reach all possible yield statements of $s$ in the
reversed order. 
In our case, $s_\mathsf{rev}$ is defined as:
\begin{equation*}
    \hlforRev{(j',y')}{ u }{\hlyield{y'}}
\end{equation*}
Remark that there are some statements that are reachable in
$s_\mathsf{rev}$, but do not correspond to a production when iterating over $s$
in reverse order, because of the boolean variables and conditionals that we
discarded. To ensure that we only output correct elements,
we will replace every $\hlyield{\cdot}$ statement in $s_{\mathsf{rev}}$ by a
copy of $s$, leading to the program $s' = s_{\mathsf{rev}}[ \hlyield{\cdot} \mapsto s ]$.
In our case, this leads to the following program:
\begin{align*}
    \hlforRev{(j',y')}{ u }{ s }
\end{align*}
It is now possible to replace every yield statement in this new program
by a conditional check ensuring that the output would actually be 
produced by the original program $s$.
\begin{equation*}
    s'' = s'[ \hlyield{e} \mapsto \hlifnoelse{\bcomp{i}{=}{j'}}{\hlyield{e}}]
\end{equation*}
In our case, the final program is the following one:
\begin{align*}
    &\mathsf{for}^{\leftarrow}~(j',y')~\mathsf{in}~ u ~\mathsf{do} \\
    &\quad \mathsf{for}^{\rightarrow}~(j, y)~\mathsf{in}~ u ~\mathsf{do} \\
    &\quad \quad \mathsf{let}~\mathsf{mut}~b = \mathsf{false}~\mathsf{in} \\
    &\quad \quad \mathsf{if} \; b \; \mathsf{then} \\
    &\quad \quad \quad \mathsf{if} \; j = j' \; \mathsf{then} \\
    &\quad \quad \quad \quad \mathsf{yield}~y \\
    &\quad \quad \mathsf{else} \\
    &\quad \quad \quad b \leftarrow \mathsf{true}
\end{align*}
This rewriting can be generalised to any program of the form
$\hlforRev{(i,x)}{\ogen{s_1}}{s_2}$ combining the construction illustrated here
with the one taking care of position variables in the case of forward loops.

\section{Low Level For Programs and Interpretations}
\label{sec:low_level}

In this section we compile low level For programs to first order interpretations,
allowing the verification of Hoare triples over low level For Programs.

\section{Benchmarks}
\label{sec:benchmarks}

\begin{description}
    \item[High Level transformation]
        We compute for the output
        \begin{itemize}
            \item The time it took to transform the program (plot)
            \item The size of the output program (ratio)
            \item The nesting depth of for loops (ratio)
            \item The number of print statements (ratio)
        \end{itemize}
    \item[Low level to interpretation]
        We compute for the output
        \begin{itemize}
            \item The time it took to transform the program (plot)
            \item The size of the formula (ratio)
            \item The quantifier rank of the formula (ratio)
        \end{itemize}
    \item[Interpretation to SMT]
        We compute for the output
        \begin{itemize}
            \item The time it took to transform the program (plot)
            \item The size of the formula (ratio)
            \item The quantifier rank of the formula (ratio)
        \end{itemize}
\end{description}

% LTeX: language=EN
\section{Conclusion}
\label{sec:conclusion}

We have show that the theory of \kl{polyregular functions} can be used to
verify close to real-world programs, and have implemented a prototype tool that
can discharge simple verification goals to existing solvers. We believe that
there is potential for further investigations in this direction.

\paragraph{Optimizations.} The benchmarks indicate that one of the most
promising source of optimizations is managing the \kl{boolean depth} of the
generated \kl{simple for-programs} during compilation. This can be achieved by
post-compilation optimizations (constant propagation, dead code elimination),
or by improving the code generation mechanism itself, which are low-hanging
fruits for future work. One source of the boolean variables seems to be the
\kl{elimination of Literal Equality} step (\ref{item:lit_elim}), which could be
mitigated by adding explicit successor and predecessor predicates to the
language of \kl{simple for-programs}.

At the level of \kl{first-order interpretations}, we have identified several
directions for improving their efficiency. One optimization is computing the
sequential composition of programs in a way that minimizes the number of
quantified boolean variables. Similarly, there seems to be potential for
performing direct substitutions instead of quantifying over the variables in a
lot of cases. Finally, our current approach for handling loops introduces
universal quantifiers, whose number could be reduced by exploiting the
monotonicity of the state transformations.
    
\paragraph{Solver Integration.} There is a lot of potential for optimizing the
input and parameters of the solvers for our particular use-case. An interesting
research direction would be to reduce the verification problem to emptiness of
LTL formulas, allowing us to use LTL solvers such as SPOT \cite{SPOT}.

\paragraph{Modular Verification.} The benchmarks show that one of the main
bottlenecks of our approach is the expansion of loops (whether in the
translation to \kl{simple for-programs} or in the translation to
\kl{first-order interpretations}). For this reason, the ability to verify
statements of the form \texttt{for (i, e) in enumerate(f(x)) do s done}, based
on a specification of $f$ given as a Hoare triple, would be a significant
improvement. However, it remains unclear how to integrate such modular
verification in our current approach.

\paragraph{Language Design.} As mentioned in \cref{sec:high-level},
\kl{for-programs} extended with unrestricted booleans also enjoy a decidable
verification of Hoare triples. However, the verification algorithm uses of
monadic second-order logic (MSO) over words instead of first-order logic. While
this prohibits the use of traditional SMT solvers, this logic can be handled by
the \kl{MONA} solver, and it might be interesting to implement and benchmark
the unrestricted version of the language. 

Another interesting extension of the language would be to allow the use of
complex types, such as pairs and records. This would make the language closer
to real use cases such as configuration management and data processing. It
would require extending the specification language to structured data types,
bypassing the current limitation that we can only verify string-to-string
transformations.


\paragraph{Integration with Existing Tools.} It would be a natural next step to
integrate our tool inside frameworks for program verification or testing. This
could be by checking goals generated by a tool such as \texttt{Why3}
\cite{Why3}, or by verifying properties of Python programs using decorated
functions.


% Include acknowledgements
\section*{Acknowledgements}
We would like to thank Arnav Garg and Ojas Maheshwari for their participation in the early stage of this project.

% Include the bibliography
\bibliographystyle{splncs04}
\bibliography{papers.bib}

% If there are any appendices, we include them here.


\end{document}
